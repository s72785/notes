\documentclass[12pt,a4paper]{scrreprt}

\usepackage{amsmath,amssymb,mathrsfs,dsfont}
\usepackage[utf8]{inputenc}
\usepackage[ngerman]{babel}
\usepackage[colorlinks=false, pdfborderstyle={/S/U/W 1}]{hyperref}
\usepackage{lmodern}
\usepackage{comment}
\oddsidemargin0mm \evensidemargin3mm \textwidth150mm \textheight23cm
\parindent0mm \pagestyle{empty} \topmargin-1cm
\pagestyle{headings}

\begin{document} 
\begin{flushleft}
\href{gruening@informatik.htw-dresden.de}{Prof. Wolf-Eckart Grüning}
\end{flushleft}

\begin{center}
\large{\textbf{ Allg. Betriebswirtschaftslehre I}} \\
WS 2014/15 \end{center}

\begin{center}\large{\textbf{ Übung 3 }} \end{center}

\bigskip
\begin{enumerate}%[label=Aufgabe*]

	%1.
 	\item{Aufgabe:}
 	Sie wollen mit einem Ihrer Studienfreunde nach dem Abschluss ein Projekt starten. Bevor Sie damit
beginnen, machen Sie sich Gedanken über mögliche Rechtsformen der Gesellschaft. Die
Geschäftspartner bringen unterschiedliche Zielvorstellung mit

	\begin{tabular}{ll}
	Sie 
		& Haftung minimieren, \\
		& volle Mitarbeit, \\
		& finanzielle Beteiligung, \\
		& Gewinnbeteiligung und \\
		& keine Verlustbeteiligung. \\
\\
	Er
		& keine Mitarbeit, \\
		& finanzielle Beteiligung, \\
		& Gewinnbeteiligung, \\
		& das Unternehmen soll leicht verkäuflich sein, \\
		& persönliche Haftung wird übernommen und \\
		& Verlustbeteiligung wird in Kauf genommen.
	\end{tabular}

	\begin{enumerate} 
		\item Welche Gesellschaftsformen kommen in Frage?
		\item Wie könnte der Kompromiss aussehen? Erstellen Sie hierfür eine geeignete Übersicht der
möglichen Rechtsformen und der jeweiligen Zielerfüllung.
	\end{enumerate} 

	%2.
 	\item{Aufgabe:}

Seit längerem kann beobachtet werden, dass Unternehmen vermehrt Verbindungen eingehen. Worin
liegen die Gründe?

	%3.
 	\item{Aufgabe:}

Nennen Sie wesentliche Standortfaktoren für ein Unternehmen.

	%4.
 	\item{Aufgabe:}

Die Gesellschafter einer GmbH mit einem Eigenkapital in Höhe von 20 Mio. EUR wollen eine Investi-
tion von 20 Mio. EUR realisieren. Sie haben die Wahl zwischen Erhöhung des Stammkapitals und
Aufnahme eines Darlehens. Diskutieren Sie vor- und Nachteile beider Varianten.

	%5.
 	\item{Aufgabe:}

Welcher Zusammenhang besteht zwischen der Unsicherheit künftiger Einzahlungs-
und Auszahlungsvorgänge einerseits und der Liquiditätsvorsorge andererseits?

	%6.
 	\item{Aufgabe:}

Durch welche Maßnahmen lässt sich Überliquidität bzw. Unterliquidität ausgleichen?

	%7.
 	\item{Aufgabe:}

Die Maschinenbau GmbH hat zum 31.12.2012 folgende Bilanz aufzuweisen (in TEUR):

\begin{tabular}{|lr|lr|}
\hline \multicolumn{2}{|c|}{Aktiva} & \multicolumn{2}{|c|}{Passiva} \\\hline
Grundstücke & 180 & Eigenkapital & 90 \\\hline
Geschäftsausstattung & 30 & Rückstellungen & 60 \\\hline
Langfristige Darlehensforderungen & 50 & Darlehensverbindlichkeiten & 150 \\\hline
Warenvorräte & 50 & Kurzfristige Kreditverbindlichkeiten & 70 \\\hline
Forderungen aus LL & 40 & Verbindlichkeiten aus LL & 30 \\\hline
Zahlungsmittel & 30 &  \multicolumn{2}{|l|}{} \\\hline
 \multicolumn{2}{|r|}{400} & \multicolumn{2}{|r|}{400} \\\hline
\end{tabular}

\end{enumerate}

\end{document}
