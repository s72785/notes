\documentclass[12pt,a4paper]{scrreprt}

\usepackage{amsmath,amssymb,mathrsfs,dsfont}
\usepackage[utf8]{inputenc}
\usepackage[ngerman]{babel}
\usepackage[colorlinks=false, pdfborderstyle={/S/U/W 1}]{hyperref}
\usepackage{lmodern}
\usepackage{comment}
%\oddsidemargin0mm \evensidemargin3mm \textwidth150mm \textheight23cm
%\parindent0mm \pagestyle{empty} \topmargin-1cm
%\pagestyle{headings}
\newcommand{\Lsg}{\textbf{Lösung}\nobreak}

\begin{document} 
\begin{flushleft}
\href{mailto:gruening@informatik.htw-dresden.de}{Prof. Wolf-Eckart Grüning}
\end{flushleft}

\begin{center}
\large{\textbf{ Allg. Betriebswirtschaftslehre I}} \\
WS 2014/15 \end{center}

\begin{center}\large{\textbf{ Übung 5 }} \end{center}

\bigskip

\begin{enumerate}

%1
\item
Die WMW Werkzeugmaschinen GmbH beschäftigt 1.200 Mitarbeiter, davon 300 im Verwaltungsbereich und 900 Produktionsarbeiter. Der Betriebsrat schlägt vor, für alle Mitarbeiter gleitende Arbeitszeit einzuführen. Kernarbeitszeit zwischen 9.00 und 13.00 Uhr. Der Gleitzeitbereich soll von 6.30 Uhr bis 18.00 Uhr gehen. Die tägliche Arbeitszeit ist mit 8 Std. vorgegeben.

\begin{enumerate}

\item Welche Argumente bringt der Betriebsrat vor?

Motivationssteigerung d.
\begin{itemize}
\item Eigenverantworlichkeit
\item Flexibilität f. eigene Ziele
\item (Überstundenumfang sinkt)
\item Ausfallzeiten werden gesenkt
\item besseres Betriebsklima durch Leistungsbereitschaft
\item höhere Arbeitsproduktivität
\end{itemize}

\item Welche Entscheidung wird die Betriebsleitung treffen und warum?

Fertigung: 
\begin{itemize}
\item bei Schichtarbiet nicht machbar
\item hoher Grad der Arbeitsteilung, MA sind aufeinander angewiesen
\end{itemize}
Daher müssen die Mitarbeiter in der Fertigung gleichzeitig anwesend sein.

Verwaltung:
\begin{itemize}
\item Hier ist Gleitzeit grds. möglich
\item Mindestbesetzung während der gesamten Arbeitszeit zu definieren
\item Kernarbeitszeit 6 statt 4 Std.
\end{itemize}

\end{enumerate}

%2
\item
Die WMW Werkzeugmaschinen GmbH plant für das Jahr 2015 einen Produktionsausstoß im Wert von
60 Mio. EUR. Man geht von einer Jahresleistung von 60.000 EUR je Produktionsarbeiter aus. Zum
31.03.2014 sind 900 Produktionsarbeiter beschäftigt. Davon werden bis Jahresende 20 MA in
Altersrente gehen bzw. kündigen. Die 12 Auszubildenden werden nach bestandener Gesellenprüfung im
Sommer übernommen. Wie groß ist der zusätzliche Personalbedarf/Personalüberhang zum Jahresende 2014?

Sollbedarf = 60.000.000 EUR / 60.000 EUR/MA  = 1000 MA
1000 - ( 900 - 20 + 12*x ) MA = ~108 neue MA

%3
\item
Welche Möglichkeiten hat die WMW Werkzeugmaschinen GmbH, diesen Personalbedarf zu decken?

Überstunden
+Azubi/Werben
(+Schulung)
Neueinstellung (befr. o. unbefr.)
(+Rationalisierung)
Leiharbeiter
+Outsourcing
+streichen von unrentablen Produkten zugunsten der rentableren, freiwerdende Mitarbeiter weiterverwenden

%4
\item

\begin{enumerate}
\item Was bedeuten die Begriffe Job enlargement und Job enrichment?

job enlargement
= Quantitative Erweiterung des Arbeitsfeldes im Umfang der Arbeitsleistung

job enrichment
= Qualitative Erwieterung des Arbietsfeldes im Umfang der Kompetenzen, neues Niveau

\item Erläutern Sie diese Begriffe an selbst gewählten Beispielen des Wartungsingenieurs in unserem
IT-Systemhaus.
\end{enumerate}

%5
\item
Untersuchungen der Personalentwicklung in einem IT-Systemhaus ergaben, dass seit zwei Jahren die Fluktuation steigt.

\begin{enumerate}

\item Was ist Ihre erste Aktivität, nachdem Sie Kenntnis von dieser Entwicklung erhalten?

Interne Ursachenforschung für erhöhtes Kündigungsaufkommen (mit internen Zahlen eher möglich)
vs.
Erkundigungen über andere Firmen der Brange (schwer möglich, ggf. über Brangenverband oder eine preisintensive Studie)

\item Welche Ursachen kann ein hoher Anteil von Kündigungen durch den Angestellten haben?

schlechte Arbeitsbedingungen
fehlenden Aufstiegschancen
Lohnzahlungsverzug
Betriebs-/Arbeitsklima mangelhaft
altersbedingte Abgänge
Auslaufen befristeter Verträge
persönenbedingte Kündigungen

\end{enumerate}

\item 
Eine wichtige Aufgabe des Personalmanagements ist die Bestimmung der Lohn-/Gehaltshöhe. In
diesem Zusammenhang steht immer wieder das Problem der Lohngerechtigkeit.

\begin{enumerate}
\item Welche Schwierigkeiten sehen Sie bei der Bestimmung einer gerechten Vergütung?

- Gerechtigkeit ist subjektiv, Vergütung sollte möglichst nach objektiven Kriterien zustande kommen
- generell ist eine Bewertung zu häufig subjektiv geprägt
- Kriterien für die objektivierung sind strittig

- Anforderungen müssen bewertet werden
- Arbeitsergebnisse sollten einfließen, Vergleichbarkeit erneut erforderlich - z.B. reine Stück-Zahlen könnten wenig aussagekräftig sein
- Anteil am Gesamtprozess
- Ersetzbarkeit
- soziale Vergütung (Kinder, Pflege im Zusammenhang m. Betriebszugehörigkeit)
- Verhalten v. MA

\item Welche Faktoren beeinflussen die gerechte Vergütung?

- Schwierigkeitsgrad / Qualifikation
- Ersatzbarkeit: Verantwortungsbereich / Entscheidungskompetenz
- Berufserfahrung / Brangenkenntnis
- soziale Belange
- Brangendurchschnitt für gleichartige Beschäftigung

\end{enumerate}

\item Wie kann sich eine Laufbahnplanung für untere und mittlere Angestellte auf deren Leistung auswirken? Begründen Sie.

Hemmend
- wenn Zeitbezug besteht
Motivierend
- mat. Anreize, Gehaltssteigerung
- erreichbare Ziele klar vorgegeben sind
- jobenrichment
- Ansehen/Status
- Jobsicherheit

\item Worauf achten Sie besonders, wenn Sie Ihnen unterstellte Mitarbeiter beurteilen müssen?

- Kompetenz
- Leistungsbereitschaft und -erfüllung
- Integration in Betrieb / Anteil an positivem Betriebsklima

\end{enumerate}


\end{document}
