\begin{document}

\begin{enumerate}

\item %1.

totales Differential

f : \R^2 \rightarrow \R : f(x,y) = 5x^2y-4y^2 .

df(x,y) = f_x(x,y)*dx + f_y(x,y)*dy

"Wie ändert sich f(x,y) ungeföhr,wenn x -> x+dx und y->y+dy ?"

f(x,y) = 5x^2y-4y^2

df(x,y) = f_x + f_y

f_x(x,y) = (10xy) dx

f_y(x,y) = (5x^2-8y) dy

df(x,y) = (10xy) dx + (5x^2-8y) dy

\item %2.

y(p_1 , p_2 ) = 8p_1^{\frac{1}{4}} p_2^{frac{3}{4}}

\begin{enumerate}

\item totales Differential aufstellen %a

y(100,200) =  8*100^{\frac{1}{4}} *200^{frac{3}{4}} = 1345,43

p_1 = 100 + 1 = 101
p_2 = 200 - 2 = 198

\item ungefähre Änderung %b

dy(p_1 , p_2 ) = y_{p_1}(p_1,p_2) dp_1 + y_{p_2}(p_1,p_2) dp_2
= 8p_2^{\frac{3}{4}}*\frac{1}{4}p_1^{-\frac{3}{4}}*dp_1 + 8p_1^{\frac{1}{4}}*\frac{3}{4}*p_2^{-\frac{1}{4}}*dp_2
= 2p_2^{\frac{3}{4}}p_1^{-\frac{3}{4}}dp_1 + 6p_1^{\frac{1}{4}}p_2^{-\frac{1}{4}}dp_2

dy(100,200) = 2*200^{\frac{3}{4}}*100^{-\frac{3}{4}} * dp_1 + 6*100^{\frac{1}{4}}*200^{-\frac{1}{4}}*dp_2 = -6,73

\Rightarrow ungefährt 7 Einheiten weniger

\item exakte Änderung

y(100,200) - y(101,198) = -6,78

\end{enumerate}

\item verallgemeinerte Kettenregel  %3.

z = f(x(t),y(t))
z'(t) = f_x*x'(t) + f_y*y'(t)

\begin{enumerate}

\item %a

z=f(x(t),y(t))=x^2y + y^3

z'=2yx * 2t + (x^2*1+3y^2)*e^t

mit

x(t)=t^2
y(t)=e^t

z(t)=t^4e^t + e^3t
z'= \ldots

\item %b

z = f (x(t), y(t)) = (x − y)^2

mit

x(t) = 2 \cos t, y(t) = 2 \sin t

an der Stelle

t=\frac{\pi}{3}

z'=f_x * x'(t) + f_y * y'(t)
= 2*(x-y)^1*1 * 2(-\sin t) + 2*(x-y)^1*(-1) * 2 \cos t
= -4(x-y) \sin t -4(x-y) \cos t
= -4(x-y) (\sin t + \cos t)
= -4(2 \cos t - 2 \sin t)(\sin t + \cos t)
= -8( \cos t - \sin t)( \cos t + \sin t)
= -8( \cos^2 t - \sin^2 t)
= 4
\end{enumerate}

\item %4.

\begin{enumerate}

\item %a

y(x) gegeben durch

2x^2 + xy=3y^2 

y(x) ist implizit gegeben durch F(x,y(x)) 
y(x) = \ldots \text{geht nicht!}

implizite Ableitung von F(x,y(x))=0
\begin{comment}
1) Ableitung (! y ist eine Fkt in Abh. von x)
2) nach y'(x) umstellen
\end{comment}

2x^2+xy(x)-3y(x)^2 = 0 %mit Produktregel für a*y(x)
'(x) = 4x + 1y(x) + x*y'(x) - 3(2*y(x)^1 * y'(x)) = 0
-4x-y(x) = (y'(x))( x-6y(x) )
y'(x) = \frac{ -4x-y(x) }{ x-6y(x) }


\item %b

x^2+y^2 = 16

x^2+y(x)^2-16 = 0

2x + 2y(x)*y'(x) = 0
y'(x) = \frac{ -2x }{ 2y }

\end{enumerate}

\item %5.

f^{(n)}(x) = ?

f(x) = \ln(1+x)

f'(x)  = \frac{1}{1+x}*1
f''(x) = ( (1+x)^{-1} )'
 = -1(1+x)^{-2}*1
f'''(x) = ( (1+x)^{-1} )''
 = 2*(1+x)^-3*1
f^{(4)}(x) = -6(1+x)^{-4}*1

f^{(n)} = (-1)^{n-1} (n-1)! * (1+x)^{-n}

Beweis per (Vollst.) Induktion

Induktionsanfang:

n=1, ja.

Induktionsvoraussetzung:

für n \in \N

f^{(n)} = (-1)^{n-1} * (n-1)! * (1+x)^{-n}

Induktionsbehauptung:

gilt auch für n+1, d.h.

f^{(n+1)} = (-1)^{n} * n! * (1+x)^{-n+1}

Indiktionsbeweis:

f^{(n+1)}(x) = [f^{(n)}(x)]'
= (-1)^{n-1} * (n-1)! * (-n) * (1+x)^{-n-1} %(-1)*n=(-n)
= (-1)^{n-1} * (n-1)! * (1+x)^{-n}

q.e.d.


\end{enumerate}



\end{document}
