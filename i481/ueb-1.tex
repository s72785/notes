\item %1.

Mindestens eine Nullstelle ist Teiler des Absolutgliedes c in f(x) = ax + c

\item %a)

$
f(x) = x^4 + 10x^3 + 35x^2 + 50x + 24
$

Faktorisierung

1. Ausprobieren...

x_0 = -1, f(x_0) = 0
x_1 = -2, f(x_1) = 0
x_2 = -4, f(x_2) = 0
x_3 = 0, f(0) = 24

f(x) = (x+1)(x+2)(x+4)(g(x))
g(x) = \frac{ f(x) }{ (x+1)(x+2)(x+4) }
 = x^4 + 10x^3 + 35x^2 + 50x + 24 : x^3 + 7x^2 + 14^x + 8  = x + 3
 -(x^4 +  7x^3 + 14x^2 +  8x +  0)
 --------------------------------
     0 +  3x^3 + 21x^2 + 42x + 24
        -(3x^3 + 21x^2 + 42x + 24)
        -------------------------
             0 +     0 +   0 +  0 = 0 \Rightarrow x_3 = -3

\item %b

f(x) = x^4 + 3x^3 - 2x^2 - 12x - 8

x_0=-2, f(x_0)=0
x_1=2, f(x_1)=0
f(3) \neq 0
f(x) = (x-2)(x+2)(g(x)) = x^2-4(g(x))
g(x) = \frac{ x^4 + 3x^3 - 2x^2 - 12x - 8 }{ x^2-4 }
  x^4 + 3x^3 - 2x^2 - 12x - 8 : x^2 - 4 = x^2 + 3x + 2
-(x^4 +    0 - 4x^2)
--------------------
    0 + 3x^3 + 2x^2 - 12x - 8
      -(3x^3        - 12x )
      -----------------------
           0 + 2x^2 -   0 - 8
             -(2x^2       - 8)
             -----------------
                  0       + 0 = 0 \Rightarrow x^2 + 3x + 2 kann bis zu zwei NSt liefern: (x+2)(x+1)

\item 2.

\item %a

f(x) = \frac{ x^5 - x^3 + x - 1}{ x^2-1}
     = x^5 -x^3 +x -1 : x^2-1 = x^3 + \frac { x-1 }{ x^2-1 }
     -(x^5 -x^3)
     -----------------
         0 -  0 +x -1

\item %b

f(x) = \frac{ x^4 -8x^2 +16 }{ (x^2-3x-10)(x+1) }
= x^4 +   0 - 8x^2 + 0  +16 : x^3-2x^2-13x-10 = x +2 +\frac{ 9x^2 +36x +36 }{ x^3-2x^2-13x-10 } = Asymptote + Abstrand bzw. Rest
-(x^4 -2x^3 -13x^2 -10x)
  -------------
    0 +2x^3 + 5x^2 +10x +16
     -(2x^3 - 4x^2 -26x -20)
     -----------------------
          0 + 9x^2 +36x +36

