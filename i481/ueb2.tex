\documentclass[12pt,a4paper]{scrreprt}
\usepackage{tikz}
\usetikzlibrary{positioning,calc}
\usepackage{amsmath,amssymb,mathrsfs,dsfont}
\usepackage[utf8]{inputenc}
\usepackage[ngerman]{babel}
\usepackage[colorlinks=false, pdfborderstyle={/S/U/W 1}
]{hyperref}
\usepackage{stmaryrd}
\usepackage{comment}
\oddsidemargin0mm \evensidemargin3mm \textwidth150mm \textheight23cm
\parindent0mm \pagestyle{empty} \topmargin-1cm
\newcommand{\E}{\mathds{E}}
\newcommand{\V}{\mathds{V}}
\newcommand{\C}{\mathds{C}}
\newcommand{\N}{\mathds{N}}
\newcommand{\Z}{\mathds{Z}}
\newcommand{\R}{\mathds{R}}
\pagestyle{headings}


\begin{document}

%das muss ich alles können und noch mehr.. mehr üben!

\begin{enumerate}

\item %1. (7)

\begin{enumerate}

\item
Bestimme die Gleichung der Tangentialebene der folgenden Funktion

\[
f(x, y) = \sin(2x) - \cos(xy), x,y \in \R
\]

im Punkt $P=(0,1; -1)$ .

\item %b
Bestimmen Sie mit Hilfe des Newton-Verfahrens (3 Iterationen) einen Näherngswert für die kleinste positive Lösung der Gleichung

\[
x+1 = 2e^{-x}, x \in \R
\]

\end{enumerate}

\item %2. (5)
Gegeben sei die gebrochen-rationale Funktion

\[
f(x) = \frac{2(x^2-3x+2)}{x^2+x-6}, x \in \R, x \neq 2, x \neq -3
\].

Finden Sie alle Polstellen und alle Nullstellen von $f$ und ermitteln Sie das Verhalten von $f$ für $x \rightarrow \infty$
sowie für $x \rightarrow -\infty$. Skizzieren Sie anhand dieser Informationen den Graphen der Funktion $f$.

\item %3. (8)
Das Angebot $A$ (in Stunden pro Monat) an Arbeitskräften für die Spargelernte in einem Großbetrieb in
Brandenburg hängt ab vom gezahlten Arbeitslohn $p$ (in EUR pro Stunde) und richtet sich nach folgender Funktion:

\[
A(p) = 300p(16-p), 0<p<16
\].

\begin{enumerate}
\item Bei welchem Stundenlohn ergibt sich das höchste Angebot an Arbeitskräften pro Monat?
Welche Lohnsumme (d.h. der Arbeitslohn für alle Arbeitnehmer zusammen) wird dann pro Monat gezahlt?
\item bei welchem Stundenlohn ist die pro Monat gezahlte Lohnsumme maximal? Wie hoch ist die maximale Lohnsumme?
\end{enumerate}

\item %4. (7)
Ein Unternehmen vertreibt zwei Güter, deren Nachfrage von den Preisen der beiden Güter wie folgt abhängt:

\[
x_1(p_1,p_2) = 8 - 2p_1 + p_2, x_2(p_1, p_2) = 10+p_1-3p_2
\].

Dabei sind $p_i$ die Preise und $x_i$ die nachgefragten Mengen nach Gut $i$, $i = 1, 2$. Die Kostenfunktion ist gegeben durch

\[
K(x_1, x_2) = x_1^2+x_2^2
\].

\item %5. (10)
Ein Bergwerksunternehmen fördert zwei verschiedene Erzsorten $E_1$ und $E_2$. Aus einer Tonne $E_1$ können 0.1 Tonnen Aluminium und 0.6 Tonnen Zink gewonnne werden. Aus einer Tonne $E_2$ können 0.5 Tonnen Aluminium und 0.5 Tonnen Zink gewonnen werden. Die monatliche Verarbeitungskapazität beträgt für Erzsorte $E_1$ höchstens 400 Tonnen, für Erzsorte $E_2$ höchstens 180 Tonnen. An Produktions- und Verarbeitungskosten fallen für Erzsorte $E_1$ insgesamt 10\,000 EUR pro Tonne an und für Erzsorte $E_2$ insgesamt 100\,000 EUR pro Tonne. Aufgrund fester Lieferverträge müssen mindestens 100 Tonnen Aluminium und mindestens 200 Tonnen Zink produziert werden. Bei welchen Fördermengen Mengen von $E_1$ und $E_2$ sind die Produktions- und Verarbeitungskosten minimal?

Stellen Sie ein mathematisches Optimierungsproblem auf und lösen Sie es grafisch.

\item %6. (5)
Eine Ein-Produkt-Unternehmung produziert mit folgender Grenzkostenfunktion

\[
K'(x) = 3x^2 -8x +8, x>=0
\].

Bei einem Output von 10 ME betragen die Gesamtkosten 744 GE. Man ermittle die Gesamtkosten- und die Stückkostenfunktion.

\end{enumerate}

\end{document}
