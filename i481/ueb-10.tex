\begin{enumerate}

\item %1.

\begin{enumerate}

\item %a)

f(x,y) = 5-x^2-\frac{1}{2}y^2	unter NB: x+y?2, y=2-x
\Rightarrow
f(x)=5-x^2-\frac{1}{2}(2-x)^2
f'(x) = -2x-1(2-x)(-1)
=-2x+2-x
=2-3x=0
2=3x
x=\frac{2}{3}
y=2-\frac{2}{3}=\frac{4}{3}
f''(x)=-3<0 \Rightarrow Max bei (\frac{2}{3}, \frac{4}{3})

\item %b)

f(x,y) = x*y	unter NB: x^2+y^2=1, y=\pm \sqrt{1-x^2}

\begin{comment}
1. Nebenbed. nach 0 umstellen: x^2+y^2-1=0
2. Lagrange-Funktion: \mathcal{L}(x,y,z)=\underbrace{x*y}{f(x,y)} + \lamda(\underbrace{x^2+y^2-1}{NB})
3.
3.1 \mathcal{L}_x(x,y,z) = y+2 \lamda x = 0
\lamda = -\frac{y}{2x}
3.2 \mathcal{L}_y(x,y,z) = x + 2 \lamda y = 0
\lamda = -\frac{x}{2y}
3.3 \mathcal{L}_{\lamda}(x,y,z) = x^2 + y^2 -1 = 0
4.
3.1=3.2
\frac{y}{2x} = \frac{x}{2y}
2x^2=2y^2
x^2=y^2
y=\sqrt{x^2}=\pm x
5.
4. in 3.3
x^2 + x^2 -1 = 0
2y^2 = 1
y=\pm \sqrt{\frac{1}{2}}
x_1=+\sqrt{\frac{1}{2}}
x_2=-\sqrt{\frac{1}{2}}
\end{comment}

f(\sqrt{\frac{1}{2}},\sqrt{\frac{1}{2}})=f(-\sqrt{\frac{1}{2}},-\sqrt{\frac{1}{2}})=\frac{1}{2}

f(\sqrt{\frac{1}{2}},-\sqrt{\frac{1}{2}})=f(-\sqrt{\frac{1}{2}},\sqrt{\frac{1}{2}})=-\frac{1}{2}

\end{enumerate}


\item %2.

Kürzester Abstand zwischen P und Q \in e
\Rightarrow Abstand zweier Punkte ist Pythagoras: \sqrt{(x_1-x_2)^2 + (y_1+y_2)^2}

Zielfkt: \sqrt{(x-1)^2 + (y-0)^2 + (z-0)^2}

NB: z-x-y = 0

\mathcal{L}(x,y,z,\lamda)=\underbrace{ (x-1)^2 + y^2 + z^2 }{f(x,y,z)} + \lamda(\underbrace{z-x-y}{NB})
1. L_x = 2(x-1)-\lamda = 0
2. L_y = 2y-\lamda = 0, \lamda = 2y
3. L_z = 2z + \lamda = 0, \lamda = -2z
4. L_{lamda} = z-x-y

2=3
2y=-2z
z=-y

in 4
-2y-x=0
y=-\frac{x}{2}

in 2
2(-\frac{x}{2})-\lamda
-x=\lamda

in 1
x=\frac{2}{3}
y=-{x}{2}=-\frac{1}{3}
z=-y=\frac{1}{3}

\item %3.

NB: x+y=2.000

Zielfkt: A(x, y) = 3x^2 − 2xy − 10y + 1.000.000

A(x) = 3x^2 - 2x(2000-x)-10(2000-x)+1.000.000
=5x^2-3990x+980.000
A'(x)=10x-3990=0
10x=3990
x=399
y=2000-399=1601

\item %4.

Firma1 Kosten 6 EUR/h  N_1(p_1,p_2)=39-3p_1+3p_2
Firma2 Kosten 3 EUR/h  n_2(p_1,p_2)=15+4p_1-9p_2

G_1(p_1,p_2) = E(p_1,p_2)-K(p_1,p_2)=(39-3p_1+3p_2)p_1-(39-3p_1+3p_2)6
=(39-3p_1+3_p2)(p_1-6)
G_2(p_1,p_2) = (15+4p_1-9p_2)(p_2-3)
(G_1)_{p_1} =-3(p_1-6)+(39-3p_1+3p_2)1 = -6p_1+3p_2+57=0
(G_2)_{p_2} =-0(p_2-3)+(15+4p_1-9p_2)1=-18p_2+4p_1+42=0

p_1=12
p_2=5

\item %5.

G(p_1 , p_2 , p_3 ) = 800 − 3p_1 − 2p_2 − 5p_3

p_2=50
p_3=20
ges. p_1 mit NB: G>0 und G bez. p_1 elastisch

G(p_1) = 800 − 3p_1 − 200
G(p_1) = 600 − 3p_1

NB_1

600-3p_1 > 0
p_1 < 200

NB_2

Partielle Elastizität von G bez. p_1, immer erst allg. und dann einsetzen!

\epsilon_{G_{p_1}}(p_1,p_2,p_3) = \frac{ G_{p_1}(p_1,p_2,p_3) }{ G(p_1,p_2,p_3) } p_1
=p_1\frac{ -3 }{ 800 − 3p_1 − 2p_2 − 5p_3 }
\epsilon_{G_{p_1}}(p_1,50,20) =p_1\frac{ -3 }{ 600 − 3p_1 }
= \frac{ p_1 }{ -200 + p_1 }

Fallunterscheidung
elastisch: |\epsilon| > 1
unelastisch: |\epsilon| < 1
proportional: |\epsilon| = 1

\left|\frac{ p_1 }{ -200 + p_1 }\right| > 1 	NB_1: p_1 < 200
p_1>100

L: 100 < p_1 < 200

\end{enumerate}
