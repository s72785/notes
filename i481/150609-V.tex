\usepackage{csquotes}
\usepackage{ulem}

09.06.2015, Voß-Böhme

s. Blatt Lokales Extremum ohne Nebenbedingungen

\[
%
z=(x^2+y^2)e^{-x} \\
z_x = e^{-x}(2x-x^2-y^2) \\
z_{xx} = e^{-x}(2-4x-x^2-y^2) \\
z_{xy}=e^{-x}(-2y) \\
z_y = e^{-x} \cdot 2y \\
z_{yx} = -e^{-x} \cdot 2y \text{Beachte: } z_{yx} = z_{xy} \\ 
z_{yy} = e^{-x} \cdot 2\\
%
\text{stationäre Punkte:} (0,0), (2,0) \\
%
\text{det}\begin{pmatrix}
2 & 0 \\
0 & 2
\end{pmatrix} = 2 \cdot 2 - 0 \cdot 0 = 2 > 0 \\
\Rightarrow \text{ist lokales Extremum}, z_{xx}(0,0) \Rightarrow \text{Min.} \\
\text{det}\begin{pmatrix}
 -2e^{-2} & 0 \\
 0 & 
\end{pmatrix} = -2e^{-2} \cdot 2e^{-2} - 0 \cdot 0 = -4e^{-4} < 0 \Rightarrow \text{kein Extremum}
\]

10.6.2 Extrema mit Nebenbedingungen

Geg.: Funktion $z=f(x,y), (x,y) \in D < \R^2$ und eine \uline{Nebenbedingung (Zielfunktion)}

\[
\phi(x,y)=0
\]

Ges.: Alle Punkte $(x_1,y_1), \ldots ,(x_n,y_n)$, so dass $f(x,y) \underbrace[<=]{>=} f(x,y)$ für alle $(x,y)$ mit $\phi(x,y)=0$.

Lösung: mithilfe der \uline{Lagrange-Methode}: Statt Zielfunktion $f$ und Nebenbedingung $\phi(x,y)=0$
wird eine Hilfsfunktion, sog. \uline{Lagrangefkt.} $F(x,y,\lamda)=f(x,y)-\lamda\phi(x,y)$ betrachtet.

Bsp.

Zielfunktion: $f)A,K)=20A+10K \rightarrow$ min.

Nebenbedingung: $100A^{0,8}K^{0,2}=100000$, $\phi(A,K)=0$, $100A^{0,8}K^{0,2}-10000=0$

$\Rightarrow$

%(1)
Lagrange-Funktion $F(A,K,\lamda)=\underbrace{10A+10K}{f(A,K)} + \lamda\underbrace{(100A^{0,8}K^{0,2}-10000)}{\phi(A,K)}$

%(2)
F_A=20+\lamda 100 0,8 A^{-0,2} K^{0,2} = 0		%(i)
F_K=10 + \lamda 100 A^{0,8} 0,2 K^{-0,8} = 0	%(ii)
F_{\lamda} = 100 A^{0,8} K{0,2} - 10000 = 0		%(iii)

(i) \lamda 80(\frac{K}{A})^0,2 = -20
 \lamda = -\frac{1}{4}(\frac{K}{A})^0,2
(ii) \lamda 20(\frac{A}{K})^{0,8} = -10
\lamda = -\frac{1}{2}(\frac{K}{A})^{0,8}
(i)=(iii) \Rightarrow A=2K in (iii) einsetzen \Rightarrow Schnittpunkt
100 (2K)^{0,8} K^{0,2}=10000
2^{0,8} K = 100
K=\frac{100}{208}
A=2K=\frac{200}{208}

f(\frac{200}{208},\frac{100}{208}) = 2871,5, \lamda=-\frac{1}{4}(\frac{A}{K})^{0,2} = -\frac{1}{4}2^{0,2}=-0,29
Also stationärer Punkt bei A-114,87, K=57,43 (\lamda = -0,29), f(A,K)=2871,5 .

Bedeutung des \uline{Lagrange-Multiplikators} \lamda

Falls Nebenbed. $\phi(x,y)=0$, \quote{gelockert} wird in Nebenbedingung $\phi(x,y)-c$, dann wächst
\uline{extremaler Funktionswert} um den Wert $-\lamda c$ .

Bsp.: Wie verändern sich die minimalen Kosten (bzw. was sind die minimalen Kosten) für einen Output von 10100?

Statt nebenbed. $100A^{0,8} K^{0,2}-10000=0$ bzw. $100A^{0,8} K^{0,2}-10000=0$
jetzt  $100A^{0,8} K^{0,2}-10100$ bzw. $100A^{0,8} K ^{0,2} - 10000 = 100$
bereits klar: im Originalproblem minimale Kosten: $f(A,K)=2871,5$, $\lamda=-0,29$

\[
\Rightarrow -\lamda c = +0,29 \cdot 100 = +29 \text{d.h. minimale Kosten im veränderten Problem wachsen um 29~GE auf 2900,5~GE}
\]

10.7 Lineare Optimierung

\quote{Optimum} = Min./Max. unter Nebenbedingungen (Restriktionen).

\uline{Math. Modellierung}

(1) Entscheidungsvariablen definieren: x_1, x_2 \ldots Anbaufläche f. Erbsen/Möhren
(2) Zielfunktion festelegen: $200 x_1 + 300 x_2 \rightarrow $ max. (Gesamtgewinn maximieren).
(3) Restriktionen/Ressourcen-Beschränkung beschreiben: x_1,x_2 >= 0, x_1+x_2 <= 30 Morgen.
100 x_1 + 50 x_2 <= 2500 Zeitaufwand
x1 + 2x_2 <=50 Tage

Falls Zielfkt. und Restriktionen mit linearen Funktionen beschrieben sind, spricht man von einem \uline{Linearen Optimierungsproblem (LOP/LOA)}.
