\documentclass[12pt,a4paper]{scrreprt}
\usepackage{tikz}
\usetikzlibrary{positioning,calc}
\usepackage{amsmath,amssymb,mathrsfs,dsfont}
\usepackage[utf8]{inputenc}
\usepackage[ngerman]{babel}
\usepackage[colorlinks=false, pdfborderstyle={/S/U/W 1}
]{hyperref}
\usepackage{stmaryrd}
\usepackage{comment}
\oddsidemargin0mm \evensidemargin3mm \textwidth150mm \textheight23cm
\parindent0mm \pagestyle{empty} \topmargin-1cm
\newcommand{\E}{\mathds{E}}
\newcommand{\V}{\mathds{V}}
\newcommand{\C}{\mathds{C}}
\newcommand{\N}{\mathds{N}}
\newcommand{\Z}{\mathds{Z}}
\newcommand{\R}{\mathds{R}}
\pagestyle{headings}


\begin{document}

%das muss ich alles können und noch mehr.. mehr üben!

\begin{enumerate}

\item %1. (8)

\begin{enumerate}

\item
Bestimme die Gleichung der Tangente am Graphen der Funktion

\[
f(x) = e^{2x}\cos x
\]

\item %b
Bestimme den Grenzwert

\[
\lim_{x \rightarrow 0}\left. \frac{2x^2}{e^x-e^{-x}} \right. 
\]

mit Hilfe einer geeigneten Regel für die Grenzwertberechnung.

% 0 / v = 0
% was für eine  Regel?

\item
Ermitteln Die eine gebrochen-rationale Funktion mit den Eigenschaften

\begin{itemize}
\item Nullstellen bei 1 und -1 %x_x={1,-1}
\item Polstelle bei 0 %x=0
\item Asymptote bei -2 %y(x)=-2
\end{itemize}

\end{enumerate}

\item %2. (10)
Der Absatz und die Kosten% für ein Produkt eines Unternehmens

\begin{itemize}
\item Preis-Absatzfunktion $p(x) = 222-6x, x \in [ 0, 37 ]$
\item Kostenfunktion $K(x)=293 + 159x-12x^2+x^3, x>=0$
\item $p$ ist der Preis, $x$ die produzierte und abgesetzte Menge
\end{itemize}

\begin{enumerate}
\item 
Ermittle für welche Menge $x$ maximaler Gewinn erzielt wird. Weise das Maximum nach. %Wert ermitteln, Nachweis: 2.\,Ableitung < 0, und 3.\,Ableitung \neq 0

\item 
Bestimme das Mononieverhalten der Kostenfunktion

\item 
Untersuche das Krümmungsverhalten der Erlösfunktion $E(x)=xp(x)$

\item 
Ermittle die Elastizität des Preises bezüglich des Absatzes. Für welche Mengen reagiert der Preis proportional-elastisch (d.h. $|\epsilon_p(x)|=1$ )? 

\end{enumerate}

\item 3. (7)
Ein Unternehmen vertreibt zwei Güter deren Nachfrage von den Preisen wie folgt anhängt:

\[
x_1(p_1, p_2) = 10 - p_1 + 2p_2
,
x_2(p_1, p_2) = 8 + 2p_1 - 6p_2
\]

Dabei sind $p_i$ die Preise und $x_i$ die nachgefragten Mengen nach Gut $i, i={1,2}$. 

\item %4. (5)
In einer Fabrik werden drei Sorten A, B, C von Seiten hergestellt und für 150 Euro (A), 100 Euro (B) und 50 Euro (C) je Stück verkauft.
Für Seile vom Typ A werden 16\,kg Plastik und 4kg\,Kupfer benötigt, bei B 6\,kg Plastik und 12\,kg Kupfer und C 4\,kg Plastik und kein Kupfer.
Aus produktionstechnischen Gründen darf die produzierte Menge von B nicht größer sein als die doppelte Menge von A.
Außerdem beträgt der Materialvorrat im Lager nur 252\,kg Plastik und 168\,kg Kupfer.
Stellen Sie ein mathematisches Optimierungsproblem für die Maximierung des Erlöses der Fabrik auf \emph{ohne} es zu lösen.


\item %5. (2)
Ermittle die Fläche, die der Graph der Funktion

\[
f: [1,2] \rightarrow \R: f_k(x)=\frac{1}{k} (x-1)(x-2)^2
\]

für festes k>0 mit der x-Achse im 1. Quadranten
einschließt.


\end{enumerate}

\end{document}
