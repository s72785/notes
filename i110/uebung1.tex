Übungsaufgaben (Zahlendarstellung - Teil 1)

1. Die Zahl (1997)_10 ist manuell (ohne Rechner) in eine Hexadezimalzahl, eine Dualzahl und eine Oktalzahl zu konvertieren. Das Ergebnis ist durch Rückkonvertierung in das Dezimalsystem auf Korrektheit zu überprüfen.

Umwandlung in Hexadezimal:

    1997 : 16 = 124 R 13 -> D
     124 : 16 =   7 R 12 -> C
       7 : 16 =   0 R  7 -> 7

Probe:

    7*16^2 + 12*16^1 + 13*16^0 = 1997 q.e.d.

2. Die Zahl (1997)_10 ist manuell in eine Zahl mit Basis 20 und eine Zahl mit Basis 32 zu konvertieren. Hierzu ist die Tabelle der binären Repräsentation der Hexadezimalziffern (0)_16 ..(F)_16 bis (10011)_2 = (J)_20 bzw. bis (11111)_2 = (V)_32 zu erweitern. Aus der Darstellung mit Basis 32 ist die Dualdarstellung und daraus die Hexadezimal- und die Oktaldarstellung der Zahl (1997)_10 zu ermitteln.

3. Die Zahl (1234)_10 ist im Siebener- und im Vierersystemarzustellen.

4. Die Zahl (430)_10 ist in eine Dualzahl umzuwandeln. Aus der Dualzahl ist die äquivalenten Hexadezimal- und Oktalzahlzubestimmen. Die ermittelte Dualzahl der Form z=a n b n +a n-1 b n - 1 + ... +a 2 b 2 +a 1 b 1 + a 0 b 0 ist über das Horner-Schema z = ((...(a n b+a n-1 )b+ ... a 2 ) b +a 1 ) b +a 0 wieder in die Dezimaldarstellung umzuwandeln.

5. Die Zahlen sind mit dem Horner-S ch e m a i n D e z i m a l z a hl e n u m z u w a nde l n : (375) 8 , (1210) 8 , ( 8 8 8 ) 9 , ( A D A ) 16

6. Manuell ist (ABC) 16 + (ABC) 16 zu berechnen. Die Korrektheit des E r g e b nisses ist durch Umwandlung in das Dezimalsyste m z u ve r if iz ie r e n .

7. Die folgenden Zahlen ist in das Dualsystem umzuwandeln: (0,375) 10

8. Die folgende Zahl ist in das Fünfersystem umzuwandeln: (0,24) 10

9. Die folgende Zahl ist in das Oktalsystem umzuwandeln: (0,2) 1 0

10. Die Zahl (11,625) 10 ist als Dual-, Hexadezimal- und Oktalzahl darzu- st e l l e n .

11. Die Aufgabe 11+14=25 soll in binärer Arithmetik gelöst werden.

12. Die folgenden Aufgaben sind dual zu berechnen, das Ergebnis ist danach dezimal zu verifizieren und hexadezimal und oktaldarzustellen:

(11011)_2 + (10101)_2
(11101)_2 - (10010)_2
(11011)_2 * (10101)_2
(1001000)_2 : (1000)_2

13 . Die folgenden Aufgaben sind hexadezimal zu berechnen, das Ergebnis ist danach dezimal zu verifizieren:

(ABC)_16 + (987)_16
(DDE9)_16 - (3F7B)_16
(CDE)_16 * (A1)_16

14. Die folgenden Aufgaben sind oktal zu berechnen, das Ergebnis ist danach dezimal zu verifizieren:

(1775) 8 + (766) 8
(3321) 8 - (763) 8
(721) 8 * (72) 8

15. Die Aufgabe 151.875 + 27.625 = 179.5 soll in binärer Arithmetik gelöst werden.

16. Die Aufgabe 10*13=130 ist in binärer Arithmetik zu lösen

17. Die Aufgabe 17,375*9,75 = 169,40625 ist in binärer Arithmetik zu lösen.

18. Wenn man eine Dualzahl um n (n>0) Positionen nach links verschiebt und rechts mit 0 auffüllt, ist das äquivalent der M u l t i p l i k a t i o n m i t 2 n . Es ist 15*8 sowohl mittels dualer Mul t i p l i k a t i o n a l s a u ch m i t t el s Verschiebung nach links auszurechnen.

19. Wenn man eine Dualzahl um n (n>0) Positione n na c h r e c hts ve r sc hie bt u n d l inks m it 0 auffüllt, ist das äquivalent der Division durch 2 n . E s i s t 1 2 0 / 4 s o w o h l m i t t e l s d ualer Division als auch mittels Verschiebung nach rechts auszurechnen.

20. Eine echt gebrochene Zahl n (n<1)

n =
\Sigma
bi ⋅ B
i
läßt sich mit Hilfe
i = –M
des Hornerschem a s w i e f o l g t d ar s t e l l e n :
n = 1/B*(b -1 + 1 / B * ( b -2 + 1/B*(b -3 + ... + 1/B*(b -M +1 + 1 / B * b -M ) ...)))

Die folgenden Zahlen sind mit dem Hornerschema in Dezimalzahlen zu konvertieren:

    (0,375)_8 = 
    (0,1210)_10 = 
    (0,888)_9 = 
    (0,ADDA)_16 = 

21. Mit dem Zweierkompliment soll die negative Dualzahlz =-1 in einem Bereich von 4 Byte dargestellt wer den. Das Ergebnis ist mit 23^2-|z| zu vergleichen. Das Ergebnis ist zusätzlich in Form von 8 Hexadezimalziffern anzugeben. Wie wird z=|-2^31| dargestellt, auch hexadezimal?
Wie lautet dual, hexadezimal und dezimal die größte positive Zahl mit Vorzeichen für 4 Byte? Wie lautet der Wertebereich für Zahlen mit Vorzeichen in einem 4 Byte Bereich ?

22. Man stelle (-1)_10 , (-64)_10 und (-128)_10 für ein Byte dual dar.

      (-1)_10 => (11111111)_2
     (-64)_10 => (01000000)_2 xor 255_10 + 1_10 => (10111111)_2 + 1_10 => (11000000)_2
    (-128)_10 => (10000000)_2 xor 255_10 + 1_10 => (01111111)_2 + 1_10 => (10000000)_2
