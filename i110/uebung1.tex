\documentclass[12pt,a4paper]{scrreprt}
\usepackage{amsmath,amssymb,mathrsfs,dsfont}
\usepackage[utf8]{inputenc}
\usepackage[ngerman]{babel}
\usepackage[pdfborderstyle={/S/U/W 1}]{hyperref}
%\usepackage{siunitx}
\usepackage{color}
\definecolor{dkgreen}{rgb}{0,0.6,0}
\definecolor{gray}{rgb}{0.5,0.5,0.5}
\definecolor{mauve}{rgb}{0.58,0,0.82}

\usepackage{listings}
\lstset{frame=tb,
  language=C,
  aboveskip=3mm,
  belowskip=3mm,
  showstringspaces=false,
  columns=flexible,
  basicstyle={\small\ttfamily},
  numbers=none,
  numberstyle=\tiny\color{gray},
  keywordstyle=\color{blue},
  commentstyle=\color{dkgreen},
  stringstyle=\color{mauve},
  breaklines=true,
  breakatwhitespace=true,
  tabsize=3
}

\oddsidemargin0mm \evensidemargin3mm \textwidth150mm \textheight23cm
\parindent0mm \pagestyle{empty} \topmargin-1cm
\pagestyle{headings}

%euro-zeichen f. utf8
\usepackage{eurosym}
\DeclareUnicodeCharacter{20AC}{\euro}
\newcommand{\Lsg}{\par \textbf{Lsg.: \hfill }}
\newcommand{\lsg}[1]{\underline{\underline{#1}}}

\begin{document}

\begin{flushleft}%https://www.htw-dresden.de/fakultaet-informatikmathematik/personal/professuren/prof-dr-wilfried-nestler.html
\href{mailto:nestler@informatik.htw-dresden.de}{Prof. Wilfried Nestler}
\end{flushleft}

\large\textbf{Übungsaufgaben (Zahlendarstellung - Teil 1)}

\begin{enumerate}

%1.
\item Die Zahl $(1997)_{10}$ ist manuell (d.h. ohne Rechner%na das muss ich hier mal relativ ignorieren, aber ich rechne ausnahmsweise mal selbst ;)
) in
\begin{enumerate}
\item eine Hexadezimalzahl,
\item eine Dualzahl und
\item eine Oktalzahl
\end{enumerate}
zu konvertieren. Das Ergebnis ist durch Rückkonvertierung in das Dezimalsystem auf Korrektheit zu überprüfen.

%-------------
\Lsg

\begin{enumerate}

\item Umwandlung in Hexadezimal: 

\begin{tabular}{rrrrrrr}
    $1997$ : & $16 =$ & $124$ R & $13 \rightarrow $ & & & D \\
    $ 124$ : & $16 =$ & $  7$ R & $12 \rightarrow $ & & C \\
    $   7$ : & $16 =$ & $  0$ R & $ 7 \rightarrow $ & 7 \\
\hline
 \multicolumn{4}{}{} & 7 & C & D
\end{tabular}
\par Probe: \\
\begin{math}
7*16^2
+12*16^1
+13*16^0
= \lsg{
	1997_{10} = \texttt{DC7}_{16}
}
\end{math}

\item \label{aufgabe1_dual} Umwandlung in Dual:

\begin{tabular}{rrrrrrrrrrrrrr}
    $1997$ : & $2 =$ & $998$ R & & & & & & & & & & & $1$ \\
    $ 998$ : & $2 =$ & $499$ R & & & & & & & & & & $0$ \\
    $ 499$ : & $2 =$ & $249$ R & & & & & & & & & $1$ \\
    $ 249$ : & $2 =$ & $124$ R & & & & & & & & $1$ \\
    $ 124$ : & $2 =$ & $ 62$ R & & & & & & & $0$ \\
    $  62$ : & $2 =$ & $ 31$ R & & & & & & $0$ \\
    $  31$ : & $2 =$ & $ 15$ R & & & & & $1$ \\
    $  15$ : & $2 =$ & $  7$ R & & & & $1$ \\
    $   7$ : & $2 =$ & $  3$ R & & & $1$ \\
    $   3$ : & $2 =$ & $  1$ R & & $1$ \\
    $   1$ : & $2 =$ & $  0$ R & $1$ \\
\hline
 \multicolumn{3}{}{} & 1 & 1 & 1 & 1 & 1 & 0 & 0 & 1 & 1 & 0 & 1
\end{tabular} %ich werde auch noch rausfinden wie man das scriptet -.-'
\par Probe: \\
\begin{math}
1*2^0
+0*2^1
+1*2^2
+1*2^3
+0*2^4
+0*2^5
+1*2^6
+1*2^7
+1*2^8
+1*2^9
+1*2^10
 = \lsg{
	1997_{10} = {11111001101}_{2}
}
\end{math}

\item Umwandlung in Oktal: \\
folgere ich mal %aus lauter Faulheit
aus \ref{aufgabe1_dual} \ldots 

\begin{tabular}{cccc}
 $\underbrace{011}$ & $\underbrace{111}$ & $\underbrace{001}$ & $\underbrace{101}$ \\
                  3 &                  7 &                  1 &                  5 \\
\end{tabular}
\par Probe: \\
\begin{math}
 3*8^3 +7*8^2 +1*8^1 +5*8^0 = \lsg{1997_{10} = {3715}_{8} }
\end{math}

\end{enumerate}
%-------------

%2.
\item \begin{enumerate} \item Die Zahl ${1997}_{10}$ ist manuell
\begin{enumerate}
\item in eine Zahl mit Basis $20$ und
\item eine Zahl mit Basis $32$
\end{enumerate}
zu konvertieren.
Hierzu ist die Tabelle der binären Repräsentation der Hexadezimalziffern {$0_{16} \ldots \texttt{F}_{16}$} bis {$10011_{2} = (\texttt{J})_{20}$} bzw. bis {$(11111)_{2} = (\texttt{V})_{32}$} zu erweitern.
\item Aus der Darstellung mit Basis $32$ ist
\begin{enumerate}
\item die Dualdarstellung und daraus 
\item die Hexadezimal- und
\item die Oktaldarstellung
\end{enumerate}
der Zahl $1997_{10}$ zu ermitteln.
\end{enumerate}
%-------------
\Lsg%-------------
\begin{enumerate}
\item \begin{enumerate}
\item Mit Basis 20

\begin{tabular}{rrrrrrr}
    $1997$ : & $20 =$ & $ 99$ R & $17 \rightarrow $ & & & H \\
    $  99$ : & $20 =$ & $  4$ R & $19 \rightarrow $ & & J \\
    $   4$ : & $20 =$ & $  0$ R & $ 4 \rightarrow $ & 4 \\
\hline
 \multicolumn{4}{}{} & 4 & J & H
\end{tabular}
\par Probe: \\
\begin{math}
4*20^2
+19*20^1
+17*20^0
= \lsg{
	1997_{10} = \texttt{4JH}_{20}
}
\end{math}

\item Mit Basis 32

\begin{tabular}{rrrrrrr}
%00 01 02 03 04 05 06 07 08 09 10 11 12 13 14 15 16 17 18 19 20 21 22 23 24 25 26 27 28 29 30 31
% 0  1  2  3  4  5  6  7  8  9  A  B  C  D  E  F  G  H  I  J  K  L  M  N  O  P  Q  R  S  T  U  V
    $1997$ : & $32 =$ & $ 62$ R & $13 \rightarrow $ & & & D \\
    $  62$ : & $32 =$ & $  1$ R & $30 \rightarrow $ & & U \\
    $   1$ : & $32 =$ & $  0$ R & $ 1 \rightarrow $ & 1 \\
\hline
 \multicolumn{4}{}{} & 1 & U & D
\end{tabular}
\par Probe: \\
\begin{math}
1*32^2
+30*32^1
+13*32^0
= \lsg{
	1997_{10} = \texttt{1UD}_{32}
}
\end{math}

\end{enumerate}
\begin{enumerate}
\item Aus Basis 32 in Dual 

\begin{tabular}{ccc}
%00 01 02 03 04 05 06 07 08 09 10 11 12 13 14 15 16 17 18 19 20 21 22 23 24 25 26 27 28 29 30 31
% 0  1  2  3  4  5  6  7  8  9  A  B  C  D  E  F  G  H  I  J  K  L  M  N  O  P  Q  R  S  T  U  V
 \texttt{1} & \texttt{U} & \texttt{D} \\
 $\overbrace{00001}$ & $\overbrace{11110}$ & $\overbrace{01101}$ %000011111001101
\end{tabular}
\par Probe: \\
\begin{math}
 1*2^10
+1*2^9
+1*2^8
+1*2^7
+1*2^6
+0*2^5
+0*2^4
+1*2^3
+1*2^2
+0*2^1
+1*2^0
= \lsg{
	{1997}_{10} = \texttt{1UD}_{32} = {000011111001101}_{2}
}
\end{math}

\item Aus Basis 32 in Hexadezimal 

\begin{tabular}{ccc}
%00 01 02 03 04 05 06 07 08 09 10 11 12 13 14 15 16 17 18 19 20 21 22 23 24 25 26 27 28 29 30 31
% 0  1  2  3  4  5  6  7  8  9  A  B  C  D  E  F  G  H  I  J  K  L  M  N  O  P  Q  R  S  T  U  V
 \texttt{1} & \texttt{U} & \texttt{D} \\
 $\overbrace{00001}$ & $\overbrace{11110}$ & $\overbrace{01101}$ \\ %000011111001101
 $\underbrace{0111}$ & $\underbrace{1100}$ & $\underbrace{1101}$ \\ %000011111001101
 \texttt{7} & \texttt{C} & \texttt{D} 
\end{tabular}
\par Probe: \\
\begin{math}
  7*16^2
+12*16^1
+13*16^0
= \lsg{
	{1997}_{10} = \texttt{1UD}_{32} = \texttt{7CD}_{16}
}
\end{math}

\item Aus Basis 32 in Oktal 

\begin{tabular}{cccc}
%00 01 02 03 04 05 06 07 08 09 10 11 12 13 14 15 16 17 18 19 20 21 22 23 24 25 26 27 28 29 30 31
% 0  1  2  3  4  5  6  7  8  9  A  B  C  D  E  F  G  H  I  J  K  L  M  N  O  P  Q  R  S  T  U  V
 & \texttt{1} & \texttt{U} & \texttt{D} \\
 & $\overbrace{00001}$ & $\overbrace{11110}$ & $\overbrace{01101}$ \\ %000011111001101
 $\underbrace{011}$ & $\underbrace{111}$ & $\underbrace{001}$ & $\underbrace{101}$ \\ %000011111001101
 3 & 7 & 1 & 5
\end{tabular}
\par Probe: \\
\begin{math}
  3*8^3
+ 7*8^2
+ 1*8^1
+ 5*8^0
= \lsg{
	{1997}_{10} = \texttt{1UD}_{32} = \texttt{3715}_{8}
}
\end{math}

\end{enumerate}
\end{enumerate}

%3.
\item Die Zahl $(1234)_{10}$ ist \begin{enumerate}
\item im Siebener- und 
\item im Vierersystem 
\end{enumerate}
darzustellen.

%-------------
\Lsg
\begin{enumerate}
\item Mit Basis 7

\begin{tabular}{rrrrrrrr}
	$1234$ : & $7 =$ & $176$ R & $2 \rightarrow $ & & & & 2 \\
	$ 176$ : & $7 =$ & $ 25$ R & $1 \rightarrow $ & & & 1 \\
	$  25$ : & $7 =$ & $  3$ R & $4 \rightarrow $ & & 4 \\
	$   3$ : & $7 =$ & $  0$ R & $3 \rightarrow $ & 3 \\
\hline
 \multicolumn{4}{}{} & 3 & 4 & 1 & 2
\end{tabular}
\par Probe: \\
\begin{math}
 3*7^3
+4*7^2
+1*7^1
+2*7^0
= \lsg{
	1234_{10} = \texttt{3412}_{7}
}
\end{math}

\item Mit Basis 4

\begin{tabular}{rrrrrrrrrr}
	$1234$ : & $4 =$ & $308$ R & $2 \rightarrow $ & & & & & & 2 \\
	$ 208$ : & $4 =$ & $ 77$ R & $0 \rightarrow $ & & & & & 0 \\
	$  77$ : & $4 =$ & $ 19$ R & $1 \rightarrow $ & & & & 1 \\
	$  19$ : & $4 =$ & $  4$ R & $3 \rightarrow $ & & & 3 \\
	$   4$ : & $4 =$ & $  1$ R & $0 \rightarrow $ & & 0 \\
	$   1$ : & $4 =$ & $  0$ R & $1 \rightarrow $ & 1 \\
\hline
 \multicolumn{4}{}{} & 1 & 0 & 3 & 1 & 0 & 2
\end{tabular}
\par Probe: \\
\begin{math}
 1*4^5
+0*4^4
+3*4^3
+1*4^2
+0*4^1
+2*4^0
= \lsg{
	1234_{10} = \texttt{103102}_{4}
}
\end{math}

\end{enumerate}
%-------------

%4.
\item Die Zahl $(430)_{10}$ ist in eine Dualzahl umzuwandeln. Aus der Dualzahl ist die äquivalenten Hexadezimal- und Oktalzahl zu bestimmen.

Die ermittelte Dualzahl der Form $z=a_n b^n +a_{n-1} b^{n - 1} + \ldots + a_2 b^2 + a_1 b^1 + a_0 b^0 $ ist über das Horner-Schema $ z = ((\ldots(a_n b + a_{n-1} )b + \ldots a_2 ) b + a_1 ) b + a_0$ wieder in die Dezimaldarstellung umzuwandeln.

%-------------
\Lsg

\begin{tabular}{rrrrrrrrrrr}
430 : 2 = & 215 R & & & & & & & & & 0 \\
215 : 2 = & 107 R & & & & & & & & 1 \\
107 : 2 = &  53 R & & & & & & & 1 \\
 53 : 2 = &  26 R & & & & & & 1 \\
 26 : 2 = &  13 R & & & & & 0 \\
 13 : 2 = &   6 R & & & & 1 \\
  6 : 2 = &   3 R & & & 0 \\
  3 : 2 = &   1 R & & 1 \\
  1 : 2 = &   0 R & 1 \\
\hline
 & & 1 & 1 & 0 & 1 & 0 & 1 & 1 & 1 & 0 \\
 & & $\underbrace{0001}$ & $\underbrace{1010}$ & $\underbrace{1110}$ \\
 & & 1 & A & E
\end{tabular}

Probe:

\begin{math}
 1*2^8
+1*2^7
+0*2^6
+1*2^5
+0*2^4
+1*2^3
+1*2^2
+1*2^1
+0*2^0
= \lsg{
	430_{10} = {110101110}_{2}
}
\end{math}

%-------------

%5.
\item Die Zahlen sind mit dem Horner-Scema in Dezimalzahlen umzuwandeln: \begin{enumerate}
\item $375_8$,
\item $1210_8$,
\item $888_9$,
\item $\texttt{ADA}_{16}$
\end{enumerate}

%-------------
\Lsg

\begin{enumerate}
\item $375_8$
\item $1210_8$
\item $888_9$
\item $\texttt{ADA}_{16}$
\end{enumerate}%-------------

%6.
\item Manuell ist $\texttt{ABC}_{16} + \texttt{ABC}_{16}$ zu berechnen. Die Korrektheit des Ergebnisses ist durch Umwandlung in das Dezimalsystem zu verifizieren.

%01 02 03 04 05 06 07 08 09 10 11 12 13 14 15
% 1  2  3  4  5  6  7  8  9  A  B  C  D  E  F
%-------------
\Lsg

\begin{tabular}{lllll}
  & & $A$ & $B$ & $C$ \\
+ & & $A_{1}$ & $B_{1}$ & $C$ \\
\hline 
= & 1 & 5 & 7 & 8
\end{tabular}

$\texttt{ABC}_{16} + \texttt{ABC}_{16} = \texttt{1578}_{16}$
%-------------

%7.
\item Die folgenden Zahlen ist in das Dualsystem umzuwandeln: ${0,375}_{10}$

%-------------
\Lsg

\begin{math}
{0,375}_{10} = 
\frac{1}{4} + \frac{1}{8} 
 = 0*2^{-1} + 1*2^{-2} + 1*2^{-3} = \lsg{
	\texttt{011}_{2}
}
\end{math}
%-------------

%8.
\item Die folgende Zahl ist in das Fünfersystem umzuwandeln: ${0,24}_{10}$

%-------------
\Lsg%-------------

%9.
\item Die folgende Zahl ist in das Oktalsystem umzuwandeln: ${0,2}_{10}$

%-------------
\Lsg%-------------

%10.
\item Die Zahl ${11,625}_{10}$ ist als Dual-, Hexadezimal- und Oktalzahl darzustellen.

%-------------
\Lsg%-------------

%11.
\item Die Aufgabe $11+14=25$ soll in binärer Arithmetik gelöst werden.

%-------------
\Lsg%-------------

%12.
\item Die folgenden Aufgaben sind dual zu berechnen, das Ergebnis ist danach dezimal zu verifizieren und hexadezimal und oktaldarzustellen:

\begin{enumerate}
\item $(11011)_2 + (10101)_2$

\begin{tabular}{rr}
 & 11011 \\
+& 10101 \\
\hline
 & 110000
\end{tabular}

\item $(11101)_2 - (10010)_2$

\begin{tabular}{rr}
 & 11101 \\
-& 10010 \\
\hline
 & 1011
\end{tabular}

\item $(11011)_2 * (10101)_2$

\begin{tabular}{llcr}
 & 11101 & * & 10010 \\
 & 11101  \\
 & 0 \\
 & 0011101  \\
 & 0 \\
+& 000011101  \\
\hline 
1000110111
\end{tabular}

\item $(1001000)_2 : (1000)_2$

\begin{tabular}{llcr}
 & 11101 & : & 10010 \\
\hline 
1000110111
\end{tabular}


\end{enumerate}

%-------------
\Lsg%-------------

%13.
\item Die folgenden Aufgaben sind hexadezimal zu berechnen, das Ergebnis ist danach dezimal zu verifizieren: \\
\begin{math}
\texttt{ABC}_{16} + \texttt{987}_{16}   \\
\texttt{DDE9}_{16} - \texttt{3F7B}_{16} \\
\texttt{CDE}_{16} * \texttt{A1}_{16}
\end{math}

%-------------
\Lsg%-------------

%14.
\item Die folgenden Aufgaben sind oktal zu berechnen, das Ergebnis ist danach dezimal zu verifizieren: \\
\begin{math}
{1775}_8 + {766}_8 \\
{3321}_8 - {763}_8 \\
{721}_8 * {72}_8
\end{math}

%-------------
\Lsg%-------------

%15.
\item Die Aufgabe ${151.875}_{10} + {27.625}_{10} = {179.5}_{10}$ soll in binärer Arithmetik gelöst werden.

%-------------
\Lsg%-------------

%16.
\item Die Aufgabe ${10}_{10}*{13}_{10}={130}_{10}$ ist in binärer Arithmetik zu lösen

%-------------
\Lsg%-------------

%17.
\item Die Aufgabe ${17,375}_{10}*{9,75}_{10} = {169,40625}_{10}$ ist in binärer Arithmetik zu lösen.

%-------------
\Lsg%-------------

%18.
\item Wenn man eine Dualzahl um $n, n>0$ Positionen nach links verschiebt und rechts mit $0$ auffüllt, ist das äquivalent der Multiplikation mit $2 n$ . Es ist ${15}*{8}$ sowohl mittels dualer Multiplikation als auch mittels Verschiebung nach links auszurechnen.

%-------------
\Lsg%-------------

%19.
\item Wenn man eine Dualzahl um $n, n>0$ Positione nnach rechts verschiebt und links m it $0$ auffüllt, ist das äquivalent der Division durch $2n$ . Es ist $\frac{120}{4}$ sowohl mittels dualer Division als auch mittels Verschiebung nach rechts auszurechnen.

%-------------
\Lsg%-------------

%20.
\item Eine echt gebrochene Zahl $n, n<1$ \\
\begin{math}
n =
\Sigma\,bi \cdot B
i
\end{math}
läßt sich mit Hilfe
$i = -M$
\begin{math}
n = 1/B*(b -1 + 1 / B * ( b -2 + 1/B*(b -3 + ... + 1/B*(b -M +1 + 1 / B * b -M ) \ldots)))
\end{math}

Die folgenden Zahlen sind mit dem Hornerschema in Dezimalzahlen zu konvertieren:

\begin{enumerate}
	\item ${0,375}_{8}$
	\item ${0,1210}_{10}$
	\item ${0,888}_{9}$
	\item $\text{0,ADDA}_{16}$
\end{enumerate}

%-------------
\Lsg%-------------

%21.
\item Mit dem Zweierkompliment soll die negative Dualzahl $z =-1$ in einem Bereich von $4$\,Byte dargestellt wer den. Das Ergebnis ist mit $23^2-|z|$ zu vergleichen. Das Ergebnis ist zusätzlich in Form von $8$ Hexadezimalziffern anzugeben. Wie wird $z=|-2^31|$ dargestellt, auch hexadezimal?
Wie lautet dual, hexadezimal und dezimal die größte positive Zahl mit Vorzeichen für $4$\,Byte? Wie lautet der Wertebereich für Zahlen mit Vorzeichen in einem $4$\,Byte Bereich ?

%-------------
\Lsg%-------------

%22.
\item Man stelle ${-1}_{10}, {-64}_{10}$ und ${-128}_{10}$ für $1$\,Byte dual dar. \\

%-------------
\Lsg

\begin{math}
      (-1)_{10} => (11111111)_2 \\
     (-64)_{10} => (01000000)_2 xor 255_{10} + 1_{10} \\ => (10111111)_2 + 1_{10} => (11000000)_2 \\
    (-128)_{10} => (10000000)_2 xor 255_{10} + 1_{10} \\ => (01111111)_2 + 1_{10} => (10000000)_2 
\end{math}
%-------------

\end{enumerate}
\end{document}
