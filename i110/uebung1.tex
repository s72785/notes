\documentclass[12pt,a4paper]{scrreprt}
\usepackage{amsmath,amssymb,mathrsfs,dsfont}
\usepackage[utf8]{inputenc}
\usepackage[ngerman]{babel}
\usepackage{hyperref}

\usepackage{color}
\definecolor{dkgreen}{rgb}{0,0.6,0}
\definecolor{gray}{rgb}{0.5,0.5,0.5}
\definecolor{mauve}{rgb}{0.58,0,0.82}

\usepackage{listings}
\lstset{frame=tb,
  language=C,
  aboveskip=3mm,
  belowskip=3mm,
  showstringspaces=false,
  columns=flexible,
  basicstyle={\small\ttfamily},
  numbers=none,
  numberstyle=\tiny\color{gray},
  keywordstyle=\color{blue},
  commentstyle=\color{dkgreen},
  stringstyle=\color{mauve},
  breaklines=true,
  breakatwhitespace=true,
  tabsize=3
}

\oddsidemargin0mm \evensidemargin3mm \textwidth150mm \textheight23cm
\parindent0mm \pagestyle{empty} \topmargin-1cm
\pagestyle{headings}

%euro-zeichen f. utf8
\usepackage{eurosym}
\DeclareUnicodeCharacter{20AC}{\euro}

\begin{document}

\begin{flushleft}%https://www.htw-dresden.de/fakultaet-informatikmathematik/personal/professuren/prof-dr-wilfried-nestler.html
\href{mailto:nestler@informatik.htw-dresden.de}{Prof. Wilfried Nestler}
\end{flushleft}

\large\textbf{Übungsaufgaben (Zahlendarstellung - Teil 1)}

\begin{enumerate}

%1.
\item Die Zahl $(1997)_10$ ist manuell (ohne Rechner) in eine Hexadezimalzahl, eine Dualzahl und eine Oktalzahl zu konvertieren. Das Ergebnis ist durch Rückkonvertierung in das Dezimalsystem auf Korrektheit zu überprüfen.

Umwandlung in Hexadezimal:
\[
    1997 : 16 = 124 R 13 -> D
     124 : 16 =   7 R 12 -> C
       7 : 16 =   0 R  7 -> 7
\]
Probe:
\[
    7*16^2 + 12*16^1 + 13*16^0 = 1997 q.e.d.
\]
%2.
\item Die Zahl $(1997)_10$ ist manuell in eine Zahl mit Basis $20$ und eine Zahl mit Basis $32$ zu konvertieren. Hierzu ist die Tabelle der binären Repräsentation der Hexadezimalziffern $(0)_16 \ldots (F)_16$ bis $(10011)_2 = (J)_20$ bzw. bis $(11111)_2 = (V)_32$ zu erweitern. Aus der Darstellung mit Basis $32$ ist die Dualdarstellung und daraus die Hexadezimal- und die Oktaldarstellung der Zahl $(1997)_10$ zu ermitteln.

%3.
\item Die Zahl $(1234)_10$ ist im Siebener- und im Vierersystemarzustellen.

%4.
\item Die Zahl $(430)_10$ ist in eine Dualzahl umzuwandeln. Aus der Dualzahl ist die äquivalenten Hexadezimal- und Oktalzahlzubestimmen. Die ermittelte Dualzahl der Form $z=a n b n +a n-1 b n - 1 + ... +a 2 b 2 +a 1 b 1 + a 0 b 0$ ist über das Horner-Schema $z = ((...(a n b+a n-1 )b+ ... a 2 ) b +a 1 ) b +a 0$ wieder in die Dezimaldarstellung umzuwandeln.

%5.
\item Die Zahlen sind mit dem Horner-Scema in Dezimalzahlen umzuwandeln: $(375)_8 , (1210)_8 , (888)_9 , (ADA)_16$

%6.
\item Manuell ist $(ABC) 16 + (ABC) 16$ zu berechnen. Die Korrektheit des Ergebnisses ist durch Umwandlung in das Dezimalsystem zu verifizieren.

%7.
\item Die folgenden Zahlen ist in das Dualsystem umzuwandeln: $(0,375)_10$

%8.
\item Die folgende Zahl ist in das Fünfersystem umzuwandeln: $(0,24)_10$

%9.
\item Die folgende Zahl ist in das Oktalsystem umzuwandeln: $(0,2)_10$

%10.
\item Die Zahl $(11,625)_10$ ist als Dual-, Hexadezimal- und Oktalzahl darzustellen.

%11.
\item Die Aufgabe $11+14=25$ soll in binärer Arithmetik gelöst werden.

%12.
\item Die folgenden Aufgaben sind dual zu berechnen, das Ergebnis ist danach dezimal zu verifizieren und hexadezimal und oktaldarzustellen:
\[
(11011)_2 + (10101)_2
(11101)_2 - (10010)_2
(11011)_2 * (10101)_2
(1001000)_2 : (1000)_2
\]
%13.
\item Die folgenden Aufgaben sind hexadezimal zu berechnen, das Ergebnis ist danach dezimal zu verifizieren:
\[
(ABC)_16 + (987)_16
(DDE9)_16 - (3F7B)_16
(CDE)_16 * (A1)_16
\]
%14.
\item Die folgenden Aufgaben sind oktal zu berechnen, das Ergebnis ist danach dezimal zu verifizieren:
\[
(1775) 8 + (766) 8
(3321) 8 - (763) 8
(721) 8 * (72) 8
\]
%15.
\item Die Aufgabe $151.875 + 27.625 = 179.5$ soll in binärer Arithmetik gelöst werden.

%16.
\item Die Aufgabe $10*13=130$ ist in binärer Arithmetik zu lösen

%17.
\item Die Aufgabe $17,375*9,75 = 169,40625$ ist in binärer Arithmetik zu lösen.

%18.
\item Wenn man eine Dualzahl um $n (n>0)$ Positionen nach links verschiebt und rechts mit $0$ auffüllt, ist das äquivalent der Multiplikation mit $2 n$ . Es ist $15*8$ sowohl mittels dualer Multiplikation als auch mittels Verschiebung nach links auszurechnen.

%19.
\item Wenn man eine Dualzahl um $n (n>0)$ Positione nnach rechts verschiebt und links m it $0$ auffüllt, ist das äquivalent der Division durch $2n$ . Es ist $1 2 0 / 4$ sowohl mittels dualer Division als auch mittels Verschiebung nach rechts auszurechnen.

%20.
\item Eine echt gebrochene Zahl $n (n<1)$
\[
n =
\Sigma
bi \cdot B
i
\text{läßt sich mit Hilfe}
i = -M
\text{des Hornerschemas wie folgt darstellen:} \\
n = 1/B*(b -1 + 1 / B * ( b -2 + 1/B*(b -3 + ... + 1/B*(b -M +1 + 1 / B * b -M ) \ldots)))
\]

Die folgenden Zahlen sind mit dem Hornerschema in Dezimalzahlen zu konvertieren:

\begin{enumerate}
	\item $(0,375)_8$
	\item $(0,1210)_10$
	\item $(0,888)_9$
	\item $(0,ADDA)_16$
\end{enumerate}

%21.
\item Mit dem Zweierkompliment soll die negative Dualzahl $z =-1$ in einem Bereich von $4$ Byte dargestellt wer den. Das Ergebnis ist mit $23^2-|z|$ zu vergleichen. Das Ergebnis ist zusätzlich in Form von $8$ Hexadezimalziffern anzugeben. Wie wird $z=|-2^31|$ dargestellt, auch hexadezimal?
Wie lautet dual, hexadezimal und dezimal die größte positive Zahl mit Vorzeichen für $4$ Byte? Wie lautet der Wertebereich für Zahlen mit Vorzeichen in einem $4$ Byte Bereich ?

%22.
\item Man stelle $(-1)_10, (-64)_10 und (-128)_10$ für $1$ Byte dual dar.
\[
      (-1)_10 => (11111111)_2
     (-64)_10 => (01000000)_2 xor 255_10 + 1_10 => (10111111)_2 + 1_10 => (11000000)_2
    (-128)_10 => (10000000)_2 xor 255_10 + 1_10 => (01111111)_2 + 1_10 => (10000000)_2
\]
\end{enumerate}
\end{document}
