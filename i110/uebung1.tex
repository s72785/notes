\documentclass[12pt,a4paper]{scrreprt}
\usepackage{amsmath,amssymb,mathrsfs,dsfont}
\usepackage[utf8]{inputenc}
\usepackage[ngerman]{babel}
\usepackage[pdfborderstyle={/S/U/W 1}]{hyperref}

\usepackage{color}
\definecolor{dkgreen}{rgb}{0,0.6,0}
\definecolor{gray}{rgb}{0.5,0.5,0.5}
\definecolor{mauve}{rgb}{0.58,0,0.82}

\usepackage{listings}
\lstset{frame=tb,
  language=C,
  aboveskip=3mm,
  belowskip=3mm,
  showstringspaces=false,
  columns=flexible,
  basicstyle={\small\ttfamily},
  numbers=none,
  numberstyle=\tiny\color{gray},
  keywordstyle=\color{blue},
  commentstyle=\color{dkgreen},
  stringstyle=\color{mauve},
  breaklines=true,
  breakatwhitespace=true,
  tabsize=3
}

\oddsidemargin0mm \evensidemargin3mm \textwidth150mm \textheight23cm
\parindent0mm \pagestyle{empty} \topmargin-1cm
\pagestyle{headings}

%euro-zeichen f. utf8
\usepackage{eurosym}
\DeclareUnicodeCharacter{20AC}{\euro}
\newcommand{\Lsg}{\par \textbf{Lsg.: }}

\begin{document}

\begin{flushleft}%https://www.htw-dresden.de/fakultaet-informatikmathematik/personal/professuren/prof-dr-wilfried-nestler.html
\href{mailto:nestler@informatik.htw-dresden.de}{Prof. Wilfried Nestler}
\end{flushleft}

\large\textbf{Übungsaufgaben (Zahlendarstellung - Teil 1)}

\begin{enumerate}

%1.
\item Die Zahl $(1997)_{10}$ ist manuell (ohne Rechner) in eine Hexadezimalzahl, eine Dualzahl und eine Oktalzahl zu konvertieren. Das Ergebnis ist durch Rückkonvertierung in das Dezimalsystem auf Korrektheit zu überprüfen.

%-------------
\Lsg
\par Umwandlung in Hexadezimal: \\
\begin{tabular}{rrrrr}
    $1997$ : & $16 =$ & $124$ R & $13 \rightarrow $ & $D$ \\
    $ 124$ : & $16 =$ & $  7$ R & $12 \rightarrow $ & $C$ \\
    $   7$ : & $16 =$ & $  0$ R & $ 7 \rightarrow $ & $7$
\end{tabular}
\par Probe: \\
\begin{math}
    7*16^2 + 12*16^1 + 13*16^0 = 1997
\end{math}
%-------------

%2.
\item Die Zahl $(1997)_{10}$ ist manuell in eine Zahl mit Basis $20$ und eine Zahl mit Basis $32$ zu konvertieren. Hierzu ist die Tabelle der binären Repräsentation der Hexadezimalziffern $(0)_{16} \ldots (F)_{16}$ bis $(10011)_{2} = (J)_{20}$ bzw. bis $(11111)_{2} = (V)_{32}$ zu erweitern. Aus der Darstellung mit Basis $32$ ist die Dualdarstellung und daraus die Hexadezimal- und die Oktaldarstellung der Zahl $(1997)_{10}$ zu ermitteln.

%-------------
\Lsg%-------------

%3.
\item Die Zahl $(1234)_{10}$ ist im Siebener- und im Vierersystemarzustellen.

%-------------
\Lsg%-------------

%4.
\item Die Zahl $(430)_{10}$ ist in eine Dualzahl umzuwandeln. Aus der Dualzahl ist die äquivalenten Hexadezimal- und Oktalzahlzubestimmen. Die ermittelte Dualzahl der Form $z=a^n b^n +a^{n-1} b^{n - 1} + \ldots + a^2 b^2 + a^1 b^1 + a^0 b^0 $ ist über das Horner-Schema $ z = ((\ldots(a^n b+a^{n-1} )b + \ldots a^2 ) b + a^1 ) b + a^0$ wieder in die Dezimaldarstellung umzuwandeln.

%-------------
\Lsg%-------------

%5.
\item Die Zahlen sind mit dem Horner-Scema in Dezimalzahlen umzuwandeln: $(375)_8 , (1210)_8 , (888)_9 , (ADA)_{16}$

%-------------
\Lsg%-------------

%6.
\item Manuell ist $(ABC)_{16} + (ABC)_{16}$ zu berechnen. Die Korrektheit des Ergebnisses ist durch Umwandlung in das Dezimalsystem zu verifizieren.

%-------------
\Lsg%-------------

%7.
\item Die folgenden Zahlen ist in das Dualsystem umzuwandeln: $(0,375)_{10}$

%-------------
\Lsg%-------------

%8.
\item Die folgende Zahl ist in das Fünfersystem umzuwandeln: $(0,24)_{10}$

%-------------
\Lsg%-------------

%9.
\item Die folgende Zahl ist in das Oktalsystem umzuwandeln: $(0,2)_{10}$

%-------------
\Lsg%-------------

%10.
\item Die Zahl $(11,625)_{10}$ ist als Dual-, Hexadezimal- und Oktalzahl darzustellen.

%-------------
\Lsg%-------------

%11.
\item Die Aufgabe $11+14=25$ soll in binärer Arithmetik gelöst werden.

%-------------
\Lsg%-------------

%12.
\item Die folgenden Aufgaben sind dual zu berechnen, das Ergebnis ist danach dezimal zu verifizieren und hexadezimal und oktaldarzustellen: \\
\begin{math}
(11011)_2 + (10101)_2  \\
(11101)_2 - (10010)_2  \\
(11011)_2 * (10101)_2  \\
(1001000)_2 : (1000)_2
\end{math}

%-------------
\Lsg%-------------

%13.
\item Die folgenden Aufgaben sind hexadezimal zu berechnen, das Ergebnis ist danach dezimal zu verifizieren: \\
\begin{math}
(ABC)_{16} + (987)_{16}   \\
(DDE9)_{16} - (3F7B)_{16} \\
(CDE)_{16} * (A1)_{16}
\end{math}

%-------------
\Lsg%-------------

%14.
\item Die folgenden Aufgaben sind oktal zu berechnen, das Ergebnis ist danach dezimal zu verifizieren: \\
\begin{math}
(1775)_8 + (766)_8 \\
(3321)_8 - (763)_8 \\
(721)_8 * (72)_8
\end{math}

%-------------
\Lsg%-------------

%15.
\item Die Aufgabe $151.875 + 27.625 = 179.5$ soll in binärer Arithmetik gelöst werden.

%-------------
\Lsg%-------------

%16.
\item Die Aufgabe $10*13=130$ ist in binärer Arithmetik zu lösen

%-------------
\Lsg%-------------

%17.
\item Die Aufgabe $17,375*9,75 = 169,40625$ ist in binärer Arithmetik zu lösen.

%-------------
\Lsg%-------------

%18.
\item Wenn man eine Dualzahl um $n, n>0$ Positionen nach links verschiebt und rechts mit $0$ auffüllt, ist das äquivalent der Multiplikation mit $2 n$ . Es ist $15*8$ sowohl mittels dualer Multiplikation als auch mittels Verschiebung nach links auszurechnen.

%-------------
\Lsg%-------------

%19.
\item Wenn man eine Dualzahl um $n, n>0$ Positione nnach rechts verschiebt und links m it $0$ auffüllt, ist das äquivalent der Division durch $2n$ . Es ist $\frac{120}{4}$ sowohl mittels dualer Division als auch mittels Verschiebung nach rechts auszurechnen.

%-------------
\Lsg%-------------

%20.
\item Eine echt gebrochene Zahl $n, n<1$ \\
\begin{math}
n =
\Sigma
bi \cdot B
i
\end{math}
läßt sich mit Hilfe
$i = -M$
\begin{math}
n = 1/B*(b -1 + 1 / B * ( b -2 + 1/B*(b -3 + ... + 1/B*(b -M +1 + 1 / B * b -M ) \ldots)))
\end{math}

Die folgenden Zahlen sind mit dem Hornerschema in Dezimalzahlen zu konvertieren:

\begin{enumerate}
	\item $(0,375)_8$
	\item $(0,1210)_{10}$
	\item $(0,888)_9$
	\item $(0,ADDA)_{16}$
\end{enumerate}

%-------------
\Lsg%-------------

%21.
\item Mit dem Zweierkompliment soll die negative Dualzahl $z =-1$ in einem Bereich von $4$ Byte dargestellt wer den. Das Ergebnis ist mit $23^2-|z|$ zu vergleichen. Das Ergebnis ist zusätzlich in Form von $8$ Hexadezimalziffern anzugeben. Wie wird $z=|-2^31|$ dargestellt, auch hexadezimal?
Wie lautet dual, hexadezimal und dezimal die größte positive Zahl mit Vorzeichen für $4$ Byte? Wie lautet der Wertebereich für Zahlen mit Vorzeichen in einem $4$ Byte Bereich ?

%-------------
\Lsg%-------------

%22.
\item Man stelle $(-1)_{10}, (-64)_{10} und (-128)_{10}$ für $1$ Byte dual dar. \\

%-------------
\Lsg\begin{math}
      (-1)_{10} => (11111111)_2 \\
     (-64)_{10} => (01000000)_2 xor 255_{10} + 1_{10} \\ => (10111111)_2 + 1_{10} => (11000000)_2 \\
    (-128)_{10} => (10000000)_2 xor 255_{10} + 1_{10} \\ => (01111111)_2 + 1_{10} => (10000000)_2 
\end{math}
%-------------

\end{enumerate}
\end{document}
