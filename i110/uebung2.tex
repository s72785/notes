\documentclass[12pt,a4paper]{scrreprt}
\usepackage{amsmath,amssymb,mathrsfs,dsfont}
\usepackage[utf8]{inputenc}
\usepackage[ngerman]{babel}
\usepackage[pdfborderstyle={/S/U/W 1}]{hyperref}
\usepackage{siunitx}
\usepackage{struktex}
\usepackage{color}
\definecolor{dkgreen}{rgb}{0,0.6,0}
\definecolor{gray}{rgb}{0.5,0.5,0.5}
\definecolor{mauve}{rgb}{0.58,0,0.82}

\usepackage{listings}
\lstset{frame=tb,
  language=C,
  aboveskip=3mm,
  belowskip=3mm,
  showstringspaces=false,
  columns=flexible,
  basicstyle={\small\ttfamily},
  numbers=none,
  numberstyle=\tiny\color{gray},
  keywordstyle=\color{blue},
  commentstyle=\color{dkgreen},
  stringstyle=\color{mauve},
  breaklines=true,
  breakatwhitespace=true,
  tabsize=3
}

\oddsidemargin0mm \evensidemargin3mm \textwidth150mm \textheight23cm
\parindent0mm \pagestyle{empty} \topmargin-1cm
\pagestyle{headings}

%euro-zeichen f. utf8
\usepackage{eurosym}
\DeclareUnicodeCharacter{20AC}{\euro}
\newcommand{\Lsg}{\par \textbf{Lsg.: }}
\newcommand{\Byte}{\,Byte}
\newcommand{\Bit}{\,Bit}

\input binhex

\begin{document}

\large{\textbf{Übungsaufgaben} (Zahlendarstellung - Teil~2)}

\begin{enumerate}

%1.
\item Die Zahlen $312_4, \text{AB1}_{12}, \text{AFFE}_{16}, 7777_8$ sind im Dezimalsystem darzustellen.

%--------------------------
\Lsg%--------------------------

%2.
\item In welchem Zahlensystem stellt folgende Gleichung $42 + 242 = 16 2$ eine wahre Aussage dar?

%--------------------------
\Lsg%--------------------------

%3.
\item Die folgenden Zahlen sind als Dual-, Dezimal- und als Hexadezimalzahlen darzustellen.

\begin{enumerate}
\item $1573,4_8$,
\item $\text{ABC,CBA}_{16}$,
\item $1011.1101_2$,
\item $0,4_8$
\end{enumerate}

%--------------------------
\Lsg%--------------------------

%4.
\item Die folgenden Aufgaben sind zu lösen, indem die Dezimalzahlen zuerst in das Dualsystem und dann im Dualsystem die Addition durchzuführen ist:

\begin{enumerate}
\item $123_{10} + 204_{10}$

%--------------------------
\Lsg%--------------------------

\item $15_{10} + 31_{10}$

%--------------------------
\Lsg%--------------------------

\item $105_{10} + 21_{10}$

%--------------------------
\Lsg%--------------------------

\end{enumerate}

%5.
\item Die Zahlen sind mit dem Horner-Schema in Dezimalzahlen umzuwandeln:

\begin{enumerate}
\item ${0,371}_8$,
\item $\text{0,FFFF}_{16}$,
\item ${0,110011}_2$,
\item $\text{ABD,DE}_{16}$
\end{enumerate}

%--------------------------
\Lsg%--------------------------

%6.
\item Die Zahl ${1100011100,1011011011}_2$ ist als Oktal- und als Hexadezimalzahl durch Abtrennung von je~3 bzw. \si{4}{\Bit} darzustellen. Der Wert als Dezimalzahl ist zu bestimmen.

%--------------------------
\Lsg%--------------------------

%7.
\item Zu den folgenden Zahlen ist das B-Komplement bzgl. der Länge von \si{2}{\Byte} zu bestimmen. Wie lautet der Dezimalwert der Ausgangszahl
und des B-Komplements unter Beachtung des Vorzeichenbits?

\begin{enumerate}
\item $10101_2$,

%--------------------------
\Lsg%--------------------------

\item $785_{10}$,

%--------------------------
\Lsg%--------------------------

\item $\text{AFFE}_{16}$,

%--------------------------
\Lsg%--------------------------

\item $453_{16}$,

%--------------------------
\Lsg%--------------------------

\item $124_{5}$

%--------------------------
\Lsg%--------------------------

\end{enumerate}

%8.
\item Die folgenden Differenzen sind als Dualzahlen im B-Komplement für \si{8}{\Bit} zu notieren und zu addieren. Das Ergebnis ist in das Dezimalsystem zu konvertieren:

\begin{enumerate}
\item $57_{10} - 122_{10}$,

%--------------------------
\Lsg%--------------------------

\item $43_{10} - 11_{10}$,

%--------------------------
\Lsg%--------------------------

\item $17_{10} - 109_{10}$

%--------------------------
\Lsg%--------------------------

\end{enumerate}

%9.
\item Man berechne $777_8 + 777_8$ im Oktalsystem und durch Umwandlung in das Dual-, Dezimal- und Hexadezimalsystem.

%--------------------------
\Lsg%--------------------------

%10.
\item Man berechne $\text{AFFE}_{16} + \text{AFFE}_{16}$ im Hexadezimalsystem und durch Umwandlung in das Dual-, Dezimal- und Oktalsystem.

%--------------------------
\Lsg%--------------------------

%11.
\item Man berechne $3210_4 + 3210_4$ im Vierersystem und wandle das Ergebnis in das Dual-, Dezimal-, Hexadezimal- und Oktalsystem um.

%--------------------------
\Lsg%--------------------------

%12.
\item Mittels der Konvertierung durch sukzessive Multiplikation und Addition gemäß Folie 30 sind die Zahlen wie folgt zu konvertieren.

\begin{enumerate}
\item $11100_2$ nach Basis 10

%--------------------------
\Lsg%--------------------------

\item $555_6$ nach Basis 6

%--------------------------
\Lsg%--------------------------

\item $0110110_2$ nach Basis 4

%--------------------------
\Lsg%--------------------------

\end{enumerate}


%1 3 .
\item Zu folgenden Zahlen ist die float-Darstellung (IEEE-Format) anzugeben:

\begin{enumerate}
\item 125.875,

%--------------------------
\Lsg%--------------------------

\item -13.888,

%--------------------------
\Lsg%--------------------------

\item 0.3,

%--------------------------
\Lsg%--------------------------

\item 0.01953125,

%--------------------------
\Lsg%--------------------------

\item -2.25

%--------------------------
\Lsg%--------------------------

\end{enumerate}

%14.
\item Die Zahl $-32768_{10}$ ist als eine intern im Rechner gespeicherte \si{16}{\Bit}-Binär- und Hexadezimalzahl (im B-Komplement) darzustellen.

%--------------------------
\Lsg%--------------------------

%15.
\item In einer vorzeichenbehafteten \si{2}{\Byte}-Variablen sei rechnerintern der
Wert $(1000 0000 0000 1111)_2$ gespeichert.

\begin{enumerate}
\item Man gebe die hexadezimale Darstellung an.
\item Man wandle den binären Wert in einen dezimalen Zahlenwert mit korrektem Vorzeichenum.
\end{enumerate}

%--------------------------
\Lsg%--------------------------

%16.
\item In einer vorzeichenbehafteten \si{2}{\Byte} Variablen seirechner intern der Wert $\text{A000}_{16}$ gespeichert.
\begin{enumerate}
\item Man gebe die binäre Darstellung an.
\item Man wandle den hexadezimalen Wert in einen dezimalen Zahlenwert mit korrektem Vorzeichen um.
\end{enumerate}

%--------------------------
\Lsg%--------------------------

%17.
\item Mittels der Konvertierung durch sukzessive Multiplikation und Addition
gemäß Folie~30 ist die Zahl $2183_{10}$ in folgende Basen zu konvertieren.

\begin{enumerate}
\item 2

%--------------------------
\Lsg%--------------------------

\item 4

%--------------------------
\Lsg%--------------------------

\item 8

%--------------------------
\Lsg%--------------------------

\item 16

%--------------------------
\Lsg%--------------------------

\end{enumerate}

%18.
\item Man schreibe die größten und kleinsten Zahlen binär, als Exponential-
ausdrücke von 2, als Dezimalzahlen und als Hexadezimalzahlen,
jeweils mit und ohne Vorzeichen, für \si{1}{\Byte} (char, unsigned char),
\si{2}{\Byte} (short, unsigned short), \si{4}{\Byte} (int, long, unsigned int,
unsigned long) und \si{8}{\Byte} (long long, unsigned long long) auf und
begründe das Vorgehen.
Mit Hilfe der C-Operatoren <<, sizeof, ~, *, - ermittle man in je
einem Ausdruck für die Typen unsigned char, unsigned short,
unsigned int, unsigned long und unsigned long long die größten
speicherbaren Werte als Dualzahlen.
Welche Werte sind die minimal speicherbar ?
Für die mit Vorzeichen versehenen Typen char, short, int, long und
long long ermittle man ebenfalls in je einem Ausdruck die minimalen
und die maximalen Werte als Dualzahlen.
Man schreibe hierzu ein C-Programm und gebe für alle Typen die minimalen und maximalen Werte mittels printf dezimal (mit Vorzeichen)
und hexadezimal aus.

Die
von
von
Für
\% Formatangaben von printf lauten für die dezimale Darstellung
char, short, int \%d bzw. \%u, von long \%ld bzw. \%lu,
long long \%lld bzw. \%llu .
die Ausgaben wird \%X, \%lX und \%llX verwendet.

\par \textbf{ Beispiel: }

\begin{lstlisting}
printf(" max. signed long-Wert = \% 20ld \% 20lX\\n", slo, slo);
\end{lstlisting}

%--------------------------
\Lsg%--------------------------

%19.
\item Man wandele folgende Dezimalzahlen in das Dual- und das Hexadezimalsystem um:

\begin{enumerate}
\item $3 5 ,75$
\item $-12,95$
\item $-711,125$
\item $29,6875$
\item $5,7$
\end{enumerate}

%--------------------------
\Lsg%--------------------------

%20.
\item Man transformiere in das Dezimal-, Oktal- und Hexadezimalsystem:

\begin{enumerate}
\item $(11,00011)_2$

%--------------------------
\Lsg%--------------------------

\item $(221,1122)_4$

%--------------------------
\Lsg%--------------------------

\item $(110101,010111)_2$

%--------------------------
\Lsg%--------------------------

\end{enumerate}

%21.
\item Man transformiere in das Dual-, Oktal- und Dezimalsystem:

\begin{enumerate}
\item $\text{3C,FF}_{16}$

%--------------------------
\Lsg%--------------------------

\item $33,0002_4$

%--------------------------
\Lsg%--------------------------

\item $\text{AB,CDE}_{16}$

%--------------------------
\Lsg%--------------------------

\end{enumerate}

%22.
\item Man transformiere in das Dualsystem:

\begin{enumerate}
\item $\text{0,0C3D}_{16}$

%--------------------------
\Lsg%--------------------------

\item $\text{B3C,FE}_{16}$

%--------------------------
\Lsg%--------------------------

\item $\text{76,AEB}_{16}$

%--------------------------
\Lsg%--------------------------

\end{enumerate}

%23.
\item Man berechne im Dualsystem $a+ b, a- b , a*b, a/b$ mit $a= 1000111,01_2$
und $b = 11,11_2$ . Man überprüfe die Rechnung dezimal .

%--------------------------
\Lsg%--------------------------

%24.
\item Man führe in der Zweierkomplementdarstellung (\si{8}{\Bit}) ganzer Dualzahlen folgende Berechnungen durch. Es ist anzugeben, ob ein Überlauf
vorliegt (mit Begründung). Anderenfalls ist das Resultat durch Konvertierung in das Dezimalsystem zu kontrollieren:

\begin{enumerate}
\item 37 - 16

%--------------------------
\Lsg%--------------------------

\item - 64 - 65

%--------------------------
\Lsg%--------------------------

\item 49 + 105

%--------------------------
\Lsg%--------------------------

\item 49 - 105

%--------------------------
\Lsg%--------------------------

\item -33 + 64

%--------------------------
\Lsg%--------------------------

\item 33 - 64

%--------------------------
\Lsg%--------------------------

\end{enumerate}

%2 5 .
\item Man stelle folgende Dezimalzahlen für float (\si{32}{\Bit}) im IEEE-Format dar. Falls notwendig, ist auf die nächstgelegene darstellbare Zahl zu runden:

\begin{enumerate}
\item 3 5 ,75

%--------------------------
\Lsg%--------------------------

\item -0,95

%--------------------------
\Lsg%--------------------------

\item -711

%--------------------------
\Lsg%--------------------------

\item 0,3

%--------------------------
\Lsg%--------------------------

\item 8,125

%--------------------------
\Lsg%--------------------------

\end{enumerate}

%26.
\item Man wandle folgende im \si{32}{\Bit}-Format der IEEE-Norm angegebenen Zahlen für in Dezimalzahlen und den Rest in Ausdrücke von Dualzahlen in Potenzschreibweise um:

\item 
\item 
\item 
\item 
\item 
\item 
\item 
\item 
\item 
\item 
\item 
\item 
\item 
\item 
\item 
\item 
\item 
\item 

%--------------------------
\Lsg%--------------------------

\end{enumerate}

\end{document}
