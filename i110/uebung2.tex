\documentclass[12pt,a4paper]{scrreprt}
\usepackage{amsmath,amssymb,mathrsfs,dsfont}
\usepackage[utf8]{inputenc}
\usepackage[ngerman]{babel}
\usepackage{hyperref}
\usepackage{color}
\definecolor{dkgreen}{rgb}{0,0.6,0}
\definecolor{gray}{rgb}{0.5,0.5,0.5}
\definecolor{mauve}{rgb}{0.58,0,0.82}
\usepackage{listings}
\lstset{frame=tb,
language=C,
aboveskip=3mm,
belowskip=3mm,
showstringspaces=false,
columns=flexible,
basicstyle={\small\ttfamily},
numbers=none,
numberstyle=\tiny\color{gray},
keywordstyle=\color{blue},
commentstyle=\color{dkgreen},
stringstyle=\color{mauve},
breaklines=true,
breakatwhitespace=true,
tabsize=3
}
\oddsidemargin0mm \evensidemargin3mm \textwidth150mm \textheight23cm
\parindent0mm \pagestyle{empty} \topmargin-1cm
\pagestyle{headings}
%euro-zeichen f. utf8
\usepackage{eurosym}
\DeclareUnicodeCharacter{20AC}{\euro}

\begin{document}

Ü bu n gsau fgaben ( Zahle ndar st e llung - Te il 2)

1 . D i e Z a h len 312 ( 4) , A B 1 ( 12) , AFFE (16) , 7777 (8) sind im De z ima lsyste m
d a r z u st e llen.

2 . I n w e l chem Z ahlensystem ste llt f olge nde Gle ic hung 42 + 242 = 16 2
e i n e w a hre A ussage dar ?

3 . D i e Z a hlen 1573,4 ( 8) , A B C ,CBA (16) , 1011.1101 (2) , 0,4 (8) sind a ls
a l s D u a l-, D ezim al- und als He xa de z ima lz a hle n da r z uste lle n.

4 . D i e f o l genden A ufgaben sind z u löse n, inde m die De z ima lz a hle n z ue r st in
d a s D u a lsystem und dann im Dua lsyste m die Addition dur c hz uf ühr e n ist:
) (2)
1 2 3 ( 1 0 ) + 204 ( 10) = (
) (2)
1 5 ( 1 0 ) + 31 ( 10) = (
1 0 5 ( 1 0 ) + 21 ( 10) = (
) (2)

5 . D i e Z a hlen sind m it dem H orne r - Sc he ma in De z ima lz a hle n umz uwa nde ln :
0 , 3 7 1 ( 8) , 0,F F F F ( 16) , 0,110 011 (2) , ABD,DE (16)

6 . D i e Z a hl 1100011100,1011011011 (2) ist a ls Okta l- und a ls He xa de z ima l -
z a h l d u rch A btrennung von je 3 bz w. 4 Bits da r z uste lle n. De r W e r t a ls
D e z i m alzahl ist zu bestim m en.

7 . Z u d e n folgenden Z ahlen ist da s B- Kom ple m e nt bz gl. de r Lä nge von
2 B y t e zu bestim m en. Wie la ute t de r De z ima lwe r t de r Ausga ngsz a hl
u n d d e s B -K omplem ents unte r Be a c htung de s Vor z e ic he nbits ?
1 0 1 0 1 ( 2) , 785 ( 10) , A F F E ( 16 ) , 453 (16) , 124 (5)

8 . D i e f o lgenden D ifferenzen sind a ls Dua lz a hle n im B- Kom ple m e nt f ür
8 B i t z u notieren und zu addie r e n. Da s Er ge bnis ist in da s De z ima lsyste m
z u k o n v ertieren:
57 ( 10) - 1 22 (10) , 43 (10) - 11 (10) , 17 (10) - 109 (10)

9 . Ma n b erechne 777 ( 8) + 777 ( 8) im Okta lsyste m und dur c h Umwa ndlung in
d a s D u al-, D ezimal- und H exa de z ima lsyste m.

1 0 . Ma n b erechne A F F E ( 16) + AFFE (16) im He xa de z ima lsyste m und dur c h
U m w a ndlung in das D ual-, De z ima l- und Okta lsyste m.

1 1 . Ma n b erechne 3210 ( 4) +3210 (4) im Vie r e r syste m und wa ndle da s Er ge bnis
i n d a s D ual-, D ezimal-, H exa de z ima l- und Okta lsyste m um.

1 2 . M i t t e l s der K onvertierung dur c h sukz e ssive Multiplika tion und Additio n
g e m ä ß F olie 30 sind die Z ahle n
nach (
) ( 10)
1 1 1 0 0 ( 2)
nach (
) ( 6)
555(6)
) ( 4)
z u konve r tie r e n.
( 0 1 1 0 110) 2 nach (

1 3 . Z u f o lgenden Z ahlen ist die f loat - Da r ste llung (IEEE- For m at ) a nz u-
g e b e n : 125.875 , -13.888 , 0.3, 0.01953125 , - 2.25
Seite 1 von 3
U ebung_Zahlen_II. fm
W.Nes tler
Üb u n g GI ( Z ah l en d ar st el l u n g - T ei l 2 )

1 4 . D i e Z ahl -32768 ( 10) ist als e ine inte r n im Re c hne r ge spe ic he r te 16 Bit -
Bi n ä r - und H exadezimalzah l ( im B- Komple me nt) da r z uste lle n.

1 5 . I n e i n e r vorzeich enb ehaftete n 2 Byt e - Va r ia ble n se i r e c hne r inte r n de r
W e r t 1000 0000 0000 1111 ( 2) ge spe ic he r t. Ma n ge be die he xa de z ima le
D a r st e llung an. Man w andle de n binä r e n W e r t in e ine n de z ima le n
Z a h l e nw ert m it korrektem V or z e ic he n um.

1 6 . I n e i n er vorzeichenbehaftete n 2 Byte Va r ia ble n se i r e c hne r inte r n de r
W e r t A 000 ( 16) gespeichert. Ma n ge be die binä r e Da r ste llung a n.
M a n w andle den hexadezim ale n W e r t in e ine n de z ima le n Za hle nwe r t
m i t k o rrektem V orzeichen um.

1 7 . M i t t e ls der K onvertierung dur c h sukz e ssive Multiplika tion und Additio n
g e m ä ß F olie 30 sind die Z ah le n
)2
2 1 8 3 ( 1 0 ) nach (
)4
2 1 8 3 ( 1 0 ) nach (
)8
2 1 8 3 ( 1 0 ) nach (
) 16 zu konve r tie r e n.
2 1 8 3 ( 1 0 ) nach (

1 8 . Ma n s chreibe die größten u nd kle inste n Za hle n binär , a ls Expone ntial -
a u sd rü cke von 2, als D ezimalzahle n und a ls He xade zimalzahle n ,
j e w e i ls mit und oh n e V orzeic he n, f ür 1 Byt e ( c har , unsigne d c har) ,
2 B y t e ( sh ort, u n sign ed sh or t) , 4 Byt e (int , long, unsigne d int ,
u n si g ned lon g) und 8 B yte ( long long , unsigne d long long) a uf und
b e g r ü n de das V orgehen.
M i t H ilfe der C - O peratoren < < , siz e of , ~ , * , - e r mittle ma n in je
e i n e m A usdruck für die T ype n unsigne d c har , unsigne d shor t,
u n si g ned in t, u n sign ed lon g und unsigne d long long die gr ößte n
sp e i c herbaren Werte als D u a lz a hle n.
W e l c he Werte sind die m inima l spe ic he r ba r ?
F ü r d ie m it V orzeich en verse he ne n Type n c har , shor t, int , long und
l o n g l on g ermittle m an ebenf a lls in je e ine m Ausdr uc k die minima le n
u n d d ie m aximalen Werte als Dua lz a hle n.
M a n schreibe hierzu ein C -Pr ogr am m und gebe für alle Typen die mini-
m a l e n und m aximalen Werte mitte ls pr int f de z ima l ( mit Vor z e ic he n)
u n d h exadezim al aus.
Die
von
von
Für
% F ormatangab en von pr int f la ute n f ür die de z im ale Da r ste llung
c har, sh ort , in t %d bz w. %u , von long % ld bz w. %lu ,
l on g lon g % lld bzw . % llu .
d ie A usgaben w ird %X, %lX und %llX ve r we nde t.
B e i sp iel: prin tf(" max. sign e d long- We rt = % 20ld % 20lX\n", slo, slo);
Seite 2 von 3
U ebung_Zahlen_II. fm
W.Nes tler
Üb u n g GI ( Z ah l en d ar st el l u n g - T ei l 2 )

1 9 . M a n w andele folgende D ezima lz a hle n in da s Dua l- und da s He xa -
d e z i m a lsystem um:
a . ) 3 5 ,75
b.) -12,95
c .) - 711,125
d.) 29,6875
e .) 5,7
2 0 . M a n t ransform iere in das De z ima l- , Okta l- und He xa de z ima lsyste m:
b.) 11,00011 (2)
c .) 221,1122 (4)
a . ) 1 1 0101,010111 ( 2)
2 1 . M a n t ransformiere in das Dua l- , Okta l- und De z ima lsyste m:
b.) 3C,FF (16)
c .) 33,0002 (4)
a . ) A B ,C D E ( 16)

2 2 . M a n t ransformiere in das Dua lsyste m:
b.) 0,0C3D (16)
a . ) B 3C ,F E ( 16)
c .) 76,AEB (16)

2 3 . M a n berechne im D ualsystem a+ b, a- b , a*b, a/b mit a= 1000111,01 (2)
u n d b = 11,11 ( 2) . Man überpr üf e die Re c hnung de z ima l.

2 4 . Ma n f ühre in der Z w eierkomple me ntda r ste llung ( 8 Bit) ga nz e r Dua l-
z a h l e n folgende B erechnunge n dur c h. Es ist a nz uge be n, ob e in Übe r la u f
v o r l i e gt (mit B egründung). Ande r e nf a lls ist da s Re sulta t dur c h Konve r -
t i e r u n g in das D ezim alsyste m z u kontr ollie r e n:
a . ) 3 7 - 16
b.) - 64 - 65
c .) 49 + 105
d . ) 4 9 - 105
e .) - 33 + 64
f .) 33 - 64

2 5 . Ma n s telle folgende D ezim alz a hle n f ür f loat ( 32 Bit ) im IEEE- For ma t
d a r . F a lls notw endig, ist auf die nä c hstge le ge ne da r ste llba r e Za hl z u
r u n d e n:
a . ) 3 5 ,75
b.) -0,95
c .) - 711
d.) 0,3 e .) 8,125

2 6 . M a n w andle folgende im 32- Bit - For ma t de r IEEE- Nor m a nge ge be ne n
Z a h l e n für a.) - d.) in D ezima lz a hle n und de n Re st in Ausdr üc ke von
D u a l z a hlen in P otenzschreib we ise um:
a.)
b.)
c.)
d.)
e.)
f.)
g.)
h.)
i.)
j.)
k.)
l.)
m.)
n.)
o.)
p.)
q.)
r.)
1
0
1
0
0
0
1
1
0
1
0
1
0
0
0
0
1
1
|
|
|
|
|
|
|
|
|
|
|
|
|
|
|
|
|
|
1001
1000
0111
0111
1111
1111
1111
1111
1111
1111
0000
0000
0000
1111
0000
0000
0000
0000
0001
0101
1001
1000
1111
1111
1111
1111
1111
1111
0000
0000
0001
1110
0000
0000
0000
0000
|
|
|
|
|
|
|
|
|
|
|
|
|
|
|
|
|
|
0011
1011
1011
0000
1111
0000
1111
0000
0000
0000
0000
0000
0000
1111
0000
1111
0000
1111
0011
0010
1001
0000
1111
0000
1111
0000
0000
0000
0000
0000
0000
1111
0000
1111
0000
1111
Seite 3 von 3
0000
0000
1000
0000
1111
0000
1111
0000
0000
0000
0000
0000
0000
1111
0000
1111
0000
1111
0001
0000
0000
0000
1111
0000
1111
0000
0000
0000
0000
0000
0000
1111
0000
1111
0000
1111
0101
0000
0000
0000
1111
0000
1111
0000
0000
0000
0000
0000
0000
1111
0000
1111
0000
1111
000
000
000
000
111
001
111
001
000
000
000
000
000
111
001
111
001
111

\end{document}
