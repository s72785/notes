\documentclass[12pt,a4paper]{scrreprt}
\usepackage{amsmath,amssymb,mathrsfs,dsfont}
\usepackage[utf8]{inputenc}
\usepackage[ngerman]{babel}
\usepackage[pdfborderstyle={/S/U/W 1}]{hyperref}
\usepackage{color}
\definecolor{dkgreen}{rgb}{0,0.6,0}
\definecolor{gray}{rgb}{0.5,0.5,0.5}
\definecolor{mauve}{rgb}{0.58,0,0.82}
\usepackage{listings}
\lstset{frame=tb,
language=C,
aboveskip=3mm,
belowskip=3mm,
showstringspaces=false,
columns=flexible,
basicstyle={\small\ttfamily},
numbers=none,
numberstyle=\tiny\color{gray},
keywordstyle=\color{blue},
commentstyle=\color{dkgreen},
stringstyle=\color{mauve},
breaklines=true,
breakatwhitespace=true,
tabsize=3
}
\oddsidemargin0mm \evensidemargin3mm \textwidth150mm \textheight23cm
\parindent0mm \pagestyle{empty} \topmargin-1cm
\pagestyle{headings}
%euro-zeichen f. utf8
\usepackage{eurosym}
\DeclareUnicodeCharacter{20AC}{\euro}

\begin{document}

\large{\textbf{Übungsaufgaben} ( Zahlendarstellung - Teil 2)}

\begin{enumerate}

%1.
\item Die Zahlen $312_(4), AB1_(12), AFFE_(16), 7777_(8)$ sind im Dezimalsystem darzustellen.

%2.
\item In welchem Zahlensystem stellt folgende Gleichung $42 + 242 = 16 2$ eine wahre Aussage dar?

%3.
\item Die Zahlen $1573,4_(8), ABC,CBA_(16), 1011.1101_(2), 0,4_(8)$ sind als Dual-, Dezimal- und als Hexadezimalzahlen darzustellen.

%4.
\item Die folgenden Aufgaben sind zu lösen, indem die Dezimalzahlen zuerst in das Dualsystem und dann im Dualsystem die Addition durchzuführen ist:

\begin{enumerate}
\item $123_(10) + 204_(10)$
\item $15_(10) + 31_(10)$
\item $105_(10) + 21_(10)$
\end{enumerate}

%5.
\item Die Zahlen sind mit dem Horner-Schema in Dezimalzahlen umzuwandeln: $0, 371_(8), 0,FFFF_(16), 0,110011_(2), ABD,DE_(16)$

%6.
\item Die Zahl $1100011100,1011011011_(2)$ ist als Oktal- und als Hexadezimalzahl durch Abtrennung von je 3 bzw. 4\,Bits darzustellen. Der Wert als Dezimalzahl ist zu bestimmen.

%7.
\item Zu den folgenden Zahlen ist das B-Komplement bzgl. der Länge von 2 Byte zu bestimmen. Wie lautet der Dezimalwert der Ausgangszahl
und des B-Komplements unter Beachtung des Vorzeichenbits?

\begin{enumerate}
\item $10101_(2)$,
\item $785_(10)$,
\item $AFFE_(16)$,
\item $453_(16)$,
\item $124_(5)$
\end{enumerate}

%8.
\item Die folgenden Differenzen sind als Dualzahlen im B-Komplement für 8 Bit zu notieren und zu addieren. Das Ergebnis ist in das Dezimalsystem zu konvertieren:

\begin{enumerate}
\item $57_( 10) - 122_(10)$,
\item $43_(10) - 11_(10)$,
\item $17_(10) - 109_(10)$
\end{enumerate}

%9.
\item Man berechne $777_(8) + 777_(8)$ im Oktalsystem und durch Umwandlung in das Dual-, Dezimal- und Hexadezimalsystem.

%10.
\item Man berechne $AFFE_(16) + AFFE_(16)$ im Hexadezimalsystem und durch Umwandlung in das Dual-, Dezimal- und Oktalsystem.

%11.
\item Man berechne $3210_(4) + 3210_(4)$ im Vierersystem und wandle das Ergebnis in das Dual-, Dezimal-, Hexadezimal- und Oktalsystem um.

%12.
\item Mittels der Konvertierung durch sukzessive Multiplikation und Addition gemäß Folie 30 sind die Zahlen wie folgt zu konvertieren.

\begin{enumerate}
\item $11100_(2)$ nach Basis 10
\item $555_(6)$ nach Basis 6
\item $0110110_2$ nach Basis 4
\end{enumerate}


( 0 1 1 0 110) 2 nach (

%1 3 .
\item Z u f o lgenden Z ahlen ist die f loat - Da r ste llung (IEEE- For m at ) a nz u-
g e b e n : 125.875 , -13.888 , 0.3, 0.01953125 , - 2.25

%1 4 .
\item D i e Z ahl -32768 ( 10) ist als e ine inte r n im Re c hne r ge spe ic he r te 16 Bit -
Bi n ä r - und H exadezimalzah l ( im B- Komple me nt) da r z uste lle n.

%1 5 .
\item I n e i n e r vorzeich enb ehaftete n 2 Byt e - Va r ia ble n se i r e c hne r inte r n de r
W e r t 1000 0000 0000 1111 ( 2) ge spe ic he r t. Ma n ge be die he xa de z ima le
D a r st e llung an. Man w andle de n binä r e n W e r t in e ine n de z ima le n
Z a h l e nw ert m it korrektem V or z e ic he n um.

%1 6 .
\item I n e i n er vorzeichenbehaftete n 2 Byte Va r ia ble n se i r e c hne r inte r n de r
W e r t A 000 ( 16) gespeichert. Ma n ge be die binä r e Da r ste llung a n.
M a n w andle den hexadezim ale n W e r t in e ine n de z ima le n Za hle nwe r t
m i t k o rrektem V orzeichen um.

%17.
\item M i t t e ls der K onvertierung dur c h sukz e ssive Multiplika tion und Additio n
g e m ä ß F olie 30 sind die Z ah le n
)2
2 1 8 3 ( 1 0 ) nach (
)4
2 1 8 3 ( 1 0 ) nach (
)8
2 1 8 3 ( 1 0 ) nach (
) 16 zu konve r tie r e n.
2 1 8 3 ( 1 0 ) nach (

%1 8 .
\item Ma n s chreibe die größten u nd kle inste n Za hle n binär , a ls Expone ntial -
a u sd rü cke von 2, als D ezimalzahle n und a ls He xade zimalzahle n ,
j e w e i ls mit und oh n e V orzeic he n, f ür 1 Byt e ( c har , unsigne d c har) ,
2 B y t e ( sh ort, u n sign ed sh or t) , 4 Byt e (int , long, unsigne d int ,
u n si g ned lon g) und 8 B yte ( long long , unsigne d long long) a uf und
b e g r ü n de das V orgehen.
M i t H ilfe der C - O peratoren < < , siz e of , ~ , * , - e r mittle ma n in je
e i n e m A usdruck für die T ype n unsigne d c har , unsigne d shor t,
u n si g ned in t, u n sign ed lon g und unsigne d long long die gr ößte n
sp e i c herbaren Werte als D u a lz a hle n.
W e l c he Werte sind die m inima l spe ic he r ba r ?
F ü r d ie m it V orzeich en verse he ne n Type n c har , shor t, int , long und
l o n g l on g ermittle m an ebenf a lls in je e ine m Ausdr uc k die minima le n
u n d d ie m aximalen Werte als Dua lz a hle n.
M a n schreibe hierzu ein C -Pr ogr am m und gebe für alle Typen die mini-
m a l e n und m aximalen Werte mitte ls pr int f de z ima l ( mit Vor z e ic he n)
u n d h exadezim al aus.
Die
von
von
Für
% F ormatangab en von pr int f la ute n f ür die de z im ale Da r ste llung
c har, sh ort , in t %d bz w. %u , von long % ld bz w. %lu ,
l on g lon g % lld bzw . % llu .
d ie A usgaben w ird %X, %lX und %llX ve r we nde t.
B e i sp iel: prin tf(" max. sign e d long- We rt = % 20ld % 20lX\n", slo, slo);

%1 9 .
\item M a n w andele folgende D ezima lz a hle n in da s Dua l- und da s He xa -
d e z i m a lsystem um:
a . ) 3 5 ,75
b.) -12,95
c .) - 711,125
d.) 29,6875
e .) 5,7
2 0 . M a n t ransform iere in das De z ima l- , Okta l- und He xa de z ima lsyste m:
b.) 11,00011 (2)
c .) 221,1122 (4)
a . ) 1 1 0101,010111 ( 2)
2 1 . M a n t ransformiere in das Dua l- , Okta l- und De z ima lsyste m:
b.) 3C,FF (16)
c .) 33,0002 (4)
a . ) A B ,C D E ( 16)

2 2 . M a n t ransformiere in das Dua lsyste m:
b.) 0,0C3D (16)
a . ) B 3C ,F E ( 16)
c .) 76,AEB (16)

2 3 . M a n berechne im D ualsystem a+ b, a- b , a*b, a/b mit a= 1000111,01 (2)
u n d b = 11,11 ( 2) . Man überpr üf e die Re c hnung de z ima l.

2 4 . Ma n f ühre in der Z w eierkomple me ntda r ste llung ( 8 Bit) ga nz e r Dua l-
z a h l e n folgende B erechnunge n dur c h. Es ist a nz uge be n, ob e in Übe r la u f
v o r l i e gt (mit B egründung). Ande r e nf a lls ist da s Re sulta t dur c h Konve r -
t i e r u n g in das D ezim alsyste m z u kontr ollie r e n:
a . ) 3 7 - 16
b.) - 64 - 65
c .) 49 + 105
d . ) 4 9 - 105
e .) - 33 + 64
f .) 33 - 64

2 5 . Ma n s telle folgende D ezim alz a hle n f ür f loat ( 32 Bit ) im IEEE- For ma t
d a r . F a lls notw endig, ist auf die nä c hstge le ge ne da r ste llba r e Za hl z u
r u n d e n:
a . ) 3 5 ,75
b.) -0,95
c .) - 711
d.) 0,3 e .) 8,125

%26.
\item Man wandle folgende im 32-Bit-Format der IEEE-Norm angegebenen
Zahlen für in Dezimalzahlen und den Rest in Ausdrücke von Dualzahlen in Potenzschreibweise um:

\item 
\item 
\item 
\item 
\item 
\item 
\item 
\item 
\item 
\item 
\item 
\item 
\item 
\item 
\item 
\item 
\item 
\item 

\end{enumerate}

\end{document}
