\documentclass{scrartcl}

% Kodierung dieser Datei angeben
\usepackage[utf8]{inputenc}

% Schönere Schriftart laden
\usepackage[T1]{fontenc}
\usepackage{lmodern}

% Deutsche Silbentrennung verwenden
\usepackage[ngerman]{babel}

% Bessere Unterstützung für PDF-Features
\usepackage[breaklinks=true]{hyperref}

\KOMAoptions{%
  % Absätze durch Abstände
  parskip=full,%
  % Satzspiegel berechnen lassen
  DIV=calc%
}

% Mathematikumgebungen von der AMS laden
\usepackage{amsmath}
\usepackage{amssymb}

% TikZ laden
\usepackage{tikz}

% Verwendete TikZ-Bibliotheken laden
\usetikzlibrary{positioning,shapes.geometric}

\begin{document}
  \begin{tikzpicture}[
      % Stil für Ein- und Ausgabe
      io/.style={trapezium, trapezium left angle=70, trapezium right angle=110, fill=magenta!10, draw=magenta},
      % Stil für Operationen
      op/.style={rectangle, fill=orange!10, draw=orange},
      % Stil für Entscheidungen
      cn/.style={diamond, aspect=2, inner sep=2pt, fill=red!10, draw=red},
      % Distanz zwischen den Knoten
      node distance=5mm]
    % Knoten
    \node[io] (in) {Eingabe $a,b$};
    \node[op, below=of in] (div) {$r=a \mod b$};
    \node[op, below=of div] (set) {$a=b,\ b=r$};
    \node[cn, below=of set] (cond) {$b=0?$};
    \node[io, below=of cond] (out) {Ausgabe $a$};
    % Kanten
    \path[->]
      (in) edge (div)
      (div) edge (set)
      (set) edge (cond)
      (cond) edge node[right] {Ja} (out);
    \draw[->] (cond) -- (current bounding box.north east |- cond) -- +(.5ex,0) node[below] {Nein} |- (div);
%    \draw[->] (cond) -- node[below] {Nein} ++(1.5,0) |- (div);
  \end{tikzpicture}
\end{document}
