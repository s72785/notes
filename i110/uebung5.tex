\documentclass[12pt,a4paper,ngerman]{scrreprt}
\usepackage{amsmath,amssymb,mathrsfs,dsfont}
\usepackage[utf8]{inputenc}
\usepackage[ngerman]{babel}
\usepackage[pdfborderstyle={/S/U/W 1}]{hyperref}
\usepackage[markup=nocolor,deletedmarkup=xout]{changes}
\usepackage{siunitx}
\usepackage{struktex}
\usepackage{color}
\usepackage{tikz}
\usetikzlibrary{shapes.geometric,arrows}% für ellipse
\usepackage{verbatim}

\newcommand\mrahmen[3][]{%
  \tikz[anchor=base,baseline]\node[inner sep=2pt,draw=#2,#1]{$\displaystyle#3\mathstrut$};}
\colorlet{mfarbe}{red}
\newcommand{\kringel}[1]{\mrahmen[circle]{mfarbe}{#1}}

\usepackage{listings}
\lstset{frame=tb,
  language=C,
  aboveskip=3mm,
  belowskip=3mm,
  showstringspaces=false,
  columns=flexible,
  basicstyle={\small\ttfamily},
  numbers=none,
  numberstyle=\tiny\color{gray},
  keywordstyle=\color{blue},
  commentstyle=\color{violet},
  stringstyle=\color{pruple},
  breaklines=true,
  breakatwhitespace=true,
  tabsize=3
}

\oddsidemargin0mm \evensidemargin3mm \textwidth150mm \textheight23cm
\parindent0mm \pagestyle{empty} \topmargin-1cm
\pagestyle{headings}

%euro-zeichen f. utf8
\usepackage{eurosym}
\newcommand{\Lsg}{\par \textbf{Lsg.: \hfill }}
\newcommand{\Byte}{\,Byte}
\newcommand{\Bit}{\,Bit}

\input binhex


\begin{document}

\large{\textbf{Übungsaufgaben} (Teil~5)}

\begin{enumerate}

\item Aufgabe %1:

Die Funktion $ e^x $ ist wie folgt über eine unendliche Reihe definiert: \\

\begin{math}
e^x = 1 + \frac{x^1}{1!} + \frac{x^2}{2!} + \frac{x^3}{3!} + \frac{x^4}{4!} + ...
= \Sigma_{n=0}^{\infty} \frac{x^n}{n!}
\end{math}

\par Formulieren Sie einen \textbf{Algorithmus in Form eines Flussdiagrammes}, der die Berechnung der Summe abbricht, wenn zwei aufeinanderfolgende Summen eine Differenz kleiner als $\epsilon$ haben.
Die Varaible $x$ und $\epsilon$ (z.B. $10^{-8}$ ) sind einzulesen. Der genäherte Wert für $ e^x $ und das erreichte $ n $ sind auszugeben.

%-------------------------------
\Lsg

%<<<<<<<<<<<<
% Define block styles for flowchart
\tikzstyle{decision} = [diamond, draw, fill=green!20, 
    text width=4.25em, text badly centered, node distance=3cm, inner sep=0pt,
    minimum height=2em] 
%\tikzstyle{block} = [rectangle, draw, fill=blue!20, text width=5em, text centered, rounded corners, minimum height=2em]
\tikzstyle{line} = [draw, -latex']
\tikzstyle{cloud} = [draw, ellipse,fill=red!20, node distance=3cm,
    minimum height=2em]
\tikzstyle{io} = [trapezium, trapezium left angle=70, trapezium right angle=110, minimum width=3cm, minimum height=1em, text centered, draw=black, fill=violet!30]
\tikzstyle{process} = [rectangle, minimum width=2cm, minimum height=1cm, text centered, draw=black, fill=orange!30]
%>>>>>>>>>>>>
\begin{tikzpicture}[node distance = 1.5cm, auto, thick, nodes = {align = center},
    >=latex]

	\node [cloud] (start) {Start};
	\node [io, right of=start, node distance=4.5cm] (readx) {x};
	\path [line] (start) -- (readx);
	\node [io, right of=readx, node distance=4.5cm] (reade) {epsilon};
	\path [line] (readx) -- (reade);
	\node [process, below of=reade, node distance=1.5cm] (initsum2) {sum2=1.0};
	\path [line] (reade) -- (initsum2);
	\node [process, left of=initsum2, node distance=4.5cm] (initsum1) {sum1=0.0};
	\path [line] (initsum2) -- (initsum1);
	\node [process, left of=initsum1, node distance=4.5cm] (initn) {n=0};
	\path [line] (initsum1) -- (initn);
	\node [decision, below of=initn, node distance=3cm] (decide) { (sum2 - sum1) gt epsilon };
		\path [line] (initn) -- (decide);
    		\node [process, right of=decide, yshift=-1.5cm, node distance=4.5cm] (setsum1) {sum1 = sum2};
		\path [line] (decide) -- node [near start, below] {yes} (setsum1);
    		\node [process, right of=setsum1, yshift=1.5cm, node distance=4.5cm] (setsum2) {sum2 += $\frac{pow(x, n)}{fac(n)}$};
		\path [line] (setsum1) -- (setsum2);
    		\node [process, above of=setsum1, yshift=1.5cm, node distance=1.5cm] (setn) {n++};
		\path [line] (setsum2) -- (setn);
		\path [line] (setn) -- (decide);
    		\node [cloud, below of=decide, node distance=3cm] (stop) {Stop};
		\path [line] (decide) -- node [near start] {no} (stop);

\end{tikzpicture}

\begin{lstlisting}
    #include<math.h>
    void ehochx(double x, double epsilon, double* sum1, int* n){
      double sum2=1.0;
      *sum1=0.0;
      *n=0;
      while( (sum2 - *sum1) > epsilon  ){
        *sum1 = sum2;
        sum2 += pow(x, *n)/fac(*n);
        *n++;
      }
    }
\end{lstlisting}
%-------------------------------

\item Aufgabe %2:

Die folgenden Summen sind solange zu berechnen, bis der Absolutbetrag des Quotienten eines Summanden und der bisherigen Summe kleiner einer vorher eingelesenen positiven Schranke Epsilon ist.
Die \textbf{Algorithmen sollen als Struktogramme, Flussdiagramme bzw. Programmablaufpläne} formuliert werden, \textit{\textbf{optional} können C-Programme formuliert} werden.

\begin{enumerate}

\item Berechnung von $\pi$ nach Leibniz: $ \Sigma_{n=0}^{\infty} \frac{(-1)^n}{2n+1} = 1-\frac{1}{3}+\frac{1}{5}-\frac{1}{7}+\frac{1}{9}-\ldots=\frac{\pi}{4} $

%-------------------------------
\Lsg
\begin{lstlisting}
    #include<math.h>
    void leibnitzpi(double epsilon, double* sum1, int* n){
      double sum2=1.0;
      *sum1=1.0;
      *n=1;
      while( (sum2 / *sum1) >= epsilon ){
        *sum1 = sum2;
        sum2 += pow(-1,*n)/(*n*2 + 1);
        *n+=2;
      }
    }
\end{lstlisting}
%-------------------------------

\item Berechnung von $\pi$ nach Bailey-Borwein-Plouffe (1996): $ \pi = \Sigma_{k=0}^{\infty} \frac{1}{16^k} \left( \frac{4}{8k+1} \frac{2}{8k+4} \frac{1}{8k+5} \frac{1}{8k+6} \right)  $

%-------------------------------
\Lsg
%-------------------------------

\item Berechnung von $ln(x)$:

%-------------------------------
\Lsg
%-------------------------------

\item Berechnung von $cos(x)$:

%-------------------------------
\Lsg
%-------------------------------

\item Berechnung von $arctan(x) \text{für} |x|<=1$:

%-------------------------------
\Lsg
%-------------------------------

\end{enumerate}

\item Aufgabe %3:

Auf Produkte $x, y$ und $z$ gibt es bei einem Händler folgende Rabatte:
\par Beim Kauf von \textit{Produkt $x$} gilt: \\
Bis zum 9. Stück. gibt es keinen Rabatt, für das 10. -- 19. Stück gibt es 10\% Rabatt (pro Stück).
Für \numrange{20}{39}{ Stück} gibt es \si{15}{\%} Rabatt pro Stück.
Für \numrange{40}{49}{ Stück} gibt es nur noch \si{5}{\%} Rabatt pro Stück.
Für \si{50}{Stück} und weitere Stücke des Produktes x gibt es keinen Rabatt mehr.
\par Beim Kauf von \textit{Produkt $y$} gilt: \\
Ab \si{6}{ Stück} gibt es \EUR{5,-} Rabatt pro Stück für alle Stücke von $1$ an.
Ab \si{10}{ Stück} gibt es \EUR{10,-} Rabatt pro Stück für alle Stücke von $1$ an.
Der Gesamtrabatt für Produkt y ist auf maximal \EUR{140} beschränkt.
\par Beim Kauf von \textit{Produkt $z$} gilt: \\
Auf jedes Stück gibt es 8\% Rabatt, ab \si{25}{ Stück} gibt es keinen Rabatt mehr.
Für die drei Produkte $x, y$ und $z$ gemeinsam ist die Gesamtrabattsumme auf \EUR{500}
beschränkt, d.h. beim Überschreiten von \EUR{500} Gesamtrabattsumme für einen Warenkorb
aus $x, y$ und $z$ gibt es keinen Rabatt mehr!
Es ist ein Flussdiagramm zu formulieren, welches die Anzahl der gekauften Produkte von
$x, y$ und $z$ und die Preise/Produkt von $x, y$ und $z$ jeweils getrennt einliest und aufgrund der
genannten Vorgaben algorithmisch den Gesamtpreis sowohl ohne als auch mit Rabatt ermittelt
und ausgibt. Der Geldbetrag des erzielten Rabatts ist ebenfalls auszugeben. Beim Einlesen soll sichergestellt werden, dass Anzahl $0$ und Preis \EUR{500} ist.
\par \textbf{Optional} kann der Algorithmus als ein C-Programm geschrieben werden.

%-------------------------------
\Lsg
%-------------------------------

\item Aufgabe %4:

\begin{enumerate}
\item Führen Sie das Siebverfahren des Eratosthenes manuell auf Papier für die natürlichen Zahlen von \numrange{1}{60} durch.

%-----------------
\Lsg
%\begin{flushleft}
\begin{tabular}{cccccccccc}
 \xout{1}  &   \kringel{2}  &   \kringel{3}  &   \xout{4}  &   \kringel{5}  &   \xout{6}  &   \kringel{7}  &   \xout{8}  &   \xout{9}  &  \xout{10} \\
\mrahmen[circle]{mfarbe}{11}  &  \xout{12}  &  \kringel{13}  &  \xout{14}  &  \xout{15}  &  \xout{16}  &  \kringel{17}  &  \xout{18}  &  \kringel{19}  &  \xout{20} \\
 \xout{21}  &  \xout{22}  &  \kringel{23}  &  \xout{24}  &  \xout{25}  &  \xout{26}  &  \xout{27}  &  \xout{28}  &  \kringel{29}  &  \xout{30} \\
 \kringel{31}  &  \xout{32}  &  \xout{33}  &  \xout{34}  &  \xout{35}  &  \xout{36}  &  \kringel{37}  &  \xout{38}  &  \xout{39}  &   \xout{40} \\
\mrahmen[circle]{mfarbe}{41}  &  \xout{42}  &  \kringel{43}  &  \xout{44}  &  \xout{45}  &  \xout{46}  &  \kringel{47}  &  \xout{48}  &  \xout{49}  &  \xout{50} \\
 \xout{51}  &  \xout{52}  &  \kringel{53}  &  \xout{54}  &  \xout{55}  &  \xout{56}  &  \xout{57}  &  \xout{58}  &  \kringel{59}  &  \xout{60} \\
\end{tabular}
%\end{flushleft}
%-----------------

\item Formulieren Sie einen Programmablaufplan bzw. ein Flussdiagramm bzw. ein Struktogramm, welches das Siebverfahren von Eratosthenes bis $n, n > 0$ realisiert.

%-------------------------------
\Lsg
    solange i<n
      j = wähle nöchste zahl aus ungestrichenen zahlen
      solange j<n
        streiche vielfache von j
%-------------------------------

\item Führen Sie das optimierte Verfahren, \href{run:Algorithmen_Heron_Euklid_Prim.pdf}{Algorithmen, S.~11, Vorlesung~4} manuell auf Papier für die natürlichen Zahlen von \numrange{1}{60} durch.

%-------------------------------
\Lsg
%-------------------------------

\item Formulieren Sie einen Programmablaufplan bzw. ein Flussdiagramm bzw. ein Struktogramm, welches das optimierte Verfahren bis $n, n > 0$ realisiert.

%-------------------------------
\Lsg
%-------------------------------

\item \textbf{optional:} Schreiben Sie ein C-Programm zum Sieb- und zum optimierten Verfahren
\end{enumerate}

\item Aufgabe %5:

Man teste den rekursiven Algorithmus zur Bestimmung der Primfaktoren am
Beispiel PFR(330). Es ist ein Programmablaufplan bzw. ein Flussdiagramm bzw. ein Struktogramm zu formulieren, welches das rekursive Verfahren beschreibt.

%-------------------------------
\Lsg
%-------------------------------

\item Aufgabe %6:

Man teste den iterativen Algorithmus zur Bestimmung der Primfaktoren am
Beispiel primefactorization(330). Es ist ein Programmablaufplan bzw. ein Flussdiagramm bzw. ein Struktogramm zu formulieren, welches das iterative Verfahren beschreibt.

%-------------------------------
\Lsg
%-------------------------------

\item Aufgabe %7:

Man formuliere Programmablaufpläne für eine einfach verkettete Liste mit

\begin{lstlisting}
struct el {
	int value;				/* Daten */
	struct el *next;			/* Zeiger auf naechstes Element */
}
struct el *start = 0, *p = 0;			/* start ist Startzeiger, p Hilfszeiger */ und gegebenem p mit
p = (struct el *)malloc(sizeof(struct el)); 	/* Speicherplatz p auf dem heap */
\end{lstlisting}

für

\begin{lstlisting}
Insertfirst(inout: start, in: p)	/* Einfuegen p nach start */
Insertlast(inout: start, in: p)		/* Einfuegen p nach start */
Removefirst(inout: start)		/* Loeschen erstes Element nach start */
Removelast(inout: start)		/* Loeschen letztes Element nach start */
int Anzahl(inout: start)		/* Rueckgabe Anzahl der Listenelement */
Show(in: start)				/* Anzeige aller value-Werte */
RemovePos(inout: start, in: pos)	/* Element an der Stelle pos loeschen (pos = 0 ..) */
Removeall(inout: start)			/* Alle Elemente loeschen */
\end{lstlisting}

%-------------------------------
\Lsg
%-------------------------------

\item Aufgabe %8:

Gesucht ist ein Algorithmus in Form eines Flussdiagrammes, Programmablaufplanes oder Struktogrammes zur Berechnung von
\begin{math}
sin(x)=\Sigma_{i=0}^{\infty} (-1)^i \cdot \frac{x^{2 \cdot i+1}}{(2 \cdot i + 1)!}
\end{math}
mit ( $0 < x <= \pi/2$ ) . Die Berechnung ist abzubrechen, wenn der absolute Betrag des Quotienten
aus dem ersten vernachlässigten Reihenglied und der Partialsumme kleiner als der Wert einer
positiven Schranke epsilon ist. \par Formulieren Sie den Algorithmus \textbf{optional} als C-Programm.

%-------------------------------
\Lsg
%-------------------------------

\item Aufgabe %9:

Es ist je ein Flussdiagramm für folgende Aufgaben zu formulieren, wobei die Vektoren, Matrizen und deren Grenzen als Parameter gegeben sind, die Zulässigkeit der mathematischen Operation (korrekte Indizes) ist jedoch zu überprüfen:

\begin{enumerate}
\item Skalarprodukt zweier Vektoren,

%-------------------------------
\Lsg
%-------------------------------

\item Summe zweier Vektoren,

%-------------------------------
\Lsg
%-------------------------------

\item Multiplikation Matrix mit Vektor,

%-------------------------------
\Lsg
%-------------------------------

\item Multiplikation Matrix mit Matrix,

%-------------------------------
\Lsg

%\begin{tikzpicture}
%\end{tikzpicture}
%-------------------------------

\item Transponieren einer quadratischen Matrix.

%-------------------------------
\Lsg

* durchlaufe die spalten
  * schreibe jedes element in sie Zeilen der zweiten matrix
* zurück solange noch weitere spalten existieren

\begin{comment}
\begin{flushleft}
\begin{struktogramm}(95,40)[Programmname]
\while{\( test \)}test
\whileend
\case[10]{3}{3}{Signum(x)}{-1}
	\assign{\(z \gets - \frac{1}{x}\)}
\switch{0}
	\assign{Ausgabe: Division durch 0}
\switch[r]{1}
	\assign{\(z \gets \frac{1}{x}\)}
\caseend
\end{struktogramm}
\end{flushleft}
\end{comment}

%-------------------------------

\end{enumerate}

\end{enumerate}

\end{document}
