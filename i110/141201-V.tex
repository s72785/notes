\usepackage{comment}

\begin{comment}
Z208, GdI-V, Prof. Nestler
\end{comment}

\section Binärer Baum

24
22
20
18
21
19
17
25
44
40
36
32
28
30
34
38
42
46
48
33
35
37
39
41
0

24:	<22
22: <20
20: <18
20: >21
18: >19
18: <17
24: >25
25: >44
44: <40
40: <36
36: <32
32: <28
28: >30
32: >34
36: >38
40: >42
44: >46
46: >48
34: <33
34: >35
38: <37
38: >39
42: <41
17: <0

Die tiefste Ebene gibt die Höhe des Baumes an. Hier: 8
Blätter sind die Endpunkte. Hier: 10
Innere Knoten sind Nicht-Blätter. Hier: 15
Pfadlänge des Baumes (in einer Ebene) sind die Summe der Pfade bis zur Ebene. Damit ist es das Produkt der Anzahl der Elemente mit der Ebene für jede Ebene.
Algorithmus s.a. binbaum.c .

Ausgabe-Methoden:

inorder-Durchlauf (zuerst links, akt. Elm., rechter Zweig)
ganz nach links: 0
Ausgabe von links unten nach oben und rechts
die Ausgabe ist von links niedrig nach rechts hoch sortiert

preorder-Durchlauf (zuerst Elm., dann linker Teil, dann rechter Teil)

postorder-Durchlauf (linker Zweig, rechter Zweig, Elm. selbst)

\section Sortierverfahren

Interessante Eigenschaften von Sortieralgorithmen sind
* Ressourcenbedarf
* Laufzeit
* Stabilität

Inplace-Verfahren benötigen keinen zusätzlichen Speicherplatz.
Beispiel: Bubblesort, Insertsort, Selectionsort
