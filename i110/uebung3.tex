\documentclass[12pt,a4paper]{scrreprt}
\usepackage{amsmath,amssymb,mathrsfs,dsfont}
\usepackage[utf8]{inputenc}
\usepackage[ngerman]{babel}
\usepackage[pdfborderstyle={/S/U/W 1}]{hyperref}
\usepackage{color}
\definecolor{dkgreen}{rgb}{0,0.6,0}
\definecolor{gray}{rgb}{0.5,0.5,0.5}
\definecolor{mauve}{rgb}{0.58,0,0.82}
\usepackage{listings}
\lstset{frame=tb,
language=C,
aboveskip=3mm,
belowskip=3mm,
showstringspaces=false,
columns=flexible,
basicstyle={\small\ttfamily},
numbers=none,
numberstyle=\tiny\color{gray},
keywordstyle=\color{blue},
commentstyle=\color{dkgreen},
stringstyle=\color{mauve},
breaklines=true,
breakatwhitespace=true,
tabsize=3
}
\oddsidemargin0mm \evensidemargin3mm \textwidth150mm \textheight23cm
\parindent0mm \pagestyle{empty} \topmargin-1cm
\pagestyle{headings}
%euro-zeichen f. utf8
\usepackage{eurosym}
\DeclareUnicodeCharacter{20AC}{\euro}
\newcommand{\Lsg}{\par \textbf{Lsg.: \hfill }}
\newcommand{\lsg}[1]{\underline{\underline{#1}}}

\begin{document}

\begin{enumerate}

\item %1.
Die Zahlen $+0 , -0 , +\infty , -\infty,$ not a number$, 64.625, 75.4, -2$ sind als 32-Bit Gleitpunktzahlen zu schreiben.

\begin{enumerate}
    \item $+0$
    \item $-0$
    \item $+\infty 1.0·2^128$
    \item $-\infty $
	\item NaN
    \item $64.625$
    \item $75.4$
    \item $-2$
\end{enumerate}

\item %2.
Die folgenden 32-Bit Gleitpunktzahlen sind dual, hexadzimal und angenähert dezimal anzugeben:

\begin{itemize}
\item die größte Gleitpunktzahl
\item die kleinste Gleitpunktzahl
\item die kleinste positive Gleitpunktzahl größer als Null
\item die größte negative Gleitpunktzahl
\end{itemize}

\item %3.
Die folgenden 64-Bit Gleitpunktzahlen sind dual, hexadzimal und angenähert dezimal anzugeben:

\begin{itemize}
\item die größte Gleitpunktzahl
\item die kleinste Gleitpunktzahl
\item die kleinste positive Gleitpunktzahl größer als Null
\item die größte negative Gleitpunktzahl
\end{itemize}

\item\label{umwandlungsalgo} %4.
Die Umwandlung einer ganzen positiven Dezimalzahl in eine Zahl mit Basis b ($2 < = b <= 36$) soll als Algorithmus in Form eines Programmablaufplanes formuliert werden. Im Algorithmus sollen die Dezimalzahl d und die Basis b eingelesen und auf $d>=0$ und $2<=b<=36$ getestet werden. Die Speicherung der Ziffern der umgewandelten Zahl soll in einem Vektor in Form von ASCII-Zeichen erfolgen. Der Vektor soll ausgegeben werden.

\item\label{umwandlungsprogramm} %5.
Der Algorithmus der \ref{umwandlungsalgo}. Aufgabe ist als C-Programm zu implementieren. Die Umrechnung von d in eine Zahl zur Basis b soll über eine Funktion mit $3$ Parametern \texttt{(unsigned long d, unsigned long b, char *z)} und dem Funktionstyp int erfolgen. Im Falle einer unzulässigen Basis $(b<2 oder b>36)$ wird eine $1$ zurückgegeben, sonst 0. Das Ergebnis der Umwandlung wird als Folge von ASCII-Zeichen in z eingetragen. Eine Funktion \texttt{reverse(char *)} kehrt den Inhalt von $z$ (vor der Ausgabe) um.

\item\label{algo2} %6.
Die Umwandlung einer gebrochenen positiven Dezimalzahl der Form $0.f..f$ in eine Zahl mit Basis $b (2 <= b <= 36)$ soll als Algorithmus formuliert werden. Im Algorithmus sollen der Bruch d, die Basisb und die Anzahl der Nachkommastellen im Zielsystem eingelesen und auf $d >= 0$ und $2 <= b <= 36$ getestet werden. 
Die Speicherung der Ziffern der umgewandelten Zahl soll in einem Vektor in Form von ASCII-Zeichen erfolgen. Der Vektor soll ausgegeben werden.

\item %7.
Der Algorithmus der \ref{algo2}. Aufgabe soll analog zur \ref{umwandlungsprogramm}. Aufgabe als C-Programm implementiert werden. Beispiel für den Dialog:

\begin{tabular}{rr}
\multicolumn{2}{c}{Umwandlung eines Dezimalbruchs in beliebiges System} \\
\hline
\multicolumn{2}{c}{Umzuwandelnder Dezimalbruch aus [0,1] ? 0.125}	\\
\multicolumn{2}{c}{Basis des Zielsystems aus [2,36] ? 16}	\\
\multicolumn{2}{c}{Anzahl der Nachkommastellen im Zielsystem ? 10}	\\
(0.125)10 = (0.2)16 &	\\
0.125 * 16 = &	2 Ueberlauf 2	\\
0 * 16 = &	0 Ueberlauf 0	
\end{tabular}

\end{enumerate}

\end{document}
