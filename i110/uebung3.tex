1. Die Zahlen +0 , -0 , +\infty , -\infty, not a number, 64.625, 75.4, -2 sind als 32-Bit Gleitpunktzahlen zu schreiben.s

    +0
    -0
    +\infty 1.0·2^128
    -\infty 
    NAN
    64.625
    75.4
    -2
    

2. Die folgenden 32-Bit Gleitpunktzahlen sind dual, hexadzimal und angenähert dezimal anzugeben:

- die größte Gleitpunktzahl
- die kleinste Gleitpunktzahl
- die kleinste positive Gleitpunktzahl größer als Null
- die größte negative Gleitpunktzahl

3. Die folgenden 64-Bit Gleitpunktzahlen sind dual, hexadzimal und angenähert dezimal anzugeben:

- die größte Gleitpunktzahl
- die kleinste Gleitpunktzahl
- die kleinste positive Gleitpunktzahl größer als Null
- die größte negative Gleitpunktzahl

4. Die Umwandlung einer ganzen positiven Dezimalzahl in eine Zahl mit Basis b ( 2 < = b <= 36) soll a ls Algor ithmus in For m e ine s Programmablaufplanes formuliert werden. Im Algor ithmus sollen die Dezimalzahl d und die Basis b eingelesen und auf d>=0 und 2<=b<=36
getestet werden. Die Speicherung der Ziffern der umgewandelten Zahl soll in einem Vektor in Form von ASCII-Zeichen erfolgen. Der Vektor soll ausgegeben werden.

5. Der Algorithmus der 4. Aufgabe ist als C-Programm zu implementieren. Die Umrechnung von d in eine Zahl zur Basis b soll über eine Funktion mit 3 Parametern (unsigned long d, unsigned long b, char *z) und dem Funktionstyp int erfolgen. Im Falle einer unzulässigen Basis (b<2 oder b>36) wird eine 1 zurückgegeben, sonst 0. Das Ergebnis der Umwandlung wird als Folge von ASCII-Zeichen in z eingetragen. Eine Funktion
reverse(char *) kehrt den Inhalt von z (vor der Ausgabe) um.

6. Die Umwandlung einer gebrochenen p o s i t i v e n D e z i m a l z a h l d e r F o r m
0.f..f in eine Zahl mitBasis b ( 2 <= b <= 36) soll als Algorithmus
formuliert werden. Im Algorithmus sollen der Bruch d , die Basisb und
die Anzahl der Nachkommastellen im Zielsystem eingelesen und auf
d >= 0 und 2 <= b <= 36 getestet werden. Die Speicherung der Ziffern
der umgewandelten Zahl sol l i n e i n e m V e k t o r in Form von ASCII-Zeichen
erfolgen. Der Vektor soll ausgegeben werden.

7. Der Algorithmus der 6. Aufgabe so ll analog zur 5.Aufgabe als C-Pro-
gramm implementiert werden. Beispiel für den Dialog:
U m w a n d l u ng e i n e s D e z i m a l b ru c h s i n b e l i e b i g e s S ys t e m
== == == == == == == == == == == == === == == == == == == == == == == == ==
Umzuwandelnder Dezimalbruch aus [0,1] ? 0.125
B as is d es Z ie ls y ste m s a u s [ 2 , 36 ] ? 1 6
Anzahl der Nachkommastellen im Zielsystem ? 10
(0.125)10 = (0.2)16
0.125 * 16 =
2 Ueberlauf 2
0 * 16 =
0 Ue berl auf 0
