\documentclass[12pt,a4paper]{scrreprt}

\usepackage{amsmath,amssymb,mathrsfs,dsfont}
\usepackage[utf8]{inputenc}
\usepackage[ngerman]{babel}
\oddsidemargin0mm \evensidemargin3mm \textwidth150mm \textheight23cm
\parindent0mm \pagestyle{empty} \topmargin-1cm
\newcommand{\E}{\mathds{E}}
\newcommand{\V}{\mathds{V}}
\newcommand{\C}{\mathds{C}}
\newcommand{\N}{\mathds{N}}
\newcommand{\Z}{\mathds{Z}}
\newcommand{\R}{\mathds{R}}
\pagestyle{headings}
\begin{document} 
\begin{flushleft}
Prof. Dr. Anja Voß-Böhme \\
Dipl.-Math. Thomas Buder
\end{flushleft}

\begin{center}
\large{\textbf{ Wirtschaftsmathematik I}} \\
WS 2014/15 \end{center}

\begin{center}\large{\textbf{ Übung 3 }} \end{center}

\bigskip
\begin{enumerate}
 			
\item Prüfen Sie, ob es sich bei folgenden Relationen um eine Äquivalenz- oder eine Ordnungsrelation handelt.

		\begin{enumerate}
			\item $x$ ist ein Teiler von $y$ auf der Grundmenge $\N$.
			\item $ R\subseteq \Z^2$ mit der Definition $x R y:\Leftrightarrow x+y$ ist gerade.
		\end{enumerate}
		
		
\item Gegeben sei die Funktion $f: \R \to \R, f(x)=2x-3.$ \\
			Bestimmen und skizzieren Sie folgende Funktionen.
 \begin{enumerate}
			\item $g_1(x)=f(x)-1$
			\item $g_2(x)=f(x-1)$
			\item $g_3(x)=f(-x)$
			\item $g_4(x)=\frac{1}{f(x)}$
 \end{enumerate}

 \item Gegeben seien folgende Abbildungen: \\
\[f_1: \N \to \R,   f(x)=\frac{x}{x+1}\]
\[f_2: \R \to \R,   f(x)=2x-1\]
\[f_3: \N \to \R,   f(x)=2x-1\]
\[f_4: \R \to \R,   f(x)=x^2\]
\[f_5: \N \to \R,   f(x)=x^2\]
\[f_6: \R \to \R^2, f(x)=(x,x) \]
\[f_7: \R^2 \to \R, f(x,y) = x+y \] 

\begin{enumerate}
			\item Bestimmen Sie die Bilder \[f_1(\{1,2,3,4\}), f_2([-1,1]), f_4([0,1]),f_7(3,3)\]
			\item Bestimmen Sie die Urbilder \[f_1^{-1}((3/4,1)), f_4^{-1}([-1,1]),f_6^{-1}(3,3)\]
			\item Untersuchen Sie die gegebenen Funktionen auf Surjektivität, Injektivität und Bijektivität.
			\item Untersuchen Sie, ob die Kompositionen $f_4\circ f_2,  f_5\circ f_3,  f_3 \circ f_5$ definiert sind. Falls ja, geben Sie die Funktionen explizit an und untersuchen Sie sie auf Surjektivität, Injektivität oder Bijektivität.
			\end{enumerate}
\item Skizzieren Sie $graph(f)$ für folgenden Funktionen, geben Sie  - falls vorhanden - die Umkehrfunktion $y=f^{-1}(x)$ an,  deren Wertebereich und Definitionsbereich und skizzieren Sie $f^{-1}(x)$.
		\begin{enumerate}
			\item $f(x)=x^3$ für $x \in \R$
			\item $f(x)= \begin{cases}
													(x+2)^2+1 \text{ für  } x \in [-2,1] \\
													2x+8 \text{ für  } x \in (1,3]
										\end{cases}$
			\end{enumerate}
 \end{enumerate}

\end{document}
