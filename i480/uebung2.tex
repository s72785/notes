\documentclass[12pt,a4paper]{scrreprt}

\usepackage{amsmath,amssymb,mathrsfs,dsfont}
\usepackage[latin1]{inputenc}
\usepackage[ngerman]{babel}
\oddsidemargin0mm \evensidemargin3mm \textwidth150mm \textheight23cm
\parindent0mm \pagestyle{empty} \topmargin-1cm
\newcommand{\E}{\mathds{E}}
\newcommand{\V}{\mathds{V}}
\newcommand{\C}{\mathds{C}}
\newcommand{\N}{\mathds{N}}
\newcommand{\Z}{\mathds{Z}}
\newcommand{\R}{\mathds{R}}
\pagestyle{headings}
\begin{document} 
\begin{flushleft}
Prof. Dr.  Anja Voß-Böhme \\
Dipl.-Math. Thomas Buder
\end{flushleft}

\begin{center}{\large\bf Wirtschaftsmathematik I} \\ \\ WS 2014/15 \end{center}

\begin{center}{\large\bf Übung 2 } 
\end{center}


\bigskip
\begin{enumerate}
 			
		 \item $A$, $B$ und $C$ seien beliebige Mengen. Untersuchen Sie die folgenden 							Gleichungen und begründen Sie mittels Venn-Diagrammen, welche der 									Beziehungen wahr und welche falsch sind.
		 				\begin{enumerate}
		 				\item $(A \backslash B) \cup (B \backslash A) = (A \cup B) \backslash (A                    \cap B)$
						\item $A \cup (B \backslash C) = (A \cup B) \backslash (C \backslash A)$
					\end{enumerate}
								\item Für zwei Mengen $A$ und $B$ ist die symmetrische Differenz     
								durch
						\[A \Delta B := (A \backslash B) \cup (B \backslash A)\]
						erklärt. Veranschaulichen Sie sich diese Definition mittels 
						Venn-Diagramm. 
						Gibt es Mengen $M$ und $N$, so dass $A\Delta M = A$ 
						beziehungsweise $A \Delta N = \emptyset$?		
					
	\item Es sei	\[A_n := \{k \in \N | k \leq n\}.\]
		Bestimmen Sie $\bigcap\limits_{i=1}^n A_i$ und  $\bigcup\limits_{i=1}^n A_i$ sowie $\bigcap\limits_{i \in \N} A_i$ und  $\bigcup\limits_{i \in \N} A_i.$
\item Skizzieren Sie in der $x,y$-Ebene das kartesische Produkt $A \times B$, wobei \\ $A=\{1;2;3\}$ und $B=[2, 4] \cup \{5\}.$
\item 
\begin{enumerate}
\item
Sei $R$ die Verwandtschaftsrelation auf der Menge aller Menschen, d.h. $a R b$ genau dann, wenn der Mensch
$a$ mit dem Mensch $b$ verwandt ist. \\
Begründen Sie informell, dass $R$ reflexiv, symmetrisch und transitiv ist. 
\item
Ein Spezialfall der Verwandtschaftsrelation ist die Vorfahrenrelation $S$, d.h. $a S b$ genau dann, wenn
$a$ Vorfahre von $b$ ist. \\
Untersuchen Sie ob, $R$ und $S$ in einer Teilmengenbeziehung stehen.
\item Ist $S$ reflexiv, symmetrisch oder
transitiv? 
\item Welche Eigenschaften hat die Relation
$R \backslash S$?
\end{enumerate}
\item 
 Finden Sie eine Relation auf
					$\{1,2,3\}$, die reflexiv und transitiv ist,
					aber nicht symmetrisch.
\item Die Relation $R$ sei definiert durch
\[R=\{(a,b) | a, b \in \R, a\cdot b= 1 \}.\]
Ist $R$ reflexiv auf $\R$, transitiv, symmetrisch bzw. antisymmetrisch? Geben Sie jeweils eine kurze Begründung oder ein Gegenbeispiel.

\item Prüfen Sie, ob es sich bei folgenden Relationen um eine Äquivalenz- oder eine Ordnungsrelation handelt.

		\begin{enumerate}
			\item $x$ ist ein Teiler von $y$ auf der Grundmenge $\N$.
			\item $ R\subseteq \Z^2$ mit der Definition $x R y:\Leftrightarrow x+y$ ist gerade.
		\end{enumerate}
    \end{enumerate}

\end{document}
