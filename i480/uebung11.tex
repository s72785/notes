\documentclass[12pt,a4paper]{scrreprt}
\usepackage{tikz}
\usetikzlibrary{positioning,calc}
\usepackage{amsmath,amssymb,mathrsfs,dsfont}
\usepackage[utf8]{inputenc}
\usepackage[ngerman]{babel}
\usepackage[colorlinks=false, pdfborderstyle={/S/U/W 1}
]{hyperref}
\usepackage{stmaryrd}
\usepackage{comment}
\oddsidemargin0mm \evensidemargin3mm \textwidth150mm \textheight23cm
\parindent0mm \pagestyle{empty} \topmargin-1cm
\newcommand{\E}{\mathds{E}}
\newcommand{\V}{\mathds{V}}
\newcommand{\C}{\mathds{C}}
\newcommand{\N}{\mathds{N}}
\newcommand{\Z}{\mathds{Z}}
\newcommand{\R}{\mathds{R}}
\pagestyle{headings}


\newcommand{\Lsg}{\textbf{Lsg.:}}

\begin{document} 
\begin{flushleft}
\href{mailto:anja.voss-boehme@htw-dresden.de}{Prof. Dr. Anja Voß-Böhme} \\
\href{mailto:buder@htw-dresden.de}{Dipl.-Math. Thomas Buder}
\end{flushleft}

\begin{center}{\large\textbf Wirtschaftsmathematik I} \\ WS 2014/15 \end{center}

\begin{center}{\large\textbf Übung 11 } 
\end{center}


\bigskip

%## Rentenrechnung

R_n = R \frac{q^n-1}{q-1}, q=1+i

allg.

K_n = k_0 q^n - R \frac{q^n-1}{q-1}

Spezialfall: K_n = 0

q^nK_0 = R \frac{q^n-1}{q-1}

Tilgung:

S_n = S q^n - A \frac{q^n-1}{q-1}

\begin{enumerate}
%1.
\item

Herr A ab 01.01.2012 pro Jahr 1200 € einzahlen bis 01.01.2022 (10y).
i = 6\%

ges.: Endbetrag 2022 = R_10

1.1.12 +1200
1.1.13 +1200 + 6\%*1200
1.1.14 +1200 + 6\%*(2400 + 6\%*1200)

R_10 = 1200 \frac{1,06^10-1}{0,06} = 15816,95
Z = R_10 - 10*1200 = 3816,95

%2.
\item

Frau B. ab 1.1. jährlich am Ende des Jahres
100 EUR bis zum 18. Geb. (18y) bei Zins v. 5\%

R_18 = 100 \frac{1,05^18-1}{0,05} = 2813,24

R_18 = 10000 = R \frac{1,05^18-1}{0,05}
R = 10000 frac{0,05}{1,05^18-1} = 355,46

%3.
\item

Herr C. am 01.01.2015 300.000,00 , i=8\%, setzt sich zur Ruhe

\begin{enumerate}

%a
\item jährlich 36.000 Entnahme

Restbetrag am 1.1.2024

K_9 = 300.000 1,08^9 - 36.000 \frac{1,08^9 - 1}{0,08} = 150.149,31

%b
\item

20 Jahre lang konstanten Betrag abheben -> welchen?

R = 300.000 1,08^20 \frac{0,08}{1,08^20 - 1} = 30.555,65

%c
\item

K_10 = 300.000 1,08^10 - 30.555,65 \frac{1,08^10 - 1}{0,08} % = 205.031,17
R = K_10 1,04^10 \frac{0,04}{1,04^10 - 1} = 25.278,50

\end{enumerate}

%4.
\item

Kredit über 10.000, i=10\%, 

\begin{enumerate}
%a
\item in 10y in Ratentilgung

Tilgung T = \frac{S}{n} = 10.000 / 10 = 1.000

Tilgungsplan mit Ratentiltung
\begin{tabular}{rrrrr}
Jahr	& Schuld	& Rate	& Tilgung	& Zinsteil \\
1.		& 10.000 & 2.000	& 1.000	& 1.000 \\
2.		& 9.000	& 1.900	& 1.000	& 900 \\
3.		& 8.000	& 1.800	& 1.000	& 800 \\
4.		& 7.000	& 1.700	& 1.000	& 700 \\
5.		& 6.000	& 1.600	& 1.000	& 600 \\
6.		& 5.000	& 1.500	& 1.000	& 500 \\
7.		& 4.000	& 1.400	& 1.000	& 400 \\
8.		& 3.000	& 1.300	& 1.000	& 300 \\
9.		& 2.000	& 1.200	& 1.000	& 200 \\
10.		& 1.000	& 1.100	& 1.000	& 100 
\end{tabular}

%b
\item in 5y mit Annuitätentilgung

konstante Annuität = Zins + Tilgung

S_5 = 10.000 q^5 - A \frac{1,1^5-1}{0,1}, S_5=0
A = 10.000 q^5 \frac{0,1}{1,1^5-1} = 2637,97

Tilgungsplan
\begin{tabular}{rrrrr}
Jahr	& Schuld	& Rate	& Tilgung	& Zinsteil \\
1.		& 10.000	& 2.637,97	& 1.637,97	& 1.000 \\
2.		& 8.362,03	& 2.637,97	& 1.811,77	& 836,20 \\
3.		& 6.550,26	& 2.637,97	& 1.982,94	& 655,03 \\
4.		& 4.578,30	& 2.637,97	& 2.180,14	& 457,83 \\
5.		& 2.398,16	& 2.398,16	& 2.398,16	& 239,82
\end{tabular}

\end{enumerate}

%5.
\item

Auto für 10.000 verkaufen
monatl. Raten 300, i=0,5\% p.m.

\begin{enumerate}
%a
\item

300 = 10.000 \frac{1,005^n(1,005-1)}{1,005^n-1}
\frac{300}{10.000} (1,005^n-1) = 1,005^n(1,005-1)
6 * 1,005^n - 6 = 1,005^n
(6-1) * 1,005^n  = 6
{1,005^n}{ln 1,005^n}=\frac{6}{5}
n = frac{ln(\frac{6}{5})}{ln(1,005)} = 36,56 => Laufzeit 37 Monate

%b
\item in 24 Monaten

300 = S \frac{1,005^24(0,005)}{1,005^24-1}
S = 300 \frac{1,005^24-1}{1,005^24(0,005)} = 6.768,86
\Rightarrow Sonderzahlung = 3.231,14

%c
\item 2.200 sofort, 24 Raten a 300

= S + SZ = 6768,86 + 2.200 = 8.968,86

%d
\item Wie große wäre die Restschuld von Familie F. nach 4 Jahren / Monaten, wenn sich beide Seiten auf Ratentilgung mit einer Laufzeit von 20 Monaten einigen?

Ratentilgung T_{n+1}=T_{n} = 10.000/20
Restschuld = (20-4)*500 = 8.000


\end{enumerate}


\end{enumerate}

\end{document}
