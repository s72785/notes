\documentclass[12pt,a4paper]{scrreprt}
\usepackage{tikz}
\usetikzlibrary{positioning,calc}
\usepackage{amsmath,amssymb,mathrsfs,dsfont}
\usepackage[utf8]{inputenc}
\usepackage[ngerman]{babel}
\oddsidemargin0mm \evensidemargin3mm \textwidth150mm \textheight23cm
\parindent0mm \pagestyle{empty} \topmargin-1cm
\newcommand{\E}{\mathds{E}}
\newcommand{\V}{\mathds{V}}
\newcommand{\C}{\mathds{C}}
\newcommand{\N}{\mathds{N}}
\newcommand{\Z}{\mathds{Z}}
\newcommand{\R}{\mathds{R}}
\pagestyle{headings}
\begin{document} 
\begin{flushleft}
Prof. Dr.  Anja Voß-Böhme \\
Dipl.-Math. Thomas Buder
\end{flushleft}

\begin{center}{\large\bf Wirtschaftsmathematik I} \\ WS 2014/15 \end{center}

\begin{center}{\large\bf Übung 5 } 
\end{center}


\bigskip
\begin{enumerate}
\item Gegeben seien folgende Matrizen. 
\[A=\begin{pmatrix}
2 & 4 & 4\\
3 & -5 & 4\\

\end{pmatrix}, \quad B=\begin{pmatrix}
2 & 4 & 4\\
3 & -5 & 4\\
3 & 4 & 4\\
\end{pmatrix}, \quad C=\begin{pmatrix}
1 & 3 & 4\\
1 & 2 & 4\\
0 & 4 & 4\\
\end{pmatrix},\quad D=\begin{pmatrix}
0 & 3 \\
1 & 2 \\
0 & 4 \\
\end{pmatrix}\]
Berechnen Sie, falls dies möglich ist. \\

 $A + B$, \quad $A^T$, \quad $B - C$, \quad $AB$, \quad $BA$, \quad $AC^T$, \quad $5A$, \quad $2A+3D^T$



\item Die Firmen $A,B$ und $C$ beliefern einander und einen Endverbraucher $E$ entsprechend des folgenden
Gozinto-Graphen (Angaben in entsprechenden Mengeneinheiten). \tikzset{
  adim/.style={rectangle,minimum width=50,minimum height= 20,draw,thick},
  nn/.style={rectangle,minimum width=210, minimum height=20,draw,thick},
}

\begin{tikzpicture}
  \node (UR) [adim] at (5,4)  {B};
  \node (UL) [adim] at (-1,4) {A};
  \node (C)  [adim] at (2,2)  {C};
  \node (B)  [nn]   at (2,0)  {E};
\draw[thick,->] (UL.north)  to[out=90, in=180, looseness=6]node[above]{10} (UL.west);
\draw[thick,->] (UR.north) to[out=90, in=0, looseness=6] node[above]{36} (UR.east);
\draw[thick,->] ($(UR.west)+(0,0.2)$) -- ($(UL.east)+(0,0.2)$) node[midway,above]{39};
\draw[thick,->] ($(UL.east)+(0,-0.2)$) -- ($(UR.west)+(0,-0.2)$) node[midway,below]{36};
\draw[thick,->] (UR) -- ($(B.north)+(3,0)$) node[midway,right]{15};
\draw[thick,->] (UL) -- ($(B.north)+(-3,0)$) node[midway,left]{14};
\draw[thick,->] (C) -- ($(B.north)+(0,0)$) node[midway,left]{48};
\draw[thick,<-] (UR.south west) -- (C.north east) node[midway,above]{10};
\draw[thick,->] (UR.south) -- (C.south east) node[midway,below]{10};
\draw[thick,<-] (UL.south east) -- (C.north west) node[midway,above]{12};
\draw[thick,->] (UL.south) -- (C.south west) node[midway,left]{40};
\draw[thick,->] ($(C.north east)+(-0.2,0)$) to[out=100,in=80] node[above]{30} ($(C.north west)+(0.2,0)$);
\end{tikzpicture} 
		\begin{enumerate}
			\item Stellen Sie die Informationen des Gozinto-   Graphen in einer Tabelle dar.
			\item Stellen Sie den Marktvektor, Produktionsvektor und die Verbrauchsmatrix auf.
			\item Berechnen Sie die Input-Output-Matrix und interpretieren Sie deren Einträge.
			\item Stellen Sie ein Gleichungssystem auf, aus dem sich die nötige Gesamtproduktion der drei Firmen berechnen lässt, wenn sich die Nachfrage des Produkts von Firma $A$ auf 30, Firma $B$ auf 50 und Firma $C$ auf 60 Einheiten ändert.
		
		\end{enumerate}
\item Gegeben sei der Marktvektor $\bf y$$=\begin{pmatrix}
2 \\
3 \\
5 \\
\end{pmatrix}$, sowie die Verbrauchsmatrix $\bf X$$=\begin{pmatrix}
2 & 5 & 7 \\
3 & 3 & 4\\
5 & 2 & 5\\
\end{pmatrix}$.  Berechnen Sie den Produktionsvektor und erstellen sie den zugehörigen Gozinto-Graphen.
\end{enumerate}
\end{document}


