\documentclass[12pt,a4paper]{scrreprt}
\usepackage{tikz}
\usetikzlibrary{positioning,calc}
\usepackage{amsmath,amssymb,mathrsfs,dsfont}
\usepackage[utf8]{inputenc}
\usepackage[ngerman]{babel}
\usepackage[colorlinks=false, pdfborderstyle={/S/U/W 1}
]{hyperref}
\usepackage{stmaryrd}
\usepackage{comment}
\oddsidemargin0mm \evensidemargin3mm \textwidth150mm \textheight23cm
\parindent0mm \pagestyle{empty} \topmargin-1cm
\newcommand{\E}{\mathds{E}}
\newcommand{\V}{\mathds{V}}
\newcommand{\C}{\mathds{C}}
\newcommand{\N}{\mathds{N}}
\newcommand{\Z}{\mathds{Z}}
\newcommand{\R}{\mathds{R}}
\pagestyle{headings}


\newcommand{\Lsg}{\textbf{Lsg.:}}

\begin{document} 
\begin{flushleft}
\href{mailto:anja.voss-boehme@htw-dresden.de}{Prof. Dr. Anja Voß-Böhme} \\
\href{mailto:buder@htw-dresden.de}{Dipl.-Math. Thomas Buder}
\end{flushleft}

\begin{center}{\large\textbf Wirtschaftsmathematik I} \\ WS 2014/15 \end{center}

\begin{center}{\large\textbf Übung 10 } 
\end{center}


\bigskip

\begin{enumerate}
%1.
\item

Bestimmen Sie Real- und Imaginärteil von $z = e^{−1 + \frac{\pi}{2} i}$


%2.
\item

%3.
\item

%4.
\item

%5.
\item

%6.
\item

\end{enumerate}

\end{document} 
