\documentclass[12pt,a4paper]{scrreprt}
\usepackage{tikz}
\usetikzlibrary{positioning,calc}
\usepackage{amsmath,amssymb,mathrsfs,dsfont}
\usepackage[utf8]{inputenc}
\usepackage[ngerman]{babel}
\usepackage[colorlinks=false, pdfborderstyle={/S/U/W 1}]{hyperref}
\usepackage{stmaryrd}
\usepackage{comment}
\oddsidemargin0mm \evensidemargin3mm \textwidth150mm \textheight23cm
\parindent0mm \pagestyle{empty} \topmargin-1cm
\newcommand{\E}{\mathds{E}}
\newcommand{\V}{\mathds{V}}
\newcommand{\C}{\mathds{C}}
\newcommand{\N}{\mathds{N}}
\newcommand{\Z}{\mathds{Z}}
\newcommand{\R}{\mathds{R}}
%\renewcommand{\i}{\text{i}}
\pagestyle{headings}


\newcommand{\Lsg}{\textbf{Lsg.:}}
\usepackage{tikz}
\usetikzlibrary{arrows,calc}
\tikzset{
%Define standard arrow tip
>=stealth',
%Define style for different line styles
help lines/.style={dashed, thick},
axis/.style={<->},
important line/.style={thick},
connection/.style={thick, dotted},
}

\begin{document} 
\begin{flushleft}
\href{mailto:anja.voss-boehme@htw-dresden.de}{Prof. Dr. Anja Voß-Böhme} \\
\href{mailto:buder@htw-dresden.de}{Dipl.-Math. Thomas Buder}
\end{flushleft}

\begin{center}{\large\textbf Wirtschaftsmathematik I} \\ WS 2014/15 \end{center}

\begin{center}{\large\textbf{Übung 9} } 
\end{center}


\bigskip

\begin{enumerate}
	%1.
	\item Gegeben sind die komplexen Zahlen
	$ z_1 = 1- i $,
	$z_2 = -5 i $,
	$z_3 = 3e^{ \pi i }$ und
	$z_4 = 2e^{ \frac{\pi}{6} i }$

\begin{tabular}{lll}
$ z=e^{i \phi}$ &$= |z| e^{i arg(z)}$ & Eulersche Form \\
$ z= a + i b$ &$= Re(z) + Im(z)  i $ & Kartesische / Arithmetische Form \\
$ z=r(cos \phi + i sin \phi) $ & & Trigonometrische Form
\end{tabular}

$Re(z):  a= $

$Im(z):  i b= $

%skizze todo
%------
%achsen x,i
%punkt
%winkel
%-------


	\begin{enumerate}
	
		%a
		\item Geben Sie die Eulersche Darstellung von $z_1$ und $z_2$, sowie die kartesische Darstellung von $z_3$ und $z_4$ an.


\[
\]

		%b
		\item Skizzieren Sie $z_1$ bis $z_4$ in der komplexen Zahlenebene.


%skizze
%---------
\begin{tikzpicture}[scale=1]
    % Axis
    \coordinate (y1) at (0,5);
    \coordinate (y0) at (0,-2);
    \coordinate (x1) at (5,0);
    \coordinate (x0) at (-2,0);
    \draw[->] (y0) -- (y1) node[above] {$\mathit{Im}$};
    \draw[->] (x0) -- (x1) node[right] {$\mathit{Re}$};
    % A grid can be useful when defining coordinates
    % \draw[step=1mm, gray, thin] (0,0) grid (5,5); 
    % \draw[step=5mm, black] (0,0) grid (5,5); 

	

    \end{tikzpicture}
%-----------

\[
\]

		%c
		\item Berechnen Sie:

			\begin{enumerate}
	
\item $z_1 \cdot \overline{z_1}$,

\[
\]

\item $z_1 + z_3$,

\[
\]

\item $\frac{z_1}{z_4}$,

\[
\]

\item $z1 \cdot z2$,

\[
\]

\item $(z_{3})^{4}$

\[
\]

			\end{enumerate}

		%d
		\item Bestimmen Sie $|z_1|$ und $|z_4|$.


\[
\]

	\end{enumerate}

	%2.
	\item Skizzieren Sie in der komplexen Zahlenebene die Menge der Punkte, für die gilt:

	\begin{enumerate}

		%a
		\item $|z - 2 + i| \leq 5$


		%b
		\item $|z + 3| = 2$


		%c
		\item $|arg z| < \frac{\pi}{2}$


	\end{enumerate}

	%3.
	\item Geben Sie die Lösungen der folgenden Gleichungen für $z \in \C$ an und skizzieren Sie diese in der komplexen Zahlenebene.

	\begin{enumerate}

		%a
		\item $z^2=-1$

\[
z^2=1 \cdot e^{\pi i}
\]

		%b
		\item $z^3=8i$

\[
\]

		%c
		\item $z^5=-1$

\[
\]

	\end{enumerate}

	%4.
	\item Welche komplexe Zahl ist das Spiegelbild von $z = a + b i (a, b \in \R)$ bei Spiegelung

	\begin{enumerate}

		%a
		\item am Ursprung,

		%b
		\item an der reellen Achse,


		%c
		\item an der imaginären Achse,


		%d
		\item an der Winkelhalbierenden des I. und III. Quadranten,


		%e
		\item an der Winkelhalbierenden des II. und IV. Quadranten?


	\end{enumerate}

	%5.
	\item Bestimmen Sie Real- und Imaginärteil von $z = e^{-	1+ \frac{\pi}{2} i}$

	%6.
	\item 

	\begin{enumerate}

		%a
		\item Für welche komplexen Zahlen $z$ gilt $z = \overline{z}$ ?



		%b
		\item Welche $z \in \C$ erfüllen sowohl $|z| = \sqrt{8}$ als auch $z + \overline{z} = 4$ ?

\begin{displaymath}
|z| = \sqrt{8} \Rightarrow z = \sqrt[4]{8}
z + \overline{z} = 4
= 
\end{displaymath}

	\end{enumerate}

\end{enumerate}
\end{document}
