\documentclass[12pt,a4paper]{scrreprt}
\usepackage{tikz}
\usetikzlibrary{positioning,calc}
\usepackage{amsmath,amssymb,mathrsfs,dsfont}
\usepackage[utf8]{inputenc}
\usepackage[ngerman]{babel}
\usepackage{color}
\oddsidemargin0mm \evensidemargin3mm \textwidth150mm \textheight23cm
\parindent0mm \pagestyle{empty} \topmargin-1cm
\newcommand{\E}{\mathds{E}}
\newcommand{\V}{\mathds{V}}
\newcommand{\C}{\mathds{C}}
\newcommand{\N}{\mathds{N}}
\newcommand{\Z}{\mathds{Z}}
\newcommand{\R}{\mathds{R}}
\pagestyle{headings}
\begin{document} 
\begin{flushleft}
Prof. Dr.  Anja Voß-Böhme \\
Dipl.-Math. Thomas Buder
\end{flushleft}

\begin{center}
\large{\textbf{ Wirtschaftsmathematik I}} \\
WS 2014/15 \end{center}

\begin{center}\large{\textbf{ Übung 6 }} \end{center}

\bigskip
\begin{enumerate}
	\item	Gegeben seien die Vektoren %1.
		$a=\begin{pmatrix}
			3 \\
			2\\
			0
		\end{pmatrix}$,
        $b=\begin{pmatrix}
			-2 \\
			4\\
			0
		\end{pmatrix}$ und
        $c=\begin{pmatrix}
			1 \\
			0\\
			-3
		\end{pmatrix}.$
	\begin{enumerate}
		%a)
		\item Berechnen Sie die Länge der Vektoren $a$, $a+b$ sowie $a^0:=\frac{a}{|a|}$.
        
        %\begin{mbox}
        \begin{equation}
        	|a| = \sqrt{3^2+2^2+0^2} = \sqrt{9+4} = \sqrt{13} \\
    	    |a+b| = sqrt{(a_1+b_1)^2+(a_2+b_2)^2+(a_3+b_3)^2} =  \\
	        a^0=\frac{a}{|a|}=1 \\
        \end{equation}
        %\end{mbox}}
        
        %b)
	\item Berechnen Sie die Skalarprodukte $a^Tb$ und  $b^Ta$.

%----------------------------
        \begin{equation}
a \circ b = 3*-2+2*4 = -6+8 = 2 \\

        \end{equation}
%----------------------------

	%c)
	\item Welchen Winkel schließen die Vektoren $a$ und  $b$
 ein?
 	%d)
	\item Berechnen Sie die Länge des Vektors $d=a+2b-c.$
	%e)
	\item Bestimmen Sie einen Vektor $f$, der dieselbe Länge wie $a$ besitzt, jedoch parallel zu $b$ verläuft.
	%f)
	\item Stellen Sie den Vektor $a$ dar als die Summe zweier Vektoren $g$ und $h$, so dass $g$ parallel zu $b$ verläuft und $h$ senkrecht zu $b$ verläuft. \\
(Hinweis: Man bezeichnet $g$ dann als Projektion des Vektors $a$ in Richtung des Vektors $b$.)

\end{enumerate}

\item Für  $x \in \R$ seien die Vektoren 

	$a=\begin{pmatrix}
		1 \\
		2x\\
		3
	\end{pmatrix}$ und
	$b=\begin{pmatrix}
		1 \\
		x\\
		x
	\end{pmatrix}$

	gegeben.
	\begin{enumerate}
		%a)
		\item  Für welche $x \in \R$ stehen die Vektoren $\vec{a}$ und $\vec{b}$ senkrecht aufeinander?
		%b)
		\item Bestimmen Sie $x$ so, dass die Vektoren $a$ und $b$ linear abhängig sind.
		%c)
		\item Für welche $x$ stehen die Vektoren $a$ und $b$ senkrecht zum Vektor \\
		$c=\begin{pmatrix}
			1 \\
			-2\\
			-1
		\end{pmatrix}?$
	\end{enumerate}

\item Gegeben seien die Vektoren
	$a=\begin{pmatrix}
		2 \\
		4 \\
		0
	\end{pmatrix}$,
	$b=\begin{pmatrix}
		0 \\
		-3\\
		0
	\end{pmatrix}$ und
	$c=\begin{pmatrix}
		1 \\
		0 \\
		-2
	\end{pmatrix}$.

	\begin{enumerate}
		\item Untersuchen Sie, ob die Vektoren $a$, $b$ und $c$ linear unabhängig sind.
		\item Bilden die Vektoren $a$, $b$ und $c$ eine Basis von $\R^3$?
		\item Stellen Sie den Vektor $a$, wenn möglich, als Linearkombination der Vektoren $b$ und $c$ dar.
	\end{enumerate}

\item Berechnen Sie die Determinanten folgender Matrizen!
	\begin{enumerate}
		\item $A=\begin{pmatrix}
			-2 & 3  \\
			4 & 5\\
		\end{pmatrix}$

%-------------
$det A=|A|=-2*5-4*3=-10-12 = -22$
%-------------

		\item  $B=\begin{pmatrix}
			1 & 1   \\
			-2 & -2\\
		\end{pmatrix}$

%-------------
$det B	=	|\begin{pmatrix}
			1 & 1   \\
			-2 & -2\\
		\end{pmatrix}|
	= 0 \text{, weil die Spaltenvektoren gleich sind}$
%-------------

		\item $C=\begin{pmatrix}
			3 	& 1 	& 0  \\
			-1 	& 2 	& 4 \\
			4 	& 1 	& 5
		\end{pmatrix}$

%-------------
$
det C	= 3*2*5 + 4*4 + 0 - ( 0 + 3*4 + -1*5) = 30+16-(12-5) = 46-7=33
$
%-------------

	\end{enumerate}
	\item Für welche Zahl $x \in \R$ gilt
	\[
    	\det \begin{pmatrix}
			-x & 2 & -1 \\
			-1 & 3x & x \\
			1 & -2 & 0
		\end{pmatrix} =0?
	\]

%----------------
	\item Sind die Vektoren $\begin{pmatrix}
		8 \\ -1 \\ 4
	\end{pmatrix}, \begin{pmatrix}
		1 \\ 2 \\ 3
	\end{pmatrix}, \begin{pmatrix}
		2 \\ 4 \\ 5
	\end{pmatrix}, \begin{pmatrix}
		1 \\ 0 \\ 4
	\end{pmatrix}$ 
	linear unabhängig?
	
	Erweitern um eine Zeile (Nullen) und über die Determinante bestimmen.

	\item Sind die Vektoren $\begin{pmatrix}
		4 \\ -2 \\ 1
	\end{pmatrix}, \begin{pmatrix}
		0 \\ 0 \\ 1
	\end{pmatrix}, \begin{pmatrix}
		0 \\ 1 \\ 0
	\end{pmatrix}, \begin{pmatrix}
		1 \\ 0 \\ 0 
	\end{pmatrix}$ 
	linear abhängig?
	
	Ja, weil die letzten Drei den \R^3 aufspannen und der erste dort darstellbar ist.
%----------------
\end{enumerate}
\end{document}
