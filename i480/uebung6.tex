\documentclass[12pt,a4paper]{scrreprt}
\usepackage{tikz}
\usetikzlibrary{positioning,calc}
\usepackage{amsmath,amssymb,mathrsfs,dsfont}
\usepackage[utf8]{inputenc}
\usepackage[ngerman]{babel}
\usepackage{color}
\oddsidemargin0mm \evensidemargin3mm \textwidth150mm \textheight23cm
\parindent0mm \pagestyle{empty} \topmargin-1cm
\newcommand{\E}{\mathds{E}}
\newcommand{\V}{\mathds{V}}
\newcommand{\C}{\mathds{C}}
\newcommand{\N}{\mathds{N}}
\newcommand{\Z}{\mathds{Z}}
\newcommand{\R}{\mathds{R}}
\pagestyle{headings}
\begin{document} 
\begin{flushleft}
Prof. Dr.  Anja Voß-Böhme \\
Dipl.-Math. Thomas Buder
\end{flushleft}

\begin{center}
{ \large\bf Wirtschaftsmathematik I} \\
WS 2014/15 \end{center}

\begin{center}{\large\bf Übung 6 } 
\end{center}


\bigskip
\begin{enumerate}
	\item	Gegeben seien die Vektoren
		$a=\begin{pmatrix}
			3 \\
			2\\
			0
		\end{pmatrix}$,
        $b=\begin{pmatrix}
			-2 \\
			4\\
			0
		\end{pmatrix}$ und
        $c=\begin{pmatrix}
			1 \\
			0\\
			-3
		\end{pmatrix}.$
	\begin{enumerate}
		\item Berechnen Sie die Länge der Vektoren $a$, $a+b$ sowie $a^0:=\frac{a}{|a|}$.
        
        %\begin{mbox}
        \begin{equation}
        	|a| = \sqrt{3^2+2^2+0^2} = \sqrt{9+4} = \sqrt{13} \\
    	    |a+b| = sqrt{(a_1+b_1)^2+(a_2+b_2)^2+(a_3+b_3)^2} =  \\
	        a^0=\frac{a}{|a|}=1 \\
        \end{equation}
        %\end{mbox}}
        
\item Berechnen Sie die Skalarprodukte $a^Tb$ und  $b^Ta$.

%----------------------------
        \begin{equation}
a \circ b = 3*-2+2*4 = -6+8 = 2 \\

        \end{equation}
%----------------------------

\item Welchen Winkel schließen die Vektoren $a$ und  $b$
 ein?
\item Berechnen Sie die Länge des Vektors $d=a+2b-c.$
\item Bestimmen Sie einen Vektor $f$, der dieselbe Länge wie $a$ besitzt, jedoch parallel zu $b$ verläuft.
\item Stellen Sie den Vektor $a$ dar als die Summe zweier Vektoren $g$ und $h$, so dass $g$ parallel zu $b$ verläuft und $h$ senkrecht zu $b$ verläuft. \\
(Hinweis: Man bezeichnet $g$ dann als Projektion des Vektors $a$ in Richtung des Vektors $b$.)

\end{enumerate}
\item
 Für  $x \in \R$ seien die Vektoren
$a=\begin{pmatrix}
1 \\
2x\\
3

\end{pmatrix}$ und $b=\begin{pmatrix}
1 \\
x\\
x

\end{pmatrix}$
gegeben.
\begin{enumerate}
\item  Für welche $x \in \R$ stehen die Vektoren
$a$ und $b$
senkrecht aufeinander?
\item Bestimmen Sie $x$ so, dass die Vektoren $a$ und $b$ linear abhängig sind.
\item Für welche $x$ stehen die Vektoren $a$ und $b$ senkrecht zum Vektor \\$c=\begin{pmatrix}
1 \\
-2\\
-1

\end{pmatrix}?$
\end{enumerate}
\item

 Gegeben seien die Vektoren
$a=\begin{pmatrix}
2 \\
4\\
0

\end{pmatrix}$,  $b=\begin{pmatrix}
0 \\
-3\\
0

\end{pmatrix}$ und $c=\begin{pmatrix}
1 \\
0\\
-2

\end{pmatrix}.$
\begin{enumerate}
	\item Untersuchen Sie, ob die Vektoren $a$, $b$ und $c$ linear unabhängig sind.
	\item Bilden die Vektoren $a$, $b$ und $c$ eine Basis von $\R^3$?
	\item Stellen Sie den Vektor $a$, wenn möglich, als Linearkombination der Vektoren $b$ und $c$ dar.
\end{enumerate}
	\item Berechnen Sie die Determinanten folgender Matrizen!
	\begin{enumerate}
		\item $A=\begin{pmatrix}
			-2 & 3  \\
			4 & 5\\
		\end{pmatrix}$
		\item  $B=\begin{pmatrix}
			1 & 1   \\
			-2 & -2\\
		\end{pmatrix}$
		\item $C=\begin{pmatrix}
			3 & 1 & 0  \\
			-1 & 2 & 4 \\
			4 & 1 & 5
		\end{pmatrix}$
	\end{enumerate}
	\item Für welche Zahl $x \in \R$ gilt
	\[
    	\det \begin{pmatrix}
			-x & 2 & -1 \\
			-1 & 3x & x \\
			1 & -2 & 0
		\end{pmatrix} =0?
    \]
\end{enumerate}
\end{document}

