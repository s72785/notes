\documentclass[12pt,a4paper]{scrreprt}
\usepackage{tikz}
\usetikzlibrary{positioning,calc}
\usepackage{amsmath,amssymb,mathrsfs,dsfont}
\usepackage[utf8]{inputenc}
\usepackage[ngerman]{babel}
\usepackage[colorlinks=false, pdfborderstyle={/S/U/W 1}
]{hyperref}
\usepackage{stmaryrd}
\usepackage{comment}
\oddsidemargin0mm \evensidemargin3mm \textwidth150mm \textheight23cm
\parindent0mm \pagestyle{empty} \topmargin-1cm
\newcommand{\E}{\mathds{E}}
\newcommand{\V}{\mathds{V}}
\newcommand{\C}{\mathds{C}}
\newcommand{\N}{\mathds{N}}
\newcommand{\Z}{\mathds{Z}}
\newcommand{\R}{\mathds{R}}
\pagestyle{headings}


\newcommand{\Lsg}{\textbf{Lsg.:}}

\begin{document} 
\begin{flushleft}
\href{mailto:anja.voss-boehme@htw-dresden.de}{Prof. Dr. Anja Voß-Böhme} \\
\href{mailto:buder@htw-dresden.de}{Dipl.-Math. Thomas Buder}
\end{flushleft}

\begin{center}{\large\textbf Wirtschaftsmathematik I} \\ WS 2014/15 \end{center}

\begin{center}{\large\textbf Übung 12 } 
\end{center}


\bigskip

\begin{enumerate}

%1
\item Geben Sie die ersten 6 Glieder der nachstehenden Zahlenfolgen ${a_n }_{n=0,1,2,\ldots} $
an.

\begin{enumerate}

%1a
\item $a_n = 1 + \frac{(-1)^n}{n+1}$

\[
a_0 = 2
a_1 = 1/2
a_2 = 4/3
a_3 = 3/4
a_4 = 6/5
a_5 = 5/6
\]

\item $a_1=2, a_2=3, a_n=3a_{n-1}-2a_{n-2}$ für $n>=2$.

\[
a_3 = 3*a_2-2*a_1 = 3*3 - 2*2 = 5
a_4 = 3*a3 - 2*a2 = 3*5 - 2*3 = 9
a_5 = 
a_6 = 
a_7 = 
\]

\end{enumerate}

\item  \begin{enumerate}

Geometr. Folge: Hier ist der Nachfolger ein Vielfaches des Vorgängers.

\[
{a_{n+1}}{a_n} = c, c konst. f. alle n \in \N
\]

\item Wie lautet das 8. Glied einer geometrischen Folge mit dem Anfangsglied
$a_1 = 5/4$ und dem Quotienten $q = 2$?

\[
a_2 = 2*a_1
a_3 = 2*2*a_2
a_n = 2^{n-1}*a_1
a_8 = 2^7*\frac{5}{4}=2^7*\frac{5}{4}=160
\]

\item Zeigen Sie, dass die Folge ${a_n}_{n=1,2,\ldots}$ mit $a_n = 2n − 7$ eine arithmetische
Folge ist.

a_{n+1}-a_{n} =c 

\end{enumerate}

%3.
\item Untersuchen Sie die folgenden Zahlenfolgen ${a_n}_{n=0,1,2,...}$ auf Beschränktheit
und Konvergenz. Geben Sie gegebenenfalls den Grenzwert an.

\begin{enumerate}

\item g_o = 100^3 -> konvertgent
\item 1/3^n * ( \ldots ) = "0 * \infty" -> n.d.
ausmultipliziert: 2^n/3^n+(-2)^n/3^n = (2/3)^n+(-2/3)^n \Rightarrown -> 0
\item divergent, kein Grenzwert; g_o=2, g_u=0
\item n^3/n^4 * 3-0/4-0 = 1/n * 3/4 = 0 * 3/4 -> 0
\item 1 * 2/3 = 2/3
\item n * 1 = n -> \infty \Rightarrow divergent, unbeschränkt

\end{enumerate}

\item Geben Sie für die Reihe
\sum_{n=1}^{\infty} \frac{n!}{n+2} 
die ersten 4 Glieder der Folge {s_n}_{n=0,1,2,\ldots} der
Partialsummen an.

\[
s_1 = 1/1+2 = 1/3
s_2 = 3/2+2 = 5/6
s_3 = 5/6 + 6/5 = 61/30
s_4 = 61/30 + 24/6 = 181/30
\]

\item Uberprüfen Sie die folgenden Reihen auf Konvergenz.

\begin{enumerate}

\item 
\item 
\item 
\item 

\end{enumerate}

\end{enumerate}



\end{document}
