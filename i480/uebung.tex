Reihen

\Sigma_{k=0}^{\infty} a_k
s_n = \Sigma_k=0^n a_k

Konergenzkriterien

Trivialkriterium (not wendige Bedingung)

Damit \Sigma a_k konvergiert muss a_k \rightarrow 0 .

! a_k \rightarrow 0 \Rightarrow \Sigma a_k konvertiert

Leibnitzkriterium

Das Vorzeichen muss wechslen

\Sigma_k=0^\infty (-1)^k a_k 

kovergiert, falls

\item Trivialkriterium
\item a_k monoton fallend

Quotientenkriterium

\Soigma a_k konvergiert, falls

1. Trivialkriterium
2. lim_{k \rightarrow 0} |\frac{a_{k+1}}{a_k}|
< 1 konvergent
= 1 nicht hilfreich
> 1 divergent

Wurzelkriterium

lim_{k \rightarrow 0} \sqrt[k]{|a_k|}
< 1 konvergent
= 1 nicht hilfreich
> 1 divergent

\begin{enumerate}
%1.
\item 
\begin{enumerate}
\item \Sigma_{n=0}^{\infty} \frac{n^6}{2^n}

%lim anfügen
QK: lim_{n \rightarow \infty} \frac{a_{n+1}}{a_n} = \frac{ frac{(n+1)^6}{2^{n+1}} }{ \frac{n^6}{2^n} } = frac{ (n+1)^6 2^n }{ n^{n+1} n^6 } =  frac{ (n+1)^6 }{ n^{1} n^6 } = frac{(n+1)^6}{n} \frac{1}{2} = (1 + \frac{1}{n})^6 \frac{1}{2} = 1^6 \frac{1}{2} = 1/2 < 1 < \infty \Rightarrow konvergent

Grenzwert nicht bestimmtbar.

\item \Sigma_{n=0}^{\infty} {(-1)^n}{3n+1}

TK: ist erfüllt

%lim anfügen
LK: \Sigma_{n \rightarrow 0}^{\infty} (-1)^n \frac{1}{\underbrace{3n+1}_{a_n}} , a_n \rightarrow 0 \Rightarrow erfüllt, a_n monoton fallend, d.h. a_{n+1} <= a_n = 1 <= 4 \Righarrow erfüllt

\Rightarrow konvergent

\item \Sigma_{n=0}^{\infty} \sqrt{n+9}

WK: nur gut wenn a_n leicht als Wurzel zu ziehen

TK: lim_{n \rightarrow \infty} \sqrt{n+9} \lightning = \infty

\Rightarrow divergent

\item \Sigma_{n=1}^{\infty} (\frac{n+1}{9n})^n

TK: lim_{n \rightarrow 0} (\frac{n+1}{9n})^n = (\frac{1}{9})^n = 0 \Rightarrow erfüllt

WK: lim_{n \rightarrow \oinfty} \sqrt[n]{(\frac{n+1}{9n})^n} = \frac{n+1}{9n} < 1

\Rightarrow konvergent

\item \Sigma_{n=0}^{\infty} \frac{n!}{5^n}

TK: lim_{n \rightarrow 0} \frac{n!}{5^n} = \infty > 0 

o. QK: = \frac{(n+1)!}{5 n!} = \frac{n+1}{5} > 1

\Rightarrow divergent

\item \Sigma_{n=1}^{\infty} \frac{|sin n|}{n^2}

TK: lim \frac{|sin n|}{n^2} \rightarrow 0 \Rightarrow erfüllt

QK: \frac{ \frac{|sin (n+1)|}{(n+1)^2} }{ \frac{|sin n|}{n^2} } ?!

VK:

Vergleich mit bekannter Reihe, z.B. \frac{1}{n}, divergent \Rightarrow QT=1, versagt

Vergleich mit \frac{1}{n^a}, konvergent f. a>1, divegrent für a<=1

\Rightarrow \frac{|sin n|}{n^2} = \frac{1}{n^2} |sin n| konvergent, da a>1 und |sin n| < 1

\item \Sigma_{n=0}^{\infty} \frac{2+(-1)^n}{n^2+7}

VK: wir vermuten Konvergenz (Differenz der größten Potenz = 2)

Eine Vergleichsform finden, die größer wird

\frac{2+(-1)^n}{n^2+7} <= \frac{3}{ n^2 }

\frac{2+(-1)^n}{n^2+7} <= ( \frac{3}{ n^2 } = 3 \frac{1}{ n^2 } ) < \infty

\Rightarrow konvergent

\item \Sigma_{n=0}^{\infty} \frac{2n^2 +1}{n^3-n^2-1}

VK: wir vermuten Divergenz (Differenz der größten Potenz = 1) und machen die Zahlen kleiner

\frac{2n^2 +1}{n^3-n^2-1} >= \frac{2n^2}{n^3-n^2}  >= ( \frac{2n^2}{n^3} = 2/n = 2*1/n \rightarrow \infty)

\Rightarrow divergent

\end{enumerate}

%2.
\item 

\begin{enumerate}

\item \Sima_{n=0}^{\infty} (-1)^{n+1} 2^{-n}

geometr. Reihe \Sigma_{n=0}^{\infty} q^n  = \frac{1}{1-q}, |q|<1 mit Grenzwert 2

 \Sima_{n=0}^{\infty} (-1)^{n+1} 2^{-n}
 =  \Sima_{n=0}^{\infty} \frac{(-1)^{n}(-1)}{ 2^{n} }
 =  (-1) \Sima_{n=0}^{\infty} (\frac{-1}{ 2 })^{n} = -\frac{2}{3}
 
 \item \Sigma{n=0}{\infty} \frac{5+2^n}{10^n}
 
=  \Sigma{n=0}{\infty} (\frac{5}{10^n} + \frac{2^n}{10^n})
=  5 \Sigma{n=0}{\infty} \frac{1}{10^n} +  \Sigma{n=0}{\infty} (\frac{2}{10})^n
=  5 \Sigma{n=0}{\infty} \frac{1^n}{10^n} +  \Sigma{n=0}{\infty} (\frac{2}{10})^n
=  5 \Sigma{n=0}{\infty} (\frac{1}{10})^n +  \Sigma{n=0}{\infty} (\frac{2}{10})^n
=  5 \frac{1}{1-\frac{1}{10}} + \frac{1}{1-\frac{1}{5}} = \frac{5}{4}

\item \Sigma_{n=2}^{\infty}
= \Sigma_{n=0}^{\infty} - \underbrace{(\frac{1}{3})^0}_{a_n, n=0} - \underbrace{(\frac{1}{3})^1}_{a_n, n=0}
= \frac{1}{1-\frac{1}{3}} - 1 - \frac{1}{3}

\item \Sigma_{n=0}^{\infty} (\frac{1}{n+1} - \frac{1}{n+2})
= \frac{1}{1}- \frac{1}{2} +  \frac{1}{2}- \frac{1}{3} \ldots = 1 
\end{enumerate}


\end{enumerate}
