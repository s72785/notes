\documentclass[12pt,a4paper]{scrreprt}
\usepackage{tikz}
\usetikzlibrary{positioning,calc}
\usepackage{amsmath,amssymb,mathrsfs,dsfont}
\usepackage[latin1]{inputenc}
\usepackage[ngerman]{babel}
\oddsidemargin0mm \evensidemargin3mm \textwidth150mm \textheight23cm
\parindent0mm \pagestyle{empty} \topmargin-1cm
\newcommand{\E}{\mathds{E}}
\newcommand{\V}{\mathds{V}}
\newcommand{\C}{\mathds{C}}
\newcommand{\N}{\mathds{N}}
\newcommand{\Z}{\mathds{Z}}
\newcommand{\R}{\mathds{R}}
\pagestyle{headings}
\begin{document} 
\begin{flushleft}
Prof. Dr.  Anja Vo�-B�hme \\
Dipl.-Math. Thomas Buder
\end{flushleft}

\begin{center}{\large\bf Wirtschaftsmathematik I} \\ \\ WS 2014/15 \end{center}

\begin{center}{\large\bf �bung 8 } 
\end{center}


\bigskip
\begin{enumerate}
\item Invertieren Sie die Matrix  $A=\begin{pmatrix}
 2 & 7 & 3\\
 3 & 9 & 4 \\
 1 & 5 & 3\\
\end{pmatrix}$ mit Hilfe des Gau�-Algorithmus.
\item L�sen Sie das folgende Gleichungssystem.
		\begin{align*}
x - y +  z &= 1 \\
7x - 4y - z &= -2\\
-x + 2y + 5z &= 2\\
5y + 17z &= 18
\end{align}
\item F�r welche reellen Zahlen $a,b$ hat das lineare Gleichungssystem

		\begin{align*}
					 x-2y+3z &= -4 \\
					2x +y +z &=  2 \\
					x + ay + 2z &= -b			
		 \end{align}

\begin{enumerate}
\item genau eine L�sung,
\item keine L�sung,
\item unendlich viele L�sungen ?

\end{enumerate}
Geben Sie die L�sungen in den F�llen a) und c) an.

\item 
Zeigen Sie, dass die Vektoren $a=\begin{pmatrix}
2 \\
3\\
1

\end{pmatrix}$,  $b=\begin{pmatrix}
7 \\
9\\
5

\end{pmatrix}$ und $c=\begin{pmatrix}
3 \\
4\\
3

\end{pmatrix}$ eine Basis von $\R^3$ bilden. \\
 Wie l�sst sich der Vektor $d=\begin{pmatrix}
1 \\
-2\\
-2

\end{pmatrix}$ mit dieser Basis darstellen? 

\item Bestimmen Sie den Rang der folgenden Matrizen.
	\begin{enumerate}
\item 
$A=\begin{pmatrix}
 1 & 4 & 7\\
 2 & 5 & 8 \\
 3 & 6 & 9\\
 4 & 8 & 12
\end{pmatrix}$
\item $B=\begin{pmatrix}
 1 & 1 & 1 & 1 & 1 & 1 & 1\\
 1 & 1 & 1 & 1 & 1 & 1 & 1 \\
 1 & 1 & 1 & 1 & 1 & 1 & 1\\
 1 & 1 & 1 & 1 & 1 & 1 & 1 \\
 1 & 1 & 1 & 1 & 1 & 1 & x
\end{pmatrix}$ f�r $x \in \R.$
\end{enumerate}
\item Seien $A=\begin{pmatrix}
 1 & 1 & 0 \\
 0 & 1 & 3 \\
 c & 0 & 1 \\

\end{pmatrix}$ und $B=\begin{pmatrix}
 1 & 2 & 0 \\
 0 & -c & 2c \\
-c & 0 & 2 \\
  

\end{pmatrix}.$ 
	\begin{enumerate}
\item F�r welche $c \in \R$ haben die Matrizen $A$ und $B$ denselben Rang?
\item F�r welche $c \in \R$ ist die Matrix $A$ invertierbar? Bestimmen Sie f�r diesen Fall den Rang der Matrix $A^{-1}$.
\item F�r welche $c \in \R$ gilt $\det B=0$?

\end{enumerate}
\end{document}


