\documentclass[12pt,a4paper]{scrreprt}
\usepackage{tikz}
\usetikzlibrary{positioning,calc}
\usepackage{amsmath,amssymb,mathrsfs,dsfont}
\usepackage[utf8]{inputenc}
\usepackage[ngerman]{babel}
\usepackage[colorlinks=false, pdfborderstyle={/S/U/W 1}
]{hyperref}
\usepackage{comment}
\oddsidemargin0mm \evensidemargin3mm \textwidth150mm \textheight23cm
\parindent0mm \pagestyle{empty} \topmargin-1cm
\newcommand{\E}{\mathds{E}}
\newcommand{\V}{\mathds{V}}
\newcommand{\C}{\mathds{C}}
\newcommand{\N}{\mathds{N}}
\newcommand{\Z}{\mathds{Z}}
\newcommand{\R}{\mathds{R}}
\pagestyle{headings}


\newcommand{\Lsg}{\textbf{Lsg.:}}

\begin{document} 
\begin{flushleft}
\href{mailto:anja.voss-boehme@htw-dresden.de}{Prof. Dr. Anja Voß-Böhme} \\
\href{mailto:buder@htw-dresden.de}{Dipl.-Math. Thomas Buder}
\end{flushleft}

\begin{center}{\large\textbf Wirtschaftsmathematik I} \\ WS 2014/15 \end{center}

\begin{center}{\large\bf Übung 8 } 
\end{center}


\bigskip
\begin{enumerate}
	%1.
	\item Invertieren Sie die Matrix
$A=\begin{pmatrix}
 2 & 7 & 3\\
 3 & 9 & 4 \\
 1 & 5 & 3\\
\end{pmatrix}$ mit Hilfe des Gauß-Algorithmus.

%\Lsg
$
\begin{pmatrix}
 2 & 7 & 3 \vert 1 & 0 & 0 \\
 3 & 9 & 4 \vert 0 & 1 & 0 \\
 1 & 5 & 3 \vert 0 & 0 & 1 \\
\end{pmatrix}\cdot\begin{pmatrix} 3 \\ 2 \\ 6 \end{pmatrix}
$
%2.-1. u. 3-1. Zeile
%2. Zeile *3 und zur 3. addieren
%3. Zeile durch 6 und zur zweiten hinzu
%3. Zeile mal -9 und zur ersten Zeile

	%2.
	\item Lösen Sie das folgende Gleichungssystem.\begin{align*}
x - y +  z &= 1 \\
7x - 4y - z &= -2\\
-x + 2y + 5z &= 2\\
5y + 17z &= 18
\end{align*}

%\Lsg
$
\begin{pmatrix}
1	&	-1	&	1	|	1	\\
7	&	-4	&	-1	|	-2	\\
-1	&	2	&	5	|	2	\\
0	&	5	&	17	|	18
\end{pmatrix}
%ziel ist dreiecksform
%1. zur 3. Zeile
%1- mal (-7) zur 2. Zeile
%2. und 3. Zeile tauschen
%2. mal (-5) zur 4. Zeile
%3. durch (-2) zur 4.
=
\begin{pmatrix}
1	&	-1	&	1	&	1	\\
0	&	1	&	6	&	3	\\
0	&	0	&	-26	&	-18	\\
0	&	0	&	0	&	12
\end{pmatrix}
$ ist aufgrund der letzten Zeile nicht lösbar.

	%3.
	\item Für welche reellen Zahlen $a,b$ hat das lineare Gleichungssystem	\begin{align*}
					 x-2y+3z &= -4 \\
					2x +y +z &=  2 \\
					x + ay + 2z &= -b			
		 \end{align*}

%\Lsg
$
\begin{pmatrix}
x	&	y	&	z	&		\\
1	&	-2	&	3	&	-4	\\
2	&	1	&	1	&	2	\\
1	&	a	&	2	&	-b
\end{pmatrix}
%wieder Dreiecksform ermitteln
%1. Zeile mal -2 zur 2.
%1. Zeile mal -1 zur 3.
%3. und 2. Spalte tauschen (namen merken)
%2. Zeile durch -5 und zur 3. hinzu
=\begin{pmatrix}
x	&	z	&	y	&		\\
1	&	3	&	-2	&	-4	\\
0	&	-3	&	5	&	10	\\
0	&	0	&	a+1	&	-b+2
\end{pmatrix}
$

	\begin{enumerate}
		%a
		\item genau eine Lösung,

		$a\neq -1, b egal$
		
		für $a=0, b=0$: $y=2, z=0, x=0$
		
		%b
		\item keine Lösung,
		
		sodass min. eine Zeile einen Wiederspruch ergibt
		$a=-1, b\neq 2$
		
		%c
		\item unendlich viele Lösungen ?
		
		sodass eine Zeile komplett Null ist: 
		$a=-1, b=2 $ soweit keine weiteren Gleichungen zum LGS zählen
		
		$\Rightarrow  y=t, z=2-t, x=-10+5t$ ergibt unendliche viele Lösungen für $t \in \R$
	\end{enumerate}
Geben Sie die Lösungen in den Fällen a) und c) an.
	
	%4
	\item
Zeigen Sie, dass die Vektoren $a=\begin{pmatrix} 2 \\ 3 \\ 1 \end{pmatrix}$,  $b=\begin{pmatrix} 7 \\ 9 \\ 5 \end{pmatrix}$ und $c=\begin{pmatrix} 3 \\ 4 \\ 3 \end{pmatrix}$ eine Basis von $\R^3$ bilden. \\
 Wie lässt sich der Vektor $d=\begin{pmatrix} 1 \\ -2 \\ -2 \end{pmatrix}$ mit dieser Basis darstellen? 

%1. teil
%lineare unabhängigkeit

%2.teil
%d= c_1 * a+c_2 * b+c_3 * c

	%5.
	\item Bestimmen Sie den Rang der folgenden Matrizen. %heißt: gaußalgo, wieviele Zeilen sind nicht null?
	\begin{enumerate}
		\item $A=\begin{pmatrix}
 1 & 4 & 7 \\
 2 & 5 & 8 \\
 3 & 6 & 9 \\
 4 & 8 & 12 \end{pmatrix}$
		\item $B=\begin{pmatrix}
 1 & 1 & 1 & 1 & 1 & 1 & 1\\
 1 & 1 & 1 & 1 & 1 & 1 & 1 \\
 1 & 1 & 1 & 1 & 1 & 1 & 1\\
 1 & 1 & 1 & 1 & 1 & 1 & 1 \\
 1 & 1 & 1 & 1 & 1 & 1 & x \end{pmatrix}$ für $x \in \R.$
	\end{enumerate}

%\Lsg
$
x=1 \Rightarrow Rang=1
x\neq 1 \Rightarrow Rang=2
$

	%6.
	\item Seien $A=\begin{pmatrix}
 1 & 1 & 0 \\
 0 & 1 & 3 \\
 c & 0 & 1 \\ \end{pmatrix}$
 und $B=\begin{pmatrix}
 1 & 2 & 0 \\
 0 & -c & 2c \\
-c & 0 & 2 \end{pmatrix}.$ 
%Rang für beide berechnen, anschließend Beziehung/Eigenschaften prüfen
%Rang A = 3 f. c \neq -\fraq{1}{3}; 2 f. c=-\fraq{1}{3}
%Rang B = 3 f. c\neq -\fraq{1}{2} u. c\neq 0; 2 f.  c=-\fraq{1}{2} o. c=0
	\begin{enumerate}
		\item Für welche $c \in \R$ haben die Matrizen $A$ und $B$ denselben Rang?
		%3 f. c \neq -\fraq{1}{3}, c\neq -\fraq{1}{2} u. c\neq 0
		\item Für welche $c \in \R$ ist die Matrix $A$ invertierbar? Bestimmen Sie für diesen Fall den Rang der Matrix $A^{-1}$.
		%nur sodass es eine quadratische Matrix bleibt, also Vollrang/Maximaler Rang (hier 3. Rang ) Vorraussetzung ist
		%der Rang A^{-1} entspricht dem Vollrang von A, hier 3
		\item Für welche $c \in \R$ gilt $\det B=0$?
		%det B = 0 heißt, die Matrix kann nicht invertiert werden, also für die -\fraq{1}{2} und 0
	\end{enumerate}

\end{enumerate}
\end{document}
