\documentclass[12pt,a4paper]{scrreprt}

\usepackage{amsmath,amssymb,mathrsfs,dsfont}
\usepackage[utf8]{inputenc}
\usepackage[ngerman]{babel}
\oddsidemargin0mm \evensidemargin3mm \textwidth150mm \textheight23cm
\parindent0mm \pagestyle{empty} \topmargin-1cm
\newcommand{\E}{\mathds{E}}
\newcommand{\V}{\mathds{V}}
\newcommand{\C}{\mathds{C}}
\newcommand{\N}{\mathds{N}}
\pagestyle{headings}
\begin{document} 
\begin{flushleft}
Prof. Dr.  Anja Voß-Böhme \\
Dipl.-Math. Thomas Buder
\end{flushleft}

\begin{center}{\large\bf Lineare Algebra und Höhere Mathematik} \\ WS 2014/15 \end{center}

\begin{center}{\large\bf Übung 1 } 
\end{center}


\bigskip
\begin{enumerate}

 	\item 	Prüfen Sie mit Hilfe einer Wahrheitstafel, ob die folgende Aussagen stets wahr sind.

	\begin{enumerate}
	 	\item   $A \wedge \overline{\overline{A} \Rightarrow B} $
		\item   $(A \Rightarrow B) \Leftrightarrow ( \overline{B} \Rightarrow \overline{A}) $
		\item   $\left( (A \vee B) \wedge ( A \Leftrightarrow \overline{C})\right)  \Rightarrow ( \overline{B \wedge C}) $
	\end{enumerate}

	\item Vereinfachen Sie diese logischen Ausdrücke durch Umformungen so weit wie möglich.
	
	\begin{enumerate}
		\item   $\overline{ \overline{A} \wedge \overline{B}}$
	 	\item   $A \wedge \left( \left(B \wedge \overline{A}) \vee (C \wedge B) \right)$
		\item $(A \wedge B) \vee (\overline{A} \wedge B) \vee (A \wedge \overline{B}) \vee \overline{ A \vee  B}$
	\end{enumerate}
	
 	\item Finden Sie eine Aussage $S$, die folgenden Wahrheitstabellen entspricht. Vereinfachen Sie diese so weit wie möglich.

\begin{center}

	\begin{enumerate}
		\item
			\begin{tabular}{c c | c}
				A & B & S \\
				\hline 
				0	&0	&1 \\
				0	&1	&1 \\
				1	&0	&1 \\
				1	&1	&0 
			\end{tabular}

		\item 
	
			\begin{tabular}{c c c| c}
				A & B & C &S \\
				\hline
				0	&0	&0	&0 \\
				0	&0	&1	&0 \\
				0	&1	&0	&1 \\
				0	&1	&1	&1 \\
				1	&0	&0	&0 \\
				1	& 0	&1	&1 \\
				1	&1	& 0	&0 \\
				1	&1	&1	&1 
			\end{tabular}
	
	\end{enumerate}
	\end{center}

    \item Es wird eine Menge $M$ betrachtet. $A$ und $B$ seien beliebige Teilmengen von $M$.  Gesucht wird die Menge aller Elemente von $M$, die

	\begin{enumerate}
		\item  nicht \textit{nur zu $A$ oder nur zu $B$} gehören,
		\item  \textit{entweder nur zu $A$ oder nur zu $B$} gehören,
		\item nicht \textit{gleichzeitig zu $A$ und $B$} gehören,
		\item zu \textit{$A$ gehören, aber nicht zu $B$},
		\item \textit{nicht zu $A^C$, aber zu $B$} gehören.
	\end{enumerate}
	
	Geben Sie die jeweiligen Mengen an und veranschaulichen Sie diese in einem Venn-Diagramm.

	\item Gegeben sei
	die Menge $M=\{i \in \N | 1 \leq i \leq 15\}$
	sowie zugehörige Teilmengen \\
    		\[
    		A=\{1,3,5,7,9,11,13\}, \quad
    		B=\{6,8,10,12\}, \quad
    		C=\{2,3,5,12,13\}.
    		\] 
	Bestimmen Sie
		\begin{enumerate} 
			\item $A \cup B, \quad A \cap B, \quad  A^C, \quad C^C, \quad C^C \cap B, \quad B^C \cup C$
			\item  $M \backslash B^C, \quad C \backslash A, \quad \left(M \backslash C^C \right) \cap C, \quad B \backslash \left(A \cup C\right)^C.$
		\end{enumerate}
		\item Gegeben sei die Menge $M = \{T, 1, 2, 3 \}$. Geben Sie die Potenzmenge von $M$ an.
		\item $A$, $B$ und $C$ seien beliebige Mengen. Untersuchen Sie die folgenden 							Gleichungen und begründen Sie mittels Venn-Diagrammen, welche der 									Beziehungen wahr und welche falsch sind.
			\begin{enumerate}
				\item $(A \backslash B) \cup (B \backslash A) = (A \cup B) \backslash (A                    \cap B)$
				\item $A \cup (B \backslash C) = (A \cup B) \backslash (C \backslash A)$
			\end{enumerate}
		\item Für zwei Mengen $A$ und $B$ ist die symmetrische Differenz durch
			\[
			A \Delta B := (A \backslash B) \cup (B \backslash A)
			\]
			erklärt. Veranschaulichen Sie sich diese Definition mittels Venn-Diagramm. 
			Gibt es Mengen $M$ und $N$, so dass $A\Delta M = A$ 
			beziehungsweise $A \Delta N = \emptyset$?		

    \end{enumerate}

\end{document}
