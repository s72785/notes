\documentclass[12pt,a4paper]{scrreprt}
\usepackage{tikz}
\usetikzlibrary{positioning,calc}
\usepackage{amsmath,amssymb,mathrsfs,dsfont}
\usepackage[utf8]{inputenc}
\usepackage[ngerman]{babel}
\oddsidemargin0mm \evensidemargin3mm \textwidth150mm \textheight23cm
\parindent0mm \pagestyle{empty} \topmargin-1cm
\newcommand{\E}{\mathds{E}}
\newcommand{\V}{\mathds{V}}
\newcommand{\C}{\mathds{C}}
\newcommand{\N}{\mathds{N}}
\newcommand{\Z}{\mathds{Z}}
\newcommand{\R}{\mathds{R}}
\pagestyle{headings}
\begin{document} 
\begin{flushleft}
Prof. Dr.  Anja Voß-Böhme \\
Dipl.-Math. Thomas Buder
\end{flushleft}

\begin{center}{\large\textbf Wirtschaftsmathematik I} \\ WS 2014/15 \end{center}

\begin{center}
\large{\textbf{ Übung 7 }} 
\end{center}


\bigskip
\begin{enumerate}
\item Berechnen Sie -möglichst geschickt- die Determinanten folgender Matrizen!
\begin{enumerate}
\item $A=\begin{pmatrix}
0 & 0 & 2 & 5 \\
0 & 7 & 1 & -4  \\
-1 & 3 & 2 & 4\\
0 & 0 & 0 & 4 \\
\end{pmatrix}$
\item $B=\begin{pmatrix}
0 & 0 & -1 & 0 \\
0 & 7 & 3 & 0  \\
2 & 1 & 2 & 0\\
5 & -4 & 4 & 4 \\
\end{pmatrix}$
\item $C=\begin{pmatrix}
0 & 0 & -3 & 0 \\
0 & 21 & 9 & 0  \\
6 & 3 & 6 & 0\\
15 & -12 & 12 & 12 \\
\end{pmatrix}$
\end{enumerate}
\item Für welche $a,b \in \R$ ist $A=\begin{pmatrix}
a & 2 & 3 \\
a & 3 & 4\\
a & 3 & 5
\end{pmatrix}$ die Inverse der Matrix \\ $B=\begin{pmatrix}
3 & -1 & -1 \\
-1 & b+1 &-1\\
0 & -1 & b
\end{pmatrix}$ ?
\item 
 Gegeben sei die Matrix
$A=\begin{pmatrix}
-2 & 3 & 3 \\
0 & 1 & 1\\
0 & -1 & -3
\end{pmatrix}.$
\begin{enumerate}
\item Prüfen Sie, ob die inverse Matrix $A^{-1}$ existiert.
\item Berechnen Sie die inverse Matrix $A^{-1}$ mit Hilfe von Unterdeterminanten.
\item Führen Sie eine Probe durch, ob die Inverse korrekt ist.
\end{enumerate}
\item Bestimmen Sie alle  reellwertigen 2$\times$2-Matrizen, die zu sich selbst invers sind.
\item
Für welches $a \in \R$ ist folgendes Gleichungssystem eindeutig lösbar? Lösen Sie für diesen Fall unter Verwendung der \textsc{Cramer'schen} Regel.
\begin{align*}
					 ax+y &= 3 \\
					-2x +y &=-1
\end{align*}
\item 
Lösen Sie folgendes Gleichungssystem  unter Verwendung der \textsc{Cramer'schen} Regel.
\begin{align*}
					 -z &= -y-2x+1 \\
					-x-y+z +y &=	y+2z+1	\\
						x+y &= 2z+2
\end{align*}


%\item Berechnen Sie mit Hilfe des Gauß-Algorithmus, für welche reellen Zahlen $a,b$ das lineare Gleichungssystem
%
		%\begin{align*}
					 %x-2y+3z &= -4 \\
					%2x +y +z &=  2 \\
					%x + ay + 2z &= -b			
		 %\end{align}
%
%\begin{enumerate}
%\item genau eine Lösung hat.
%\item keine Lösung hat.
%\item unendlich viele Lösungen hat.
%
%\end{enumerate}
%Geben Sie die für jeden Fall die Lösungen an.

\end{enumerate}
\end{document}
