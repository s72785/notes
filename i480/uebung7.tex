\documentclass[12pt,a4paper]{scrreprt}
\usepackage{tikz}
\usetikzlibrary{positioning,calc}
\usepackage{amsmath,amssymb,mathrsfs,dsfont}
\usepackage[utf8]{inputenc}
\usepackage[ngerman]{babel}
\usepackage{comment}
\oddsidemargin0mm \evensidemargin3mm \textwidth150mm \textheight23cm
\parindent0mm \pagestyle{empty} \topmargin-1cm
\newcommand{\E}{\mathds{E}}
\newcommand{\V}{\mathds{V}}
\newcommand{\C}{\mathds{C}}
\newcommand{\N}{\mathds{N}}
\newcommand{\Z}{\mathds{Z}}
\newcommand{\R}{\mathds{R}}
\pagestyle{headings}

\begin{document} 
\begin{flushleft}
Prof. Dr.  Anja Voß-Böhme \\
Dipl.-Math. Thomas Buder
\end{flushleft}

\begin{center}{\large\textbf Wirtschaftsmathematik I} \\ WS 2014/15 \end{center}

\begin{center}
\large{\textbf{ Übung 7 }} 
\end{center}

\bigskip
\begin{enumerate}

%1.
	\item Berechnen Sie -möglichst geschickt- die Determinanten folgender Matrizen!
	\begin{enumerate}
%a)
		\item $A=\begin{pmatrix}
0 & 0 & 2 & 5 \\
0 & 7 & 1 & -4  \\
-1 & 3 & 2 & 4\\
0 & 0 & 0 & 4 \\
\end{pmatrix}$

%1. und 3. Zeile tauschen, det(A) berechnen

%b)
		\item $B=\begin{pmatrix}
0 & 0 & -1 & 0 \\
0 & 7 & 3 & 0  \\
2 & 1 & 2 & 0\\
5 & -4 & 4 & 4 \\
\end{pmatrix}$

%Spalte 1 und 3 tauschen...

%c)
		\item $C=\begin{pmatrix}
0 & 0 & -3 & 0 \\
0 & 21 & 9 & 0  \\
6 & 3 & 6 & 0\\
15 & -12 & 12 & 12 \\
\end{pmatrix}$

% C=3B \Rightarrow det(C) = 3^4*det(B)

\begin{comment}
\item $D=\begin{pmatrix}
0 & 0 & -3 & 0 \\
0 & 21 & 9 & 0  \\
6 & 3 & 6 & 0\\
21 & -9 & 18 & 12 \\
\end{pmatrix}$
\end{comment}

% zwei Zeilen (3 und 4) wurden addiert \Rightarrow det(D) = det(C)
	\end{enumerate}

%2.
	\item Für welche $a,b \in \R$ ist $A=\begin{pmatrix}
a & 2 & 3 \\
a & 3 & 4\\
a & 3 & 5
\end{pmatrix}$ die Inverse der Matrix \\ $B=\begin{pmatrix}
3 & -1 & -1 \\
-1 & b+1 &-1\\
0 & -1 & b
\end{pmatrix}$ ?

\begin{comment}
Inverse von $A$ gibt es nur für $det(A) \neq 0$
also hier $A$ mit $B$ multiplizieren und mit der Einheitsmatrix gleichsetzen
3a-2=1 \Rightarrow a=1
-a+3(b+1)-4 \Rightarrow b=1
restliche Werte auch prüfen
\end{comment}

%3.
	\item 
 Gegeben sei die Matrix
$A=\begin{pmatrix}
-2 & 3 & 3 \\
0 & 1 & 1\\
0 & -1 & -3
\end{pmatrix}.$
	\begin{enumerate}
%a)
		\item Prüfen Sie, ob die inverse Matrix $A^{-1}$ existiert.
\begin{comment}
wenn det(A)\neq0, dann existiert A^{-1}
det(A) = (-2)*(-2) = 4 \Rightarrow A^{-1} existiert
\end{comment}

%b)
		\item Berechnen Sie die inverse Matrix $A^{-1}$ mit Hilfe von Unterdeterminanten.
\begin{comment}
(A^{-1})^T = \fraq{1}{det(A)}*A'
A'_{ij} = (-1)^{i+j}D_{ij}
A'_{11} = 1 * | 1 1 \\ -1 -3 | = (-3)-(-1) = -2
A'_{12} = -1 * | 0 1 \\ 0 -3 | = 0
A'_{13} = 1 * | 0 1 \\ 9 -1 | = 0
A'_{21} = -1* | 3 3 \\ -1 -3 | = 6
\vdots
A'_{33} = -2
A^{-1} = \left( \frac{1}{4} \begin{pmatrix} -2 0 0 \\ 6 6 2 \\ 0 -2 -2 \end{pmatrix} \right)^T
= \begin{pmatrix} -2 6 0 \\ 0 6 -2 \\ 0 2 -2 \end{pmatrix} 
\end{comment}

%c)
		\item Führen Sie eine Probe durch, ob die Inverse korrekt ist.
\begin{comment}
A*A^{-1} = I_3
\end{comment}
	\end{enumerate}

%4.
	\item Bestimmen Sie alle  reellwertigen $2 \times 2$-Matrizen, die zu sich selbst invers sind.
\begin{comment}
A=\begin{pmatrix}a b \\ a b\end{pmatrix}
A=A^{-1}
AA^{-1}=I_2 \Rightarrow AA=I-2
\begin{tabular}{rr|rr}
	&	&	a &	b \\
	&	&	c &	d \\
 a	& d	& a^2+bc & ab+bd \\
 c	& d	& ac+cd	& bc+d^2 
\end{tabular}
welche vier Zahlen erfüülen die Einheitsmatrix hier?
\end{comment}

%5.
	\item
Für welches $a \in \R$ ist folgendes Gleichungssystem eindeutig lösbar? Lösen Sie für diesen Fall unter Verwendung der \textsc{Cramer'schen} Regel.
\begin{align*}
					 ax+y &= 3 \\
					-2x +y &=-1
\end{align*}
\begin{comment}
CRAMERsche Regel - nur anwendbar wenn LGS eindeutig lösbar (d.h. det(A) \neq 0 )
\Rightleftarrow A\vec{x}=\vec{b}
mit A=\begin{pmatrix} a 1 \\ -2 1 \end{pmatrix}, \vec{x}=\begin{pmatrix} x \\ y \end{pmatrix}, \vec{b}=\begin{pmatrix} 3 \\ -1 \end{pmatrix}
det(A)=a-2 \Rightarrow lösbar für a \neq -2
x=\frac{det A_1}{det A} = \frac{4}{a-2}
y=\frac{det A_2}{det A} = \frac{-a+6}{a-2}
\end{comment}


%6.
\item 
Lösen Sie folgendes Gleichungssystem  unter Verwendung der \textsc{Cramer'schen} Regel.
	\begin{align*}
		-z			&= -y-2x+1 \\
		-x-y+z +y	&= y+2z+1 \\
		x+y			&= 2z+2
	\end{align*}
\begin{comment}
erst umforme in Standardform
A=\begin{pmatrix}2 1 -1 \\
-1 -1 -1 \\
1 1 -2
\end{pmatrix}, \vec{b} = \begin{pmatrix} 1 \\ 1 \\ 2 \end{pmatrix}, vec{x} = \begin{pmatrix} x \\ y \\ z \end{pmatrix}
det A = -1 \neq 0 \Rightarrow LGS eindeutlich lösbar \Rightarrow CRAMERsche Regel ist anwendbar
x = \frac{\begin{matrix}
1 1 -1 \\
1 -1 -1 \\
2 1 -2
\end{matrix}}{-1} = 0

y = \frac{\begin{matrix}
2  1  -1 \\
-1 1 -1 \\
1  2  -2
\end{matrix}}{-1} = 0

z = \frac{\begin{matrix}
2 1 1 \\
-1 -1 1 \\
1 1 2
\end{matrix}}{-1} = \frac{1}{-1} = -1

\end{comment}


%\item Berechnen Sie mit Hilfe des Gauß-Algorithmus, für welche reellen Zahlen $a,b$ das lineare Gleichungssystem
%
		%\begin{align*}
					 %x-2y+3z &= -4 \\
					%2x +y +z &=  2 \\
					%x + ay + 2z &= -b
		 %\end{align}
%
%\begin{enumerate}
%\item genau eine Lösung hat.
%\item keine Lösung hat.
%\item unendlich viele Lösungen hat.
%
%\end{enumerate}
%Geben Sie die für jeden Fall die Lösungen an.

\end{enumerate}
\end{document}
