\documentclass[12pt,a4paper]{scrreprt}
\usepackage{tikz}
\usetikzlibrary{positioning,calc}
\usepackage{amsmath,amssymb,mathrsfs,dsfont}
\usepackage[utf8]{inputenc}
\usepackage[ngerman]{babel}
\usepackage[colorlinks=false, pdfborderstyle={/S/U/W 1}
]{hyperref}
\usepackage{comment}
\oddsidemargin0mm \evensidemargin3mm \textwidth150mm \textheight23cm
\parindent0mm \pagestyle{empty} \topmargin-1cm
\newcommand{\E}{\mathds{E}}
\newcommand{\V}{\mathds{V}}
\newcommand{\C}{\mathds{C}}
\newcommand{\N}{\mathds{N}}
\newcommand{\Z}{\mathds{Z}}
\newcommand{\R}{\mathds{R}}
\pagestyle{headings}

\newcommand{\Lsg}{\textbf{Lsg.:}}

\begin{document} 
\begin{flushleft}
\href{mailto:anja.voss-boehme@htw-dresden.de}{Prof. Dr. Anja Voß-Böhme} \\
\href{mailto:buder@htw-dresden.de}{Dipl.-Math. Thomas Buder}
\end{flushleft}

\begin{center}{\large\textbf Wirtschaftsmathematik I} \\ WS 2014/15 \end{center}

\begin{center}
\large{\textbf{ Übung 7 }} 
\end{center}

\bigskip
\begin{enumerate}

%1.
	\item Berechnen Sie -möglichst geschickt- die Determinanten folgender Matrizen!
	\begin{enumerate}
%a)
		\item $A=\begin{pmatrix}
0 & 0 & 2 & 5 \\
0 & 7 & 1 & -4  \\
-1 & 3 & 2 & 4\\
0 & 0 & 0 & 4 \\
\end{pmatrix}$

\Lsg

%1. und 3. Zeile tauschen, det(A) berechnen, *(-1) wg. Tauschen
det$A=\begin{pmatrix}
-1 & 3 & 2 & 4\\
0 & 7 & 1 & -4  \\
0 & 0 & 2 & 5 \\
0 & 0 & 0 & 4 \\
\end{pmatrix}
= (-1) \cdot (-56)$


%b)
		\item
$B=\begin{pmatrix}
0 & 0 & -1 & 0 \\
0 & 7 & 3 & 0  \\
2 & 1 & 2 & 0\\
5 & -4 & 4 & 4 \\
\end{pmatrix}$

\Lsg

%Spalte 1 und 3 tauschen...
$B=\begin{pmatrix}
-1 & 0 & 0 & 0 \\
3 & 0 & 7 & 0  \\
2 & 2 & 1 & 0\\
4 & 5 & -4 & 4 \\
\end{pmatrix}$
%oder sehen, dass A^T=B
$B=A^T \Rightarrow det B = det A = det A^T$

%c)
		\item $C=\begin{pmatrix}
0 & 0 & -3 & 0 \\
0 & 21 & 9 & 0  \\
6 & 3 & 6 & 0\\
15 & -12 & 12 & 12 \\
\end{pmatrix}$

\Lsg

% erste und dritte Spalte tauschen
% C=3B \Rightarrow det(C) = 3^4*det(B)

\item $D=\begin{pmatrix}
0 & 0 & -3 & 0 \\
0 & 21 & 9 & 0  \\
6 & 3 & 6 & 0\\
21 & -9 & 18 & 12 \\
\end{pmatrix}
=3^4 \cdot det A = 3^4 \cdot 56 = 4536 $

\Lsg

 zwei Zeilen (3 und 4) wurden addiert $\Rightarrow$ det D = det C
	\end{enumerate}


%2.
	\item Für welche $a,b \in \R$ ist $A=\begin{pmatrix}
a & 2 & 3 \\
a & 3 & 4\\
a & 3 & 5
\end{pmatrix}$ die Inverse der Matrix \\ $B=\begin{pmatrix}
3 & -1 & -1 \\
-1 & b+1 &-1\\
0 & -1 & b
\end{pmatrix}$ ?

\Lsg

Inverse von $A$ gibt es nur für $det(A) \neq 0$
also hier $A$ mit $B$ multiplizieren und mit der Einheitsmatrix $E$ gleichsetzen

%1er Diagonale
$3a-2=1 \Rightarrow a=1$ \\
$-a+3(b+1)-4 \Rightarrow b=1$ \\

Probe: \\
$-a-3+5b \Rightarrow 1$ \\
restliche Werte auch prüfen
%alle anderen müssen 0 sein

%3.
	\item 
 Gegeben sei die Matrix
$A=\begin{pmatrix}
-2 & 3 & 3 \\
0 & 1 & 1\\
0 & -1 & -3
\end{pmatrix}.$
	\begin{enumerate}
%a)
		\item Prüfen Sie, ob die inverse Matrix $A^{-1}$ existiert.
		
\Lsg

wenn $det A \neq 0$, dann existiert $A^{-1}$ \\
Zeile (2 zu 3) addieren um Determinante einfach berechnen zu können\\
$det A = (-2)*(-2) = 4 \Rightarrow A^{-1}$ existiert

%b)
		\item Berechnen Sie die inverse Matrix $A^{-1}$ mit Hilfe von Unterdeterminanten.

\Lsg

$
(A^{-1})^T = \frac{1}{det A}*A'$	\\
$A'_{ij} = (-1)^{i+j}D_{ij}$ 	\\%Determinante, wenn i-te Zeile und j-te Spalte gestrichen
$A'_{11} = 1 * \begin{vmatrix} 1 & 1 \\ -1 & -3 \end{vmatrix} = (-3)-(-1) = -2$	\\
$A'_{12} = -1 * \begin{vmatrix} 0 & 1 \\ 0 & -3 \end{vmatrix} = 0	\\
A'_{13} = 1 * \begin{vmatrix} 0 & 1 \\ 9 & -1 \end{vmatrix} = 0	\\
A'_{21} = -1* \begin{vmatrix} 3 & 3 \\ -1 & -3 \end{vmatrix} = 6	\\
\vdots	\\
A'_{33} = \begin{vmatrix} 2 & 3 \\ 0 & 1 \end{vmatrix} = -2	\\
A^{-1} = \left( \frac{1}{4} \begin{pmatrix} -2 & 0 & 0 \\ 6 & 6 & 2 \\ 0 & -2 & -2 \end{pmatrix} \right)^T	\\
= \begin{pmatrix} -2 & 6 & 0 \\ 0 & 6 & -2 \\ 0 & 2 & -2 \end{pmatrix} 	\\
$

%c)
		\item Führen Sie eine Probe durch, ob die Inverse korrekt ist.

\Lsg

$A*A^{-1} = I_3$

	\end{enumerate}

%4.
	\item Bestimmen Sie alle  reellwertigen $2 \times 2$-Matrizen, die zu sich selbst invers sind.

$
A=\begin{pmatrix}a & b \\ a & b\end{pmatrix} \\
A=A^{-1} \\
AA^{-1}=I_2 \Rightarrow AA=I-2$ \\
\begin{tabular}{rr|ll}
	&	&	a &	b \\
	&	&	c &	d \\
	\hline 
 a	& d	& $a^2+bc$ & $ab+bd$ \\
 c	& d	& $ac+cd$	& $bc+d^2$ 
\end{tabular} $= \begin{pmatrix} 1 & 0 \\ 0 & 1 \end{pmatrix}$ \\
welche vier Zahlen erfüllen die Einheitsmatrix hier?

\begin{align*}
a^2+bc =& 1 \\
ab+bd =& 0 \\
ac+cd =& 0 \\
bc+d^2 =& 1
\end{align*}

%Fallunterscheidung \cases{ 1. $b=0$ \\ 2. $b\neq0$ }

1. b=0

$
a^2=1 \Rightarrow a=+-1 \\
c(a+d)=0 \\
a^2=1 \Rightarrow d=+-1 
$

1.1
$c=0 \\
a=+-1 \\
d=+-1 \\
$

1.2
$c\neq0$ \\
$a=-1$

2. $b\neq0$

$a+d=0 \\
d=-a$

$a^2+bc=1$ \\
$cb+a^2=1 = a^2+bc=1  \\
c=\frac{1-a^2}{b}$

$
A=\begin{pmatrix}
a & b \\
\frac{1-a^2}{b} & -a
\end{pmatrix}
$

%5.
	\item
Für welches $a \in \R$ ist folgendes Gleichungssystem eindeutig lösbar? Lösen Sie für diesen Fall unter Verwendung der \textsc{Cramer'schen} Regel.
\begin{align*}
		 ax + y & = 3 \\
		-2x + y & =-1
\end{align*}

\Lsg

\textsc{Cramer'sche Regel} \\
nur anwendbar wenn LGS eindeutig lösbar (d.h. $det A \neq 0$ ) $\Rightarrow A\vec{x}=\vec{b}$ \\
mit $A=\begin{pmatrix} a 1 \\ -2 1 \end{pmatrix}, \vec{x}=\begin{pmatrix} x \\ y \end{pmatrix}, \vec{b}=\begin{pmatrix} 3 \\ -1 \end{pmatrix}$ \\
$det A = a-2 \Rightarrow$ lösbar für $a \neq -2$ \\
$x= \frac{det A_1}{det A} = \frac{\begin{vmatrix} 3 & 1 \\ -1 & 1 \end{vmatrix}}{det A} = \frac{4}{a-2}$ \\
$y= \frac{det A_2}{det A} = \frac{\begin{vmatrix} a & 3 \\ -2 & -1 \end{vmatrix}}{det A} = \frac{-a+6}{a-2}$

%6.
\item 
Lösen Sie folgendes Gleichungssystem  unter Verwendung der \textsc{Cramer'schen} Regel.
	\begin{align*}
		-z			&= -y-2x+1 \\
		-x-y+z +y	&= y+2z+1 \\
		x+y			&= 2z+2
	\end{align*}

\Lsg

\begin{comment}
	\begin{align*}
	2x  +y  -z  & = 1 \\
	-x  -y  -z  & = 1 \\
	 x  +y  -2z & = 2 \\
	 \Rightarrow A=\begin{vmatrix}2 & 1 & -1 & 1 \\ -1 & -1 & -1 & 1 \\ 1 & 1 & -2 & 2\end{vmatrix} \\
	 det A = 4-1+1-(1-2+2) = 4-1 = 3
	\end{align*} 
\end{comment}

%\begin{comment}
erst umforme in Standardform

$
A=\begin{pmatrix}2 & 1 & -1 \\
-1 & -1 & -1 \\
1 & 1 & -2
\end{pmatrix}, \vec{b} = \begin{pmatrix} 1 \\ 1 \\ 2 \end{pmatrix}, vec{x} = \begin{pmatrix} x \\ y \\ z \end{pmatrix}$ \\
$det A = -1 \neq 0 \Rightarrow$ LGS eindeutlich lösbar $\Rightarrow$ CRAMERsche Regel ist anwendbar 

$x = \frac{\begin{pmatrix}
1 & 1 & -1 \\
1 & -1 & -1 \\
2 & 1 & -2
\end{pmatrix}}{-1} = 0$

$
y = \frac{\begin{pmatrix}
2  & 1 &  -1 \\
-1 & 1 & -1 \\
1  & 2 &  -2
\end{pmatrix}}{-1} = 0$

$z = \frac{\begin{pmatrix}
2 & 1 & 1 \\
-1 & -1 & 1 \\
1 & 1 & 2
\end{pmatrix}}{-1} = \frac{1}{-1} = -1$

%\end{comment}


%\item Berechnen Sie mit Hilfe des Gauß-Algorithmus, für welche reellen Zahlen $a,b$ das lineare Gleichungssystem
%
		%\begin{align*}
					 %x-2y+3z &= -4 \\
					%2x +y +z &=  2 \\
					%x + ay + 2z &= -b
		 %\end{align}
%
%\begin{enumerate}
%\item genau eine Lösung hat.
%\item keine Lösung hat.
%\item unendlich viele Lösungen hat.
%
%\end{enumerate}
%Geben Sie die für jeden Fall die Lösungen an.

\end{enumerate}
\end{document}
