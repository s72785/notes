\documentclass[12pt,a4paper]{scrreprt}
\usepackage{tikz}
\usetikzlibrary{positioning,calc}
\usepackage{amsmath,amssymb,mathrsfs,dsfont}
\usepackage[utf8]{inputenc}
\usepackage[ngerman]{babel}
\usepackage[colorlinks=false, pdfborderstyle={/S/U/W 1}
]{hyperref}
\usepackage{stmaryrd}
\usepackage{comment}
\oddsidemargin0mm \evensidemargin3mm \textwidth150mm \textheight23cm
\parindent0mm \pagestyle{empty} \topmargin-1cm
\newcommand{\E}{\mathds{E}}
\newcommand{\V}{\mathds{V}}
\newcommand{\C}{\mathds{C}}
\newcommand{\N}{\mathds{N}}
\newcommand{\Z}{\mathds{Z}}
\newcommand{\R}{\mathds{R}}
\pagestyle{headings}

%bilder einfuegen
\usepackage{multirow}
\usepackage{tikz}
\usetikzlibrary{calc,decorations.markings}


\newcommand{\Lsg}{\textbf{Lsg.:}}

\begin{document} 
\begin{flushleft}
\href{mailto:anja.voss-boehme@htw-dresden.de}{Prof. Dr. Anja Voß-Böhme} \\
\href{mailto:buder@htw-dresden.de}{Dipl.-Math. Thomas Buder}
\end{flushleft}

\begin{center}{\large\textbf Wirtschaftsmathematik I} \\ WS 2014/15 \end{center}

\begin{center}{\large\textbf{Übung 14}} 
\end{center}


\bigskip

\begin{enumerate}

%1.
\item

$1 + \Sigma_{l=1}^{\infty} 2 \cdot 0,8^l \ldots \\
 = 1 + 2 \cdot ( \Sigma_{n=0}^{\infty} \frac{4}{5}^n - 0,8^0 ) \\
= 1 + 2 \cdot ( \frac{n}{1-0,8} - 1 )$, \\
weil  $\Sigma_{n=0}^{\infty} q^n = \frac{1}{1-q}, |q| < 1 \\
= 1 + 2 \cdot ( 5 - 1 ) \\
= 9 [m]
$

$=
$

$
1m + 2 \cdot \frac{4}{5} + 2 \cdot \frac{4}{5}^2 \ldots
$


%2.

\item

\begin{enumerate}

\item


$f(x) = x$		Diagonale von links unte nach rechts oben

%[width=0.9\linewidth, height=0.95\textheight]
%\includegraphics{gph_betrag-x}\captionof{figure}{Kreis}\label{$f(x)=|x|$} 

$g(x) = |x|$		Spiegelung an x-Achse für $x<0$

$h(x) = -2x + 3$	Verschiebung von $x$ um $+3$ auf $y$ und senkt den Antrieg; $P1=(0,3), P2=(1,1)$

Monotonie

streng monoton fallend: $x_1 < _2 \Rightarrow f(x_1) < f(x_2)$

Nullstellen

$f: x_0=0$

$g: x_0=0$

$h: -2x+3=0 x= \frac{3}{2}$

Verhalten gegen Unentdlich

$f:	
x \rightarrow \infty$

$x \rightarrow -\infty$

$g:	
x \rightarrow \infty$

$x \rightarrow -\infty$


$h:	
x \rightarrow \infty$

$x \rightarrow -\infty$


\item

\begin{tabular}{llllll}
%						Monotonie		NSt						x \rightarrow \infty	& x \rightarrow -\infty	Scheitelpunkt	\\

$f(x)=x^2$				& nicht monoton	& $x_0=0$				& $\infty$				&	$\infty$				& \\
$g(x)=x^2-6$			& nicht monoton	& $x_{0,1}=+-\sqrt{6}$	& $\infty$				&	$\infty$			& \\
$h(x)=(x+2)^2+1$		& nicht monoton	& keine					& $\infty$				&	$\infty$				&	$-2,1$
\end{tabular}

\item

$f:$	kurve in sektor 2 und 3

$g:$	wie f nur aus sektor 3 in 1 gespiegelt

$h:$	wie f aber verschoben um 1 nach rechts

\begin{tabular}{llllll}
%						Monotonie		NSt					x \rightarrow \infty	x \rightarrow -\infty	Scheitelpunkt	\\
$f(x) = 1/x$			&str.m.fallend	&keine				&$0$					&$0$	\\
$g(x) = \frac{1}{x^2}$	&keine			&keine				&$0$					&$0$	\\
$h(x) = \frac{1}{x-1}$	&
\end{tabular}

\item

$f:	e^x			$

$g:	e^{-x}		$ Spiegelung f an y-Achse

$h:	ln(x-2)		$ Spiegelung an y=x (Umkehrfunktion von f), verschoben um 2 nach rechts

$0 = ln(x-2) \\
 e^{ln(x-2)} = e^0 \\
 x-2 = 1 \\
 x = 3$

\begin{tabular}{llllll}
%						&Monotonie		&NSt			&x \rightarrow \infty	&x \rightarrow -\infty	& Scheitelpunkt	\\
$f:$					&str.mon.steig.	&keine			&$\infty$				&$0$				\\
$g:$					&str.mon.fall.	&kein			&$0$					&$\infty$			\\
$h:$					&str.mon.wachs.	&$3$			&$\infty$				&$-\infty$
\end{tabular}

\end{enumerate}

%3.
\item

\begin{enumerate}
%a

\item
$lim_{x \rightarrow \infty} = \frac{x}{3x-7}	\\
=\frac{1}{3}$

\item
$lim_{x \rightarrow 1} = \frac{x}{3x-3}	\\
=-\infty von links, \infty von rechts	\\
= n.d. weil nicht eindeutig$

\item
$lim_{x \rightarrow -\infty} = \frac{x^5-3x^2}{2x^4+3x^2}	\\
= x \frac{1}{2}$

$  \frac{x^5-3x^2}{2x^4+3x^2}	\\
= \frac{(x^5)(1-3x^{-3})}{(x^4)(2+3x^{-2})}	\\
= \frac{x(1-3x^{-3})}{1(2+3x^{-2})}$


\item
$lim_{t \rightarrow \infty} = 1-\underbrace{e^{1-t}}_{=0}	\\
=1$


\item
$lim_{t \rightarrow 1} = 1 - e^{1-t}	$



\item
$lim_{x \rightarrow \infty} = e^{\overbrace{\frac{1}{\overbrace{t-1}{\infty}}}{0}}	\\
=1$

\end{enumerate}

%4.
\item
Min./Max

\begin{enumerate}

\item
$f(x)=x^2-5$

Min: $(0,-5)$

\item
$f(x)=\frac{1}{x^2+1}$

Max: $(0,1)$

\item

$f(x)=sin^2(x)$

$k \in \Z$ \\
Max: ${\pi}{2}+\pi k, k \in \Z$ \\
Min: $\pi k , k \in \Z$

\item

$\frac{1}{1+cos^2(x)}$

$k \in \Z$ \\
Min: $( k \pi, \frac{1}{2} )$ \\
Max: $( \frac{\pi}{2} + k \pi, 1 )$

\item



\end{enumerate}

\end{enumerate}


\end{document}
