\documentclass[12pt,a4paper]{scrreprt}

\usepackage{amsmath,amssymb,mathrsfs,dsfont}
\usepackage[utf8]{inputenc}
\usepackage[ngerman]{babel}
\usepackage[colorlinks=false, pdfborderstyle={/S/U/W 1}
]{hyperref}
\usepackage{comment}
\oddsidemargin0mm \evensidemargin3mm \textwidth150mm \textheight23cm
\parindent0mm \pagestyle{empty} \topmargin-1cm
\newcommand{\E}{\mathds{E}}
\newcommand{\V}{\mathds{V}}
\newcommand{\C}{\mathds{C}}
\newcommand{\N}{\mathds{N}}
\newcommand{\Z}{\mathds{Z}}
\newcommand{\R}{\mathds{R}}
\pagestyle{headings}
\begin{document} 
\begin{flushleft}
\href{mailto:anja.voss-boehme@htw-dresden.de}{Prof. Dr. Anja Voß-Böhme} \\
\href{mailto:buder@htw-dresden.de}{Dipl.-Math. Thomas Buder}
\end{flushleft}

\begin{center}
\large{\textbf{ Wirtschaftsmathematik I}} \\
WS 2014/15 \end{center}

\begin{center}\large{\textbf{ Übung 4 }} \end{center}

\bigskip
\begin{enumerate}
\item Übungsblatt 3, Aufgabe 3 c), d)
\item $lorem ipsum lrem capula tatep lorem ipsum lrem capula tatep lorem ipsum lrem capula tatep lorem ipsum lrem capula tatep lorem ipsum lrem capula tatep lorem ipsum lrem capula tatep lorem ipsum lrem capula tatep lorem ipsum lrem capula tatep lorem ipsum lrem capula tatep lorem ipsum lrem capula tatep lorem ipsum lrem capula tatep lorem ipsum lrem capula tatep lorem ipsum lrem capula tatep lorem ipsum lrem capula tatep lorem ipsum lrem capula tatep \underbrace{Skizzieren}_{ein sehr langer text verschiebt die umgebung} lorem ipsum lrem capula tatep lorem ipsum lrem capula tatep lorem ipsum lrem capula tatep lorem ipsum lrem capula tatep lorem ipsum lrem capula tatep lorem ipsum lrem capula tatep lorem ipsum lrem capula tatep lorem ipsum lrem capula tatep lorem ipsum lrem capula tatep lorem ipsum lrem capula tatep lorem ipsum lrem capula tatep lorem ipsum lrem capula tatep lorem ipsum lrem capula tatep lorem ipsum lrem capula tatep lorem ipsum lrem capula tatep  $ Sie $graph(f)$ für folgenden Funktionen, geben Sie  - falls vorhanden - die Umkehrfunktion $y=f^{-1}(x)$ an,  deren Wertebereich und Definitionsbereich und skizzieren Sie $f^{-1}(x)$.
		\begin{enumerate}
			\item $f(x)=x^3$ für $x \in \R$
			\item $f(x)= \begin{cases}
													(x+2)^2+1 \text{ für  } x \in [-2,1] \\
													2x+8 \text{ für  } x \in (1,3]
										\end{cases}$
			\end{enumerate}
\item Beweisen Sie mittels vollständiger Induktion, dass für alle $n \in \N$ \[ \sum\limits_{k=1}^n (2k-1) = n^2.\]
\item Berechnen Sie die Lösungsmengen der folgenden (Un-)Gleichungen.
			\begin{enumerate}
			  \item $\frac{x-2}{4+2x} < 1,  x \neq -2$
				\item $|x-7| = 5$
				\item $|x-2| + |x+1| = 5$
				\item $|x - 2| - |x + 3| \leq 8$
				\item $|4 - x| - |x + 1| \leq 2$


			\end{enumerate}
\item Es werden sieben verschiedene Familiennamen in eine Liste eingetragen. Auf wie viele
verschiedene Arten der Reihenfolge ist das möglich?
\item Wie viele verschiedene Möglichkeiten gibt es für den Code eines 5-ziffrigen Fahrradschlosses?
\item Wie viele Möglichkeiten gibt es, von fünf verschiedenen Schranktüren je eine mit den
Farben rot, weiß, grün, blau oder gelb zu streichen?
\item Ein Ausschuss aus vier Frauen und fünf Männern wählt zwei Vertreter. Wie viele
Möglichkeiten gibt es
 \begin{enumerate}
				\item insgesamt?
				\item so zu wählen, dass beide gleichen Geschlechts sind?
				\item so zu wählen, dass beide unterschiedlichen Geschlechts sind?
 \end{enumerate}
\item Begründen Sie, dass die Potenzmenge einer endlichen Menge $M$ genau $2^{|M|}$ Elemente besitzt.
\item Auf wie viele Arten kann man aus einer Gruppe mit 10 Jungen und 12 Mädchen
fünf Kinder so auswählen, dass 2 Jungen und 3 Mädchen dabei sind ?
\item In einem Eiscafe werden sechs verschiedene Eissorten angeboten, ein Kind möchte drei
Kugeln kaufen. Wie viele Möglichkeiten gibt es für die Auswahl, wenn eine Sorte
nicht mehrfach gewählt werden darf und die Reihenfolge der Kugeln egal ist?
 \end{enumerate}

\end{document}
