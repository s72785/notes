\documentclass[12pt,a4paper]{scrreprt}

\usepackage{amsmath,amssymb,mathrsfs,dsfont}
\usepackage[latin1]{inputenc}
\usepackage[ngerman]{babel}
\oddsidemargin0mm \evensidemargin3mm \textwidth150mm \textheight23cm
\parindent0mm \pagestyle{empty} \topmargin-1cm
\newcommand{\E}{\mathds{E}}
\newcommand{\V}{\mathds{V}}
\newcommand{\C}{\mathds{C}}
\newcommand{\N}{\mathds{N}}
\newcommand{\Z}{\mathds{Z}}
\newcommand{\R}{\mathds{R}}
\pagestyle{headings}
\begin{document} 
\begin{flushleft}
Prof. Dr.  Anja Vo�-B�hme \\
Dipl.-Math. Thomas Buder
\end{flushleft}

\begin{center}{\large\bf Wirtschaftsmathematik I} \\ \\ WS 2014/15 \end{center}

\begin{center}{\large\bf �bung 4 } 
\end{center}


\bigskip
\begin{enumerate}
\item �bungsblatt 3, Aufgabe 3 c), d)
\item Skizzieren Sie $graph(f)$ f�r folgenden Funktionen, geben Sie  - falls vorhanden - die Umkehrfunktion $y=f^{-1}(x)$ an,  deren Wertebereich und Definitionsbereich und skizzieren Sie $f^{-1}(x)$.
		\begin{enumerate}
			\item $f(x)=x^3$ f�r $x \in \R$
			\item $f(x)= \begin{cases}
													(x+2)^2+1 \text{ f�r  } x \in [-2,1] \\
													2x+8 \text{ f�r  } x \in (1,3]
										\end{cases}$
			\end{enumerate}
\item Beweisen Sie mittels vollst�ndiger Induktion, dass f�r alle $n \in \N$ \[ \sum\limits_{k=1}^n (2k-1) = n^2.\]
\item Berechnen Sie die L�sungsmengen der folgenden (Un-)Gleichungen.
			\begin{enumerate}
			  \item $\frac{x-2}{4+2x} < 1,  x \neq -2$
				\item $|x-7| = 5$
				\item $|x-2| + |x+1| = 5$
				\item $|x - 2| - |x + 3| \leq 8$
				\item $|4 - x| - |x + 1| \leq 2$


			\end{enumerate}
\item Es werden sieben verschiedene Familiennamen in eine Liste eingetragen. Auf wie viele
verschiedene Arten der Reihenfolge ist das m�glich?
\item Wie viele verschiedene M�glichkeiten gibt es f�r den Code eines 5-ziffrigen Fahrradschlosses?
\item Wie viele M�glichkeiten gibt es, von f�nf verschiedenen Schrankt�ren je eine mit den
Farben rot, wei�, gr�n, blau oder gelb zu streichen?
\item Ein Ausschuss aus vier Frauen und f�nf M�nnern w�hlt zwei Vertreter. Wie viele
M�glichkeiten gibt es
 \begin{enumerate}
				\item insgesamt?
				\item so zu w�hlen, dass beide gleichen Geschlechts sind?
				\item so zu w�hlen, dass beide unterschiedlichen Geschlechts sind?
 \end{enumerate}
\item Begr�nden Sie, dass die Potenzmenge einer endlichen Menge $M$ genau $2^{|M|}$ Elemente besitzt.
\item Auf wie viele Arten kann man aus einer Gruppe mit 10 Jungen und 12 M�dchen
f�nf Kinder so ausw�hlen, dass 2 Jungen und 3 M�dchen dabei sind ?
\item In einem Eiscafe werden sechs verschiedene Eissorten angeboten, ein Kind m�chte drei
Kugeln kaufen. Wie viele M�glichkeiten gibt es f�r die Auswahl, wenn eine Sorte
nicht mehrfach gew�hlt werden darf und die Reihenfolge der Kugeln egal ist?
 \end{enumerate}

\end{document}
